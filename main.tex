\documentclass[openany]{book}
\usepackage[italian]{babel}
% ... altri pacchetti ...
\usepackage[
    top=2cm,
    bottom=2cm,
    left=2cm,
    right=2cm,
    headheight=14pt % Aggiungi questo se usi intestazioni, altrimenti LaTeX potrebbe lamentarsi.
]{geometry}
% Imposta la profondità dell'indice:
% Aumenta la porzione di pagina che può essere occupata dalle figure in alto (0.9 = 90%)
\renewcommand{\topfraction}{0.9} 
% Permette che la figura occupi almeno il 90% dello spazio superiore
\renewcommand{\textfraction}{0.1} 
% Permette che resti solo il 10% di testo nella pagina superiore
\setcounter{topnumber}{2} 
% Permette 2 figure in cima alla pagina
\usepackage{graphicx}
\usepackage{fontspec}
\usepackage{xcolor}
\usepackage{titlesec}
\usepackage{float}
\usepackage{caption}
\usepackage{mathtools}
\usepackage{pagecolor} % Necessario per \pagecolor
\usepackage{chngcntr} % <-- PACCHETTO AGGIUNTO
\usepackage{amsmath}
\usepackage{array}
\usepackage{fancyhdr}
% Definisce un nuovo tipo di colonna chiamato 'C' (maiuscolo) che centra orizzontalmente e verticalmente.
\newcolumntype{C}[1]{>{\centering\arraybackslash}m{#1}}


\usepackage{makecell} 
% \renewcommand{\arraystretch}{1.5}
\tolerance=3000
% Definizione Colori
\definecolor{frontespizio}{HTML}{7DF9FF}
\definecolor{normale}{HTML}{FFFFFF}
\AtBeginDocument{\fontsize{11}{12}\selectfont}

\usepackage{etoolbox}
\makeatletter
\patchcmd{\listoffigures}{\clearpage}{}{}{} % Rimuove \clearpage da listoffigures
\makeatother

%\setmainfont{Arial} 
\graphicspath{{immagini/}}

% Modulo (PART) - Come nell'immagine 1
\titleformat{\part}[display]
  {\normalfont\Large\bfseries\centering\fontsize{30}{20}\selectfont} % Formato del titolo
  {\MakeUppercase{\partname}\ \thepart:} % Etichetta (es. MODULO 1)
  {10pt} % Spazio tra etichetta e titolo
  {\MakeUppercase} % Trasforma il titolo in maiuscolo
  [\vspace{1ex}] % Spazio aggiuntivo dopo il titolo (opzionale)

% Capitolo (CHAPTER) - Come nell'immagine 2
\titleformat{\chapter}[display]
  {\normalfont\Huge\bfseries\fontsize{18}{20}\selectfont} % Formato del titolo
  {\chaptertitlename\ \thechapter:} % Etichetta (es. Capitolo 1)
  {1em} % Spazio tra etichetta e titolo
  {} 

% Sezione (SECTION) - Come nell'immagine 2 (1.1)
\titleformat{\section}
  {\normalfont\Large\bfseries\fontsize{15}{17}\selectfont} % Formato del titolo
  {\thesection:} % Etichetta (es. 1.1)
  {1em} % Spazio orizzontale
  {}

% Sottosezione (SUBSECTION) - Come nell'immagine 2 (1.1.1)
\titleformat{\subsection}
  {\normalfont\large\bfseries\itshape\fontsize{12}{14}\selectfont} % Formato del titolo
  {\thesubsection:} % Etichetta (es. 1.1.1)
  {1em} % Spazio orizzontale
  {} % Non fare nulla al titolo
\counterwithout{figure}{chapter}
\counterwithout{chapter}{part} 
\setcounter{tocdepth}{2}

\begin{document}

% ----------------------------------------------------
% INIZIO MATERIA PRELIMINARE (Numerazione Romana)
% ----------------------------------------------------

% 1. IMPOSTA LA SEZIONE ROMANA (DEVE ESSERE PRIMA)
\frontmatter 

% 2. IMPOSTA IL CONTATORE A -1 (PER PAGINA 0)
% La pagina successiva (\begin{titlepage}) sarà la pagina -1 + 1 = 0.
\setcounter{page}{-1} 

% 3. IMPOSTA IL COLORE DI SFONDO PER LA PAGINA 0
\pagecolor{frontespizio} 

% --- Frontespizio Personalizzato (Pagina 0) ---
\begin{titlepage}
    
    \thispagestyle{empty} % Nasconde il numero 0.
    
    \centering    
    \includegraphics[width=11cm]{sfondo} 
    \vspace{0.5cm}
    
    % --- TITOLO E DATI ---
    {\Huge\bfseries Laboratorio di multimedialità \par}
    {\large A.A. 2024-25 \par}
    {\large Andrea Macale \par}
    
    % --- Spazio in fondo ---
    \vfill
    
\end{titlepage}

% ----------------------------------------------------
% --- INDICE (Inizia con Pagina i) ---

% Cambia il colore di sfondo per l'indice
\pagecolor{normale}
\setcounter{page}{1}
\pagestyle{plain} % Mostra il numero in fondo alla pagina (i, ii, iii...)
\tableofcontents
\listoffigures

% ----------------------------------------------------
% INIZIO CORPO PRINCIPALE (Numerazione Araba da 1)
\mainmatter 
\pagestyle{fancy}
\fancyhf{} % Pulisce gli header e i footer

% Intestazioni (Headers)
% Pagina PARI (Sinistra): Titolo della Sezione/Capitolo
\fancyhead[LE]{\nouppercase{\rightmark}} 
% Pagina DISPARI (Destra): Titolo della Part/Capitolo
\fancyhead[RO]{\nouppercase{\leftmark}} 

% Piè di Pagina (Footers) - Numerazione alternata
\fancyfoot[LE,RO]{\thepage} 

\renewcommand{\headrulewidth}{0.4pt} % Aggiunge una linea orizzontale
\renewcommand{\footrulewidth}{0pt} % Rimuove la linea dal piè di pagina

% Inizia la numerazione araba da 1 di default.

\part{Immagini}
\chapter{Elaborazione delle immagini}
\input{capitoli/cap1}
\chapter{Filtri nel dominio spaziale}
\input{capitoli/cap2}
\chapter{Filtri nel dominio della frequenza}
Fino ad ora, sono state trattate le immagini nel dominio spaziale, considerandole come funzioni a due variabili $f(x,y)$. Questa rappresentazione è sicuramente molto comoda ed intuitiva, poiché nelle immagini in bianco e nero, non è altro che una matrice $M\to N$ dove ogni elemento corrisponde al livello di grigio di un pixel. Tuttavia, tale rappresentazione diventa decisamente meno comoda nel momento in cui si eseguono delle operazioni di filtraggio, perché l'operazione di convoluzione risulta decisamente costosa in termini computazionali.
\section{Dominio della frequenza}
Per risolvere tale problema, si passa nel dominio della frequenza. L'idea di base, è:
\begin{itemize}
    \item una funzione periodica si può riscrivere in serie di Fourier come somma di seni e coseni moltiplicati per coefficienti distinti;
    \begin{equation*}
        f(t)=\sum_{i=-\infty}^{\infty}c_n \exp\left(j2\pi\frac{nt}{T}\right) : f(t)=f(t-T)
    \end{equation*}
    \item una funzione non periodica si può rappresentare in frequenza come l'integrale di seni e coseni moltiplicati per una funzione pesata.
    \begin{equation*}
        F(\nu)=\int_{-\infty}^{\infty}f(t)\exp\left(-j2\pi t \nu\right) \, dt
    \end{equation*}
\end{itemize}
\begin{figure}[H]
        \centering
        \includegraphics[width=0.6\textwidth]{cap3/tf} 
        \caption{Rappresentazione della trasformata di Fourier} 
        \label{fig:tf}
\end{figure}

\subsection{Trasformata di Fourier in due dimensioni}
Passando in un due dimensioni, le frequenze sono due:
\begin{itemize}
    \item frequenza spaziale lungo l'asse $x$, definita come $u$;
    \item frequenza spaziale lungo l'asse $y$, definita come $v$.
\end{itemize}
Come si può intuire facilmente, se la funzione pesata in una dimensione ha due variabili, in due dimensione ne servono ben quattro: infatti $x$ e $y$ saranno gli indici di sommatoria, stessa cosa per l'antitrasformata che saranno $u$ e $v$.
\begin{equation*}
        F(u,v)=\sum_{x=0}^{M-1}\sum_{y=0}^{N-1}f(x,y)\exp\left[-j2\pi\left(\frac{ux}{M}+\frac{vy}{N}\right)\right]
    \end{equation*}
\begin{equation*}
     f(x,y)=\frac{1}{MN}\sum_{u=0}^{M-1}\sum_{v=0}^{N-1}F(u,v)\exp\left[j2\pi\left(\frac{ux}{M}+\frac{vy}{N}\right)\right]
\end{equation*}
Il valore della trasformata di Fourier alla frequenza di origine prende il nome di componente DC: esso si calcola facendo la media tra il prodotto dell'intensità di $f(x,y)$ e un fattore $MN$. Infatti, risulta molto comune eseguire una traslazione della componente DC sul punto $(M/2, N/2)$.
\begin{equation*}
        F(u,v)=\sum_{x=0}^{M-1}\sum_{y=0}^{N-1}\left[f(x,y)+(1)^{x+y}\right]\exp\left[-j2\pi\left(\frac{ux}{M}+\frac{vy}{N}\right)\right]
\end{equation*}
\begin{figure}[htbp]
        \centering
        \includegraphics[width=0.9\textwidth]{cap3/frequenza} 
        \caption{Immagine nel dominio spaziale (a sinistra) ed in frequenza (a destra)} 
        \label{fig:frequenza}
\end{figure}
\subsection{Proprietà della trasformata di Fourier}
La trasformata di Fourier gode della seguenti proprietà:
\begin{itemize}
    \item proprietà di traslazione, che afferma che se l'immagine è traslata nello spazio, in frequenza sarà traslata solo nella fase ma non in ampiezza;
    \begin{equation*}
        F[f(x-x_0,y-y_0)]=F(u,v)\exp\left[-j2\pi\left(\frac{ux_0}{M}+\frac{vy_0}{N}\right)\right]
    \end{equation*}
    \item linearità;
    \begin{equation*}
        F[af_1(x,y)+bf_2(x,y)]=aF_1(u,v)+bF_2(u,v)
    \end{equation*}
    \item proprietà di rotazione, che afferma che se l'immagine è ruotata di un angolo $\theta$, anche in frequenza sarà ruotata dello stesso angolo;
    \item separabilità, ossia si può eseguire prima la trasformata sulle righe e poi sulle colonne (o viceversa).
\end{itemize}
\subsection{Teorema della convoluzione e applicazione al filtraggio}
La potenzialità del passaggio dal dominio spaziale al dominio della frequenza sta nel teorema della convoluzione. Tale teorema afferma che eseguire una convoluzione nel dominio spaziale corrisponde ad eseguire un prodotto nel dominio della frequenza, con un costo computazione molto meno elevato.
\begin{center}
    $f(x,y)\ast h(x,y) \longleftrightarrow F(u,v)H(u,v)$
\end{center}
Ciò consiste nel trasformare l'immagine originale in frequenza; moltiplicarla per il filtro in frequenza, denominato anche funzione di trasferimento; e trasformare l'immagine filtrata nel dominio spaziale.In particolare, si fa riferimento a due tipologie di frequenza:
\begin{itemize}
    \item le basse frequenze, che rappresentano le aree omogenee;
    \item le alte frequenze, che rappresentano i dettagli.
\end{itemize}
\begin{figure}[htbp]
        \centering
        \includegraphics[width=0.9\textwidth]{cap3/schema} 
        \caption{Filtraggio nel dominio della frequenza} 
        \label{fig:schemaFrequenza}
\end{figure}
\section{Filtro di Notch}
Il filtro più semplice da implementare è il filtro di Notch, chiamato anche filtro elimina-banda, che consiste nel far passare tutte le frequenze, eccetto per una tacca (notch in inglese), che corrisponde alla componente DC.
\begin{equation*}
        H(u,v) = 
        \begin{dcases}
               0 & \text{se } (u,v) = \left(\frac{M}{2},\frac{N}{2}\right) \\[1.5ex]
                1 & \text{se } (u,v) \ne \left(\frac{M}{2},\frac{N}{2}\right)
            \end{dcases}
    \end{equation*}
\begin{figure}[htbp]
        \centering
        \includegraphics[width=0.65\textwidth]{cap3/notch} 
        \caption{Immagine originale (a sinistra) e immagine filtrata con un filtro Notch (a destra)} 
        \label{fig:notch}
\end{figure}
\section{Filtri passa-basso}
I filtri passa-basso (LPF, dall'inglese low-pass filter) attenuano le alte frequenze, applicando quindi l'effetto di sfocatura e riduzione del rumore. Infatti, corrispondono ai filtri di smussamento nel dominio spaziale. Innanzitutto, si considera la distanza euclidea in frequenza $D(u,v)$, calcolata come segue.
\begin{equation*}
    D(u,v)=\sqrt{\left(u-\frac{M}{2}\right)^2+\left(v-\frac{N}{2}\right)^2}
\end{equation*}
Inoltre, si considera la cosiddetta frequenza di cutoff ($D_0$), che cambia il comportamento del filtro in base a tale valore.
\subsection{Filtro passa-basso ideale}
Il filtro passa-basso ideale fa passare solamente le frequenze la cui distanza euclidea è minore od al più uguale alla frequenza di cutoff.
\begin{equation*}
        H(u,v) = 
        \begin{dcases}
               1 & \text{se } D(u,v) \le D_0 \\
               0 & \text{se } D(u,v) > D_0
            \end{dcases}
\end{equation*}
\begin{figure}[htbp]
        \centering
        \includegraphics[width=0.5\textwidth]{cap3/idealeL} 
        \caption{Funzionamento di un filtro passo-basso ideale} 
        \label{fig:LPF_ideale}
\end{figure}
In particolare, maggiore sarà il cutoff e minore sarà la sfocatura, poiché passarranno più frequenze.
\begin{figure}[htbp]
        \centering
        \includegraphics[width=0.6\textwidth]{cap3/LPF_ideale} 
        \caption{Da sinistra: immagine originale, LPF ideale $D_0=8$ e LPF ideale $D_0=16$} 
        \label{fig:LPF_ideale2}
\end{figure}
\subsection{Filtro passa-basso di Butterworth}
Il filtro passa-basso di Butterworth permette di eseguire una sfocatura molto meno intensa rispetto a quella ideale, ma neanche ai livello del filtro Gaussiano. Oltre alla distanza euclidea ed alla frequenza di cutoff, si aggiunge un parametro $n$, che indica l'ordine del filtro, ovvero la sua ripidità: infatti, maggiore è l'ordine, più si avvicina ad un filtro passa basso ideale.
\begin{equation*}
        H(u,v) = 
        \frac{1}{1+\left[\frac{D(u,v)}{D_0}\right]^{2n}}
\end{equation*}
\begin{figure}[htbp]
        \centering
        \includegraphics[width=0.9\textwidth]{cap3/butterworth} 
        \caption{Vari ordini del filtro passa-basso di Butterworth} 
        \label{fig:LPF_ideale2}
\end{figure}
\subsection{Filtro passa-basso Gaussiano}
Il filtro passa-basso Guassiano è in grado di attenuare le alte frequenze con poca intensità. In particolare, il cutoff corrisponde alla deviazione standard della gaussiana.
\begin{equation*}
        H(u,v) = 
        \exp{\left[-\frac{D^2(u,v)}{2D^2_0}\right]}
\end{equation*}
\begin{figure}[htbp]
        \centering
        \includegraphics[width=0.9\textwidth]{cap3/gaussiana} 
        \caption{Filtri passa-basso Gaussiani} 
        \label{fig:LPF_ideale2}
\end{figure}
\section{Filtri passa-alto}
I filtri passa-alto (HPF, dall'inglese high-pass filter) attenuano le basse frequenze e risaltano le alte frequenze, applicando quindi l'effetto di sfocatura e riduzione del rumore. Infatti, corrispondono ai filtri di nitidezza nel dominio spaziale. Anche in questo caso, si ha la distanze euclidea e la frequenza di cutoff.\\
Inoltre, la funzione di trasferimento di un filtro passo-alto è l'inverso di quello del filtro passa-basso.
\begin{equation*}
        H_{HPF}(u,v) = 1 - H_{LPF}(u,v)
\end{equation*}
\subsection{Filtro passa-alto ideale, di Butterworth e Gaussiano}
Di seguito, sono riportate le formule dei filtri passa-alto ideale, di Butterworth e Gaussiano, che presentano le proprietà inverse di quelle del passo-basso.
\begin{center}
\begin{minipage}{0.06\textwidth} % Colonna Sinistra (48% della larghezza del testo)
    \centering
    \begin{equation*}
        H(u,v) = 
        \begin{dcases}
               0 & \text{se } D(u,v) \le D_0 \\
               1 & \text{se } D(u,v) > D_0
            \end{dcases}
    \end{equation*}
\end{minipage}
\hfill % Spazio elastico per separare le due colonne
\begin{minipage}{0.06\textwidth} % Colonna Destra (48% della larghezza del testo)
    \centering
    \begin{equation*}
        H(u,v) = 
        \frac{1}{1+\left[\frac{D_0}{D(u,v)}\right]^{2n}}
    \end{equation*}
\end{minipage}
\hfill % Spazio elastico per separare le due colonne
\begin{minipage}{0.30\textwidth} % Colonna Destra (48% della larghezza del testo)
    \centering
    \begin{equation*}
        H(u,v) = 
        1-\exp{\left[-\frac{D^2(u,v)}{2D^2_0}\right]}
    \end{equation*}
\end{minipage}
\end{center}
\begin{figure}[htbp]
        \centering
        \includegraphics[width=0.85\textwidth]{cap3/HPF} 
        \caption{Da alto a sinistra senso orario: originale, HPF ideale, HPF di Butterworth e HPF Gaussiano} 
        \label{fig:HPF}
\end{figure}
\subsection{Il Laplaciano, i filtri di contrasto e l'high-boost filtering}
Nel dominio spaziale, il Laplaciano presenta la formula seguente:
\begin{equation*}
        g(x,y)=f(x,y)-\nabla^2f(x,y)
\end{equation*}
dove:
\begin{equation*}
        \nabla^2f=\frac{\delta^2 f}{\delta x^2}+\frac{\delta^2 f}{\delta y^2}
\end{equation*}
In frequenza, la trasformata di una Fourier della derivata è la seguente:
\begin{equation*}
    F\left[ \frac{d^2 f}{d t^2} \right]=-\nu^2F(\nu)
\end{equation*}
che in due dimensioni e per le immagini, corrisponde:
\begin{equation*}
    F\left[\nabla^2f\right]=-(u^2+v^2)F(u,v)
\end{equation*}
Perciò, l'immagine trasformata in frequenza si ottiene nella maniera che segue:
\begin{equation*}
    G(u,v)=F(u,v)+(u^2+v^2)F(u,v)=[1+(u^2+v^2)]F(u,v)
\end{equation*}
da cui si ricava la funzione di trasferimento.
\begin{equation*}
    H(u,v)=1+(u^2+v^2)
\end{equation*}
Infine, per quanto riguarda i filtri di contrasto e l'high-boost filtering, le formule sono rispettivamente le seguenti.
\begin{center}
\begin{minipage}{0.48\textwidth} % Colonna Sinistra (48% della larghezza del testo)
    \centering
    \begin{equation*}
        H(u,v) = 1 - H_{lp}(u,v)
    \end{equation*}
\end{minipage}
\hfill % Spazio elastico per separare le due colonne
\begin{minipage}{0.48\textwidth} % Colonna Destra (48% della larghezza del testo)
    \centering
    \begin{equation*}
         H(u,v) = (A-1) + H_{lp}(u,v)
    \end{equation*}
\end{minipage}
\end{center}
\begin{figure}[htbp]
        \centering
        \includegraphics[width=0.9\textwidth]{cap3/Laplaciano} 
        \caption{Immagine originale (a sinistra) e maschera del filtro Laplaciano (a destra)} 
        \label{fig:HPF}
\end{figure}
\vfill
\chapter{Trasformata Wavelet}
In questo capitolo verranno mostrati i limiti della trasformata di Fourier, per poi introdurre la trasformata Wavelet, le sue potenzialità e le possibili applicazioni.
\section{Limiti della trasformata di Fourier e possibile soluzione}
Per comprendere bene i limiti della trasformata di Fourier, è necessario analizzare bene le caratteristiche di un'immagine.
\subsection{Caratteristiche di un'immagine}
Un'immagine si può vedere come un'insieme di regioni di texture e livelli di grigio simili, che combinati tra loro formano oggetti. In particolare, si considera:
\begin{itemize}
    \item l'analisi a bassa risoluzione, che analizza gli oggetti ad alto contrasto, quindi le basse frequenze;
    \item l'analisi ad alta risoluzione, che analizza gli oggetti a basso contrasto, perciò le alte frequenze;
    \item l'analisi multirisoluzione, che analizza gli oggetti a contrasto variabile.
\end{itemize}
Da qui, si nota che l'immagine è un segnale non stazionario: per prima conviene analizzare la figura seguente.
\begin{figure}[htbp]
        \centering
        \includegraphics[width=1\textwidth]{cap4/asse_tempo} 
        \caption{Variazione delle frequenze nell'asse temporale da $t_0=0\,s$ a $t_3=1\,s$} 
        \label{fig:asse_tempo}
\end{figure}
Come si può osservare:
\begin{itemize}
    \item da $t_0=0\,s$ a $t_1=0,3\,s$ si hanno le basse frequenze, in cui il segnale oscilla molto lentamente;
    \item da $t_1=0,3\,s$ a $t_2=0,85\,s$ si hanno le medie frequenze, dove le oscillazioni diventano sempre più ravvicinate e veloci;
    \item da $t_2=0,85\,s$ a $t_3=1\,s$ si hanno le alte frequenze, dove il segnale oscilla molto velocemente.
\end{itemize}
Perciò, le immagini hanno le statistiche che variano localmente: per questo motivo costruire un modello statistico che copre l'intera immagine, risulta decisamente molto complesso.
\subsection{Problemi della trasformata di Fourier}
Tornando alla figura precedente, se si eseguisse la trasformata di Fourier di quel segnale, essa darebbe come unica informazione la presenza di una bassa, media ed alta frequenza, senza fornire la loro posizione nel tempo. Generalizzando, eseguendo la trasformata di Fourier su un'immagine, si perdono completamente le informazioni temporali e spaziali.
\begin{figure}[htbp]
        \centering
        \includegraphics[width=0.9\textwidth]{cap4/problemi_fourier.png} 
        \caption{Rappresentazione dei problemi della trasformata di Fourier} 
        \label{fig:problemi_tempo}
\end{figure}
\subsection{Una soluzione temporanea: la trasformata di Fourier a breve termine}
La trasformata di Fourier a breve termine (STFT, dall'inglese Short Time Fourier Transform) è una tecnica fondamentale per analizzare come il contenuto in frequenza di un segnale cambia nel tempo. Ciò consiste in:
\begin{center}
\begin{minipage}{0.46\textwidth} % Colonna Sinistra (48% della larghezza del testo)
    \begin{enumerate}
    \item si definisce una finestra di tempo di una dimensione fissa;
    \item si applica la trasformata di Fourier al segnale finestrato;
    \item si memorizzano le frequenze trovate in quel intervallo di tempo;
    \item si scorre la finestra leggermente avanti nel tempo e si ripete le analisi, finché non finisce il segnale, costruendo una mappa.
\end{enumerate}
\end{minipage}
\hfill % Spazio elastico per separare le due colonne
\begin{minipage}[c]{0.5\textwidth} % Colonna Destra (48% della larghezza del testo)
    \centering
        \centering
        \includegraphics[width=1\textwidth]{cap4/stft} 
        \captionof{figure}{Esecuzione della STFT} 
        \label{fig:stft}
\end{minipage}
\end{center}
Come si può facilmente intuire, la scelta della finestra risulta cruciale, poiché più la finestra è piccola, maggiore sarà la risoluzione temporale. Tuttavia, se si prende una finestra troppo piccola, si rischia che le frequenze rappresentabili siano insufficienti, tanto sarà debole il potenziale di discriminazione delle frequenze.
\section{Introduzione alla trasformata Wavelet}
La trasformata Wavelet si basa su piccole onde, dette appunto wavelet, che godono delle seguenti tre proprietà:
\begin{itemize}
    \item frequenza ed ampiezza variabile;
    \item durata limitata, perciò non devono essere sinusoidi;
    \item valor medio nullo.
\end{itemize}
Di seguito, è riportata una tabella che mostra le wavelet più utilizzate.
\begin{table}[H] % Usa [H] per renderla non flottante
    \centering
    
    % Definizione delle 3 colonne: USIAMO C (Centrato orizzontalmente)
    \begin{tabular}{| C{1.5cm} | C{6cm} | C{8cm} |} 
        \hline
        
        % Usiamo \thead per le intestazioni:
        \textbf{Nome} & \textbf{Formula analitica} & \textbf{Grafico} \\
        \hline % Linea spessa sotto le intestazioni
        
        % --- RIGA 1: SHANNON ---
        Shannon &  
        $\psi(t) = 2\operatorname{sinc}(2t)-\operatorname{sinc}(t)$ &  
        \includegraphics[width=7.5cm]{cap4/shannon} \\
        \hline
        
        % --- RIGA 2: MORLET ---
        Morlet &  
        $\psi(t) = \exp\left(-\frac{t^2}{2}\right)\cos{(5t)}$ &  
        \includegraphics[width=7.5cm]{cap4/morlet} \\
        \hline
        
        % --- RIGA 3: HAAR ---
        Haar &  
        $\psi(t) = \operatorname{rect}_{1/2}\left(t-\frac{1}{4}\right) - \operatorname{rect}_{1/2}\left(t-\frac{3}{4}\right) $ &  
        \includegraphics[width=7.5cm]{cap4/haar} \\
        \hline % Linea spessa finale
    \end{tabular}
    
    % --- PUNTO CRUCIALE: USA \captionof{figure} ---
    \captionof{figure}{Tipologie wavelet fondamentali}
    \label{fig:wavelet_fondamentali}
\end{table}
\subsection{Determinazione della trasformata Wavelet}
A questo punto, ogni segnale (non necessariamente periodico), può essere scritto in serie di wavelet.
\begin{equation*}
    f(t)=\sum_{i}a_i\psi_i(t)
\end{equation*}
$\psi_{s,\tau}$ è così definito:
\begin{equation*}
    \psi_{s,\tau}(t)=\frac{1}{\sqrt{s}}\psi\left(\frac{t-\tau}{s}\right)
\end{equation*}
dove:
\begin{itemize}
    \item $s$ è la scalatura;
    \item $\tau$ è la traslazione nel tempo;
    \item $\frac{1}{\sqrt{s}}$ è la normalizzazione;
    \item $\psi$ è la wavelet madre;
    \item $\psi_{s,\tau}$ è la wavelet scalata e traslata nel tempo.
\end{itemize}
Adesso, si possono ricavare i coefficienti della trasformata Wavelet.
\begin{equation*}
    \gamma(s,\tau)=\int_{-\infty}^{\infty}f(t)\psi^{*}_{s,\tau}(t) \, dt
\end{equation*}
Una volta eseguito ciò, è possibile ricostruire il segnale con la formula seguente.
\begin{equation*}
    f(t)=\int_{-\infty}^{\infty}\int_{-\infty}^{\infty}\gamma(s,\tau)\psi_{s,\tau}(t) \, d\tau ds
\end{equation*}
Nella figura che segue, viene mostrata una rappresentazione di una possibile trasformata Wavelet, partendo da un segnale $f(t)$.
\begin{figure}[htbp]
        \centering
        \includegraphics[width=1\textwidth]{cap4/wavelet} 
        \caption{Rappresentazione di una trasformata Wavelet} 
        \label{fig:wavelet}
\end{figure}
\subsection{Proprietà}
La trasformata Wavelet gode delle seguenti proprietà:
\begin{itemize}
    \item localizzazione simultanea spaziale e temporale, in cui la localizzazione della wavelet permette esplicitamente di rappresentare gli eventi nel tempo e la forma delle wavelet permettono di rappresentare i dettagli differenti e risoluzione differenti;
    \item sparsità, in cui le funzioni usate nella pratica presentano coefficienti pari a zero o molto piccoli;
    \item adattabilità, poiché si possono rappresentare funzioni discontinue e con angoli in modo molto efficiente;
    \item complessità temporale lineare, cioè diverse trasformazioni si possono compiere in tempo $O(N)$.
\end{itemize}
\section{Analisi multirisoluzione}
Per effettuare l'analisi multirisoluzione di un'immagine, è necessario definire in maniera compatta la formula dell'espansione di un segnale $f(t)$ in serie.
\begin{equation*}
    f(t)=\sum_{k}\sum_{j}a_{jk}\psi_{jk}(t)=\sum_{k}\sum_{j}a_{jk}2^{\frac{j}{2}}\psi\left(2^jt-k\right)
\end{equation*}i
In particolare:
\begin{itemize}
    \item $a_{jk}$ sono coefficienti reali di espansione;
    \item $\psi_{jk}(t)$ sono le funzioni di espansione.
\end{itemize}
In particolare, si crea un mapping dove le $k$ sono le ascisse e $j$ le ordinate: le $\hat{f}_j(t)$ più in basso rappresentano i dettagli a bassa risoluzione; fino ad arrivare più vicino a $f(t)$, che rappresentano i dettagli ad alta risoluzione. Ne è mostrato un esempio nell'immagine di seguito.
\begin{figure}[htbp]
        \centering
        \includegraphics[width=1\textwidth]{cap4/serie} 
        \caption{Analisi dei dettagli delle serie} 
        \label{fig:serie}
\end{figure}
\subsection{Creazione delle piramide dell'immagine}
Una semplice struttura per rappresentare un'immagine a più risoluzione è la cosiddetta piramide dell'immagine, in cui:
\begin{itemize}
    \item alla base è presente la rappresentazione dell'immagine con la massima risoluzione, detto livello J;
    \item all'apice è presente la rappresentazione dell'immagine con la minima risoluzione, detto livello 0.
\end{itemize}
\begin{figure}[htbp]
        \centering
        \includegraphics[width=0.8\textwidth]{cap4/Slide2} 
        \caption{Piramide dell'immagine} 
        \label{fig:pyramid}
\end{figure}
La costruzione della piramide avviene nel modo seguente:
\begin{enumerate}
    \item si riduce la risoluzione con un filtro di approssimazione (es. Gaussiano) e si esegue un sottocampionamento di un fattore $2$;
    \item si sovracampiona di un fattore $2$ il risultato ottenuto e si applica un filtro di interpolazione, creando una predizione con la stessa risoluzione dell'ingresso iniziale;
    \item si esegue la differenza tra la predizione e l'input iniziale.
\end{enumerate}
\begin{figure}[htbp]
        \centering
        \includegraphics[width=0.8\textwidth]{cap4/Slide3} 
        \caption{Schema di costruzione della piramide dell'immagine} 
        \label{fig:cost}
\end{figure}
\subsection{Rappresentazione walet e condizioni}
La rappresentazione wavelet consiste in un'approssimazione grossolana in totale dell'immagine, in cui viene influenzata dai coefficienti dei dettagli in scale differenti.\\
Se un insieme di basi di funzioni $V$ si può rappresentare come una somma pesata di $\psi(2^jt-k)$, allora una gran parte dell'insieme ($V$ incluso) si può rappresentare da una somma pesata di $\psi(2^{j+1}t-k)$. Quindi
in questo caso, per $V_j \subseteq V_{j+1}$, se $f(t) \in V_j$ allora $f(t) \in V_{j+1}$.\\
A questo punto, l'idea di base è definire un insieme di basi di funzioni che coprono la differenza tra $V_j$ e $V_{j+1}$: costruendo la base ortogonale $W_j$.
\begin{equation*}
    V_{j+1}=V_j+W_j
\end{equation*}
Infine, si sfruttano $\varphi(t)$ per $V_j$ e $\psi(t)$ per $W_j$, per ricostruire la decomposizione come coppia di wavelet:
\begin{equation*}
    f(t)=\sum_{k}c_k\varphi\left(2^jt-k\right)+\sum_{k}d_{jk}\psi\left(2^jt-k\right)
\end{equation*}
dove:
\begin{itemize}
    \item il primo fattore indicano le basse risoluzioni, dove $\varphi(t)$ è una funzione di scala per le basse risoluzioni;
    \item il secondo fattore indicano le alte risoluzioni.
\end{itemize}
\section{Dalla trasformata di Haar alla trasformata Wavelet 2D}
In questo paragrafo viene riportata la spiegazione da come si passa dalla trasformata Haar alla trasfarmata Wavelet in due dimensioni.
\subsection{Trasformata Haar}
La trasformata di Haar si basa su base di funzioni che sono molto semplici, soprattutto che hanno la proprietà di essere simmetriche, separabili ed
esprimibili in forma matriciale.
\begin{center}
\begin{minipage}{0.48\textwidth} % Colonna Sinistra (48% della larghezza del testo)
    \centering
    \begin{equation*}
        \varphi(t)=\operatorname{rect}\left(x-\frac{1}{2}\right)
    \end{equation*}
\end{minipage}
\hfill % Spazio elastico per separare le due colonne
\begin{minipage}{0.48\textwidth} % Colonna Destra (48% della larghezza del testo)
    \centering
    \begin{equation*}
        \psi(t)=\operatorname{rect}_{1/2}\left(x-\frac{1}{4}\right)-\operatorname{rect}_{1/2}\left(x-\frac{3}{4}\right)
    \end{equation*}
\end{minipage}
\end{center}
Considerando le formule seguenti di Wavelet padre (a sinistra) e Wavelet madre (a destra):
\begin{center}
\begin{minipage}{0.48\textwidth} % Colonna Sinistra (48% della larghezza del testo)
    \centering
    \begin{equation*}
        \varphi_{jk}(x)=2^{\frac{j}{2}}\varphi\left(2^jx-k\right)
    \end{equation*}
\end{minipage}
\hfill % Spazio elastico per separare le due colonne
\begin{minipage}{0.48\textwidth} % Colonna Destra (48% della larghezza del testo)
    \centering
    \begin{equation*}
        \psi_{jk}(x)=2^{\frac{j}{2}}\psi\left(2^jx-k\right)
    \end{equation*}
\end{minipage}
\end{center} 
si considera un esempio di espansione in serie.

\chapter{Processo di compressione delle immagini}
Il processo di compressione di un'immagine, consiste nel ridurre drasticamente la quantità di dati presenti nell'immagine, poiché trasmettere un'immagine
non compressa richiede molto tempo e molta larghezza di banda, come mostrato nell'immagine seguente.
\begin{figure}[htbp]
        \centering
        \includegraphics[width=0.9\textwidth]{cap5/compressione} 
        \caption{Processo di trasmissione di un'immagine compressa}
        \label{fig:hansform}
\end{figure}
Basti pensare che già un'immagine $1024\times1024$ in scala di grigi con risoluzione bassa ($8 \, b/p$), per essere trasmessa in 2G ($v=50 \, kb/s $), sono necessari:
\begin{center}
\begin{minipage}{0.48\textwidth} % Colonna Sinistra (48% della larghezza del testo)
    \centering
    \begin{equation*}
    c=1024\times1024\times8=8388608 \, b
\end{equation*}
\end{minipage}
\hfill % Spazio elastico per separare le due colonne
\begin{minipage}{0.48\textwidth} % Colonna Destra (48% della larghezza del testo)
    \centering
        \begin{equation*}
    b=\frac{c}{v}=\frac{8388608}{50000} = 167,77216 \, s \approx 3'
    \end{equation*}
\end{minipage}
\end{center}
che è decisamente troppo elevato.
\section{Ridondanza dei dati}
Per prima cosa è necessario ribadire che dato ed informazione non sono assolutamente sinonimi. Infatti:
\begin{itemize}
    \item un'informazione è ciò che si vuole trasmettere;
    \item un dato rappresenta i modi di trasmettere l'informazione.
\end{itemize}
La ridondanza dei dati si può quantificare matematicamente come entità. Infatti, si considerano le seguenti grandezze:
\begin{itemize}
    \item rapporto di compressione ($C_R$), che è il rapporto tra due insiemi di dati che contengono la stessa informazione;
    \begin{equation*}
        C_{R} = \frac{n_1}{n_2}
    \end{equation*}
    \item ridondanza relativa dei dati ($R_D$), che indica la quantità dei dati che sono ridondanti, spesso indicati in percentuale.
    \begin{equation*}
        R_D = 1-\frac{1}{C_R}
    \end{equation*}
\end{itemize}
In questo paragrafo, verranno trattate le diverse tipologie di rindondanza.
\subsection{Ridondanza di codifica}
La ridondanza di codifica si ha nel momento in cui si sceglie un sistema di codifica non efficiente. Per calcolare l'efficienza di un
algoritmo di compressione, dipende dal livello medio di grigio, che dipende dalla probabilità che ogni livello di grigio dell'immagine
sia presente in tale immagine e quanti bit occupa ($l(r_k)$).
\begin{center}
\begin{minipage}{0.48\textwidth} % Colonna Sinistra (48% della larghezza del testo)
    \centering
    \begin{equation*}
        p_r(r_k)=\frac{n_k}{MN}
    \end{equation*}
\end{minipage}
\hfill % Spazio elastico per separare le due colonne
\begin{minipage}{0.48\textwidth} % Colonna Destra (48% della larghezza del testo)
    \centering
    \begin{equation*}
        L_{avg}=\sum_{0}^{L-1}l(r_k)p_r(r_k)
    \end{equation*}
\end{minipage}
\end{center}
Per esempio data un'immagine $256 \times 256$ con le seguenti caratteristiche:
\begin{table}[H] % Usa [H] per renderla non flottante
    \centering
    \begin{tabular}{| C{5cm} | C{5cm} |} 
            \hline
            
            % Usiamo \thead per le intestazioni:
            \textbf{$r_k$} & \textbf{$p_k(r_k)$} \\ 
            \hline % Linea spessa sotto le intestazioni
            $r_{87}=87$ & $3/10$ \\
            \hline % Linea spessa sotto le intestazioni
            $r_{128}=128$ & $1/2$ \\
            \hline % Linea spessa sotto le intestazioni
            $r_{186}=186$ & $1/10$ \\
            \hline % Linea spessa sotto le intestazioni
            $r_{255}=255$ & $1/10$ \\
            \hline % Linea spessa sotto le intestazioni
            $r_{k} \, se \, k \neq 87, 128, 186, 255$ & ${0}$ \\
            \hline
    \end{tabular}
\end{table}
per esempio si usa per ogni livello di grigio il numero massimo rappresentabile del livello di grigio più alto ($255$), che è $8$ ($\log_{2}{(255)}$): Il
livello medio di grigio è $L_{avg_{1}}=8\, b$. A questo punto conviene assegnare il numero di bit a seconda della probabilità: maggiore è la probabilità,
meno bit deve avere quel livello di grigio, così lo spazio occupato sarebbe decisamente minore.
\begin{table}[H] % Usa [H] per renderla non flottante
    \centering
    \begin{tabular}{| C{5cm} | C{5cm} | C{5cm}|} 
            \hline
            
            % Usiamo \thead per le intestazioni:
            \textbf{$r_k$} & \textbf{$p_k(r_k)$} & \textbf{$l_k(r_k)$} \\ 
            \hline % Linea spessa sotto le intestazioni
            $r_{87}=87$ & $3/10$ & $2$\\
            \hline % Linea spessa sotto le intestazioni
            $r_{128}=128$ & $1/2$ & $1$\\
            \hline % Linea spessa sotto le intestazioni
            $r_{186}=186$ & $1/10$ & $3$\\
            \hline % Linea spessa sotto le intestazioni
            $r_{255}=255$ & $1/10$ & $3$ \\
            \hline % Linea spessa sotto le intestazioni
    \end{tabular}
\end{table}
\begin{equation*}
    L_{avg_{2}}=\frac{3}{10}\times 2 + \frac{1}{2}\times 1 + \frac{1}{10} \times 3 + \frac{1}{10} \times 3 = \frac{17}{10} \, b = 1,7\, b
\end{equation*}
Infine, si calcolano le due grandezze.
\begin{center}
\begin{minipage}{0.48\textwidth} % Colonna Sinistra (48% della larghezza del testo)
    \centering
    \begin{equation*}
        C_R = \frac{8\times10}{17}=\frac{80}{17}
    \end{equation*}
\end{minipage}
\hfill % Spazio elastico per separare le due colonne
\begin{minipage}{0.48\textwidth} % Colonna Destra (48% della larghezza del testo)
    \centering
    \begin{equation*}
        R_D = 1 - \frac{17}{80} = \frac{63}{80} = 78,75\%
    \end{equation*}
\end{minipage}
\end{center}
\subsection{Ridondanza interpixel}
La ridondanza interpixel implica che qualsiasi valore di un pixel può essere predetto dai sui vicini, grazie alla correlazione. Per ridurre ciò,
i dati dovrebberero essere mappati.
\begin{figure}[htbp]
        \centering
        \includegraphics[width=0.4\textwidth]{cap5/interpixel} 
        \caption{Esempio di ridondanza di interpixel}
        \label{fig:interpixel}
\end{figure}
\subsection{Ridondanza psicovisiva}
La ridondanza psicovisiva sta nel fatto che alcuni pixel sono talmente simili che l'occhio umano non li percepisce, dato che il sistema visivo umano
non percepisce tutta l'informazione visiva con la stessa intensità: ma cerca solo caratteristiche importanti, come angoli e texture. Il classico esempio
è un immagine a tinta unita.
\begin{figure}[htbp]
        \centering
        \includegraphics[width=0.9\textwidth]{cap5/psico} 
        \caption{Esempio di ridondanza psicovisiva}
        \label{fig:psicovisiva}
\end{figure}
\section{Teoria dell'informazione}
Per effettuare una codifica efficiente, è necessario conoscere dei cenni di teoria dell'informazione.
\begin{figure}[htbp]
        \centering
        \includegraphics[width=0.9\textwidth]{cap5/teoria} 
        \caption{Esempio di rappresentazione della teoria dell'informazione} 
        \label{fig:teoria}
\end{figure}
\subsection{Concetti base di informazione}
Per prima cosa è necessario comprendere i concetti base dell'informazione:
\begin{itemize}
    \item più la probabilità di un evento è bassa più il contenuto informativo è alto;
    \item l'informazione di più messaggi indipendenti sono la somma delle informazione corrispottive.
\end{itemize}
Detto ciò, sia $X$ una sorgente che genera un messaggio $x_i$ con probabilità $p(x_i)$, l'informazione si calcola nel modo seguente:
\begin{equation*}
    I(x_i) = \log_{c}{\left[\frac{1}{p(x_i)}\right]}
\end{equation*}
dove $c$ è la base del logiritmo, che dipende dal tipo di codifica. Le più comuni sono il bit ($c=2$) ed il nat ($c=e$).
\subsection{Entropia}
Un altro concetto fondamentale è l'entropia dell'informazione, che misura l'incertezza media e l'informazione attesa della sorgente.
\begin{equation*}
    H(X)=-\sum_{i=1}^{N}p(x_i)\log_{c}{[p(x_i)]}
\end{equation*}
L'entropia dell'informazione è protagonista del teorema della codifica di sorgente di Shannon, che afferma che è impossibile comprimere i dati
che presentano come velocità di codifica, cioè il numero di simboli trasmessi per secondo, minore dell'entropia senza perdere del contenuto informativo.
\subsection{Criteri di fedeltà}
Infine, l'ultima parte analizzata della teoria dell'informazione è quella dedicata ai criteri di fedeltà, che quantifica quanta informazione è possibile
perdere mantenendo ancora l'informazione accettata oppure no. In particolare, si considerano due tipologie di criteri di fedeltà.\\
Il primo criterio è quello soggettivo, che si basano solamente sulla comparazione visiva tra due immagini, classificando le immagini in rankig da eccellente ad inutilizzabile.\\
Il secondo ed ultimo criterio è quello oggettivo, che si basa in base a delle metriche matematiche, che dipendono dall'errore ($\epsilon(x,y)$) e dall'errore assoluto ($\epsilon$),
calcolati in base all'immagine originale ($f(x,y)$) ed all'immagine compressa ($\hat{f}(x,y)$).
\begin{center}
\begin{minipage}{0.48\textwidth} % Colonna Sinistra (48% della larghezza del testo)
    \centering
    \begin{equation*}
        \epsilon(x,y) = \hat{f}(x,y)-f(x,y)
    \end{equation*}
\end{minipage}
\hfill % Spazio elastico per separare le due colonne
\begin{minipage}{0.48\textwidth} % Colonna Destra (48% della larghezza del testo)
    \centering
    \begin{equation*}
        \epsilon = \sum_{x=0}^{M-1}\sum_{y=0}^{N-1}\left[\hat{f}(x,y)-f(x,y)\right]
    \end{equation*}
\end{minipage}
\end{center}
A questo punto, i criteri di fedeltà sono dettati dalle metriche di errore quadratico medio ($RMSE$), il rapporto segnale-rumore quadrico medio ($SNR_{ms}$)
ed il rapporto segnale-rumore di picco ($PSNR$), dove questi ultimi due si misurano in Decibel (dB), che è una scala logaritmica.
\begin{center}
\begin{minipage}{0.52\textwidth} % Colonna Sinistra (48% della larghezza del testo)
    \centering
    \begin{equation*}
        RMSE=\sqrt{\frac{1}{MN}\sum_{x=0}^{M-1}\sum_{y=0}^{N-1}\left[\hat{f}(x,y)-f(x,y)\right]^2}
\end{equation*}
 \begin{equation*}
        SNR_{ms}=\sum_{y=0}^{N-1}\hat{f}^2(x,y)\left\{\sum_{x=0}^{M-1}\sum_{y=0}^{N-1}\left[\hat{f}(x,y)-f(x,y)\right]^2\right\}^{-1}
\end{equation*}
\begin{equation*}
        PSNR=L_{max}\left\{\sum_{x=0}^{M-1}\sum_{y=0}^{N-1}\left[\hat{f}(x,y)-f(x,y)\right]^2\right\}^{-1}
\end{equation*}
\end{minipage}
\hfill % Spazio elastico per separare le due colonne
\begin{minipage}[c]{0.44\textwidth} % Colonna Destra (48% della larghezza del testo)
    \centering
        \includegraphics[width=0.7\textwidth]{cap5/psnr} 
        \captionof{figure}{Misura del PSNR in un'immagine} 
        \label{fig:psnr}
\end{minipage}
\end{center}
\section{Modello di compressione delle immagini}
Il modello di compressione delle immagini consistono nei seguenti operazioni:
\begin{enumerate}
    \item mappatura, che consiste nel prendere l'immagine ed eseguire una mappatura dei dati, riducendo la ridondanza interpixel (operazione reversibile);
    \item quantizzazione, che esegue l'operazione di quantizzazione, riducendo la ridondanza psicovisiva (operazione irreversibile);
    \item codifica di simbolo, che esegue un sistema di codifica dell'immagine (operazione reversibile) e lo invia al canale;
    \item decodifica di simbolo, che dal canale prende l'immagine codifica e la ritrasforma prima della codifica;
    \item mappatura inversa, che la riporta come matrice di pixel, leggermente diversa a causa del quantizzatore.
\end{enumerate}
\begin{figure}[htbp]
        \centering
        \includegraphics[width=0.9\textwidth]{cap5/modello} 
        \caption{Schema del modello di compressione di un'immagine} 
        \label{fig:modello}
\end{figure}
A seconda dell'errore, la tipologia di compressione di un'immagine si suddividono in due grandi categorie:
\begin{itemize}
    \item compressione lossless, in cui l'errore è nullo, perciò non viene perso alcuna informazione, sfruttando la ridondanza di codfica ed interpixel;
    \item compressione lossy, invece, sfrutta ogni tipologia di ridondanza (anche quella psicovisiva), tollerando qualche errore o qualche
        perdita d'informazione.
\end{itemize}
\begin{center}
\begin{minipage}{0.48\textwidth} % Colonna Sinistra (48% della larghezza del testo)
    \centering
        \includegraphics[width=0.7\textwidth]{cap5/satellite}     
\end{minipage}
\hfill % Spazio elastico per separare le due colonne
\begin{minipage}{0.48\textwidth} % Colonna Destra (48% della larghezza del testo)
    \centering
        \includegraphics[width=0.7\textwidth]{cap5/telefono} 
\end{minipage}
\captionof{figure}{Esempio tipologia di immagine da comprimere lossless (a sinistra) od anche lossy (a destra)}
\label{fig:compressione}
\end{center}

\chapter{Formato JPEG}
Il formato JPEG (Joint Photographic Expert Group) è sicuramente il formato delle immagini più conosciuto ed usato, poiché ottiene dei risultati molto
notevoli nelle foto (sia in bianco e nero che a colori), tuttavia non per i cartoni animati e per le immagini generate al computer, nonostante implementi
un modello di compressione lossy. In particolare, ne esistono due versioni, entrambe trattate in questo capitolo.
\section{Prima versione di JPEG}
Di seguito è mostrato uno schema che riassume i passi eseguiti nel formato di compressione della prima versione di JPEG.
\begin{figure}[htbp]
        \centering
        \includegraphics[width=1\textwidth]{cap6/schema1} 
        \caption{Schema del modello di compressione JPEG} 
        \label{fig:schemaJPEG1}
\end{figure}
\subsection{Fase 1: conversione da RGB a YCbCr}
Un'immagine a colori viene rappresentata nel formato RGB, che non sono altre tre matrici dove R contiene i livelli di rosso, G di verde e B di blu. tuttavia
tale rappresentazione non risulta efficace nella compressione, dato che sono ridondanti. Per questo motivo, si convertono tali matrici nel formato YCbCr, dove Y è luminanza e CbCr
rappresentano la crominanza, che insieme non sono affatto ridondanti. Infatti, l'occhio umano percepisci la variazione di luminosità che di colore, dato
dal fatto che l'occhio umano ha molti più bastoncelli che coni. La formula di conversione è la seguente:
\begin{equation*}
    \begin{bmatrix}
        Y\\
        Cb\\
        Cr
    \end{bmatrix}=
    \begin{bmatrix}
        0,299 & 0,587 & 0,114\\
        -0,1687 & -0,3313 & 0,5\\
        0,5 & -0,4187 & -0,0813
    \end{bmatrix} \begin{bmatrix}
        R\\
        G\\
        B
    \end{bmatrix}+ \begin{bmatrix}
        0\\
        128\\
        128
    \end{bmatrix}
\end{equation*}
Inoltre, la risoluzione delle componenti cromatiche vengono ridotte di un fattore $2$. In particolare:
\begin{itemize}
    \item $4:4:4$ non si ha nessun sottocampionamento;
    \item $4:2:2$ si ha una riduzione solamente nella direzione orizzontale;
    \item $4:2:0$ si ha una riduzione sia nella direzione orizzontale sia in quella verticale.
\end{itemize}
\subsection{Fase 2: trasformata DCT}
A questo punto, viene suddivisa l'immagine in blocchi da $8x8$, poiché è la dimensione che permette la migliore qualità rispetto alla dimensione del file.
Per ogni blocco, viene eseguita la cosiddetta DCT (Discete Cosine Transform), che lavora nell'intervallo da $-128$ a $128$, perciò il blocco deve essere
prima traslato negativamente di $128$. A questo punto, la formula della DCT è la seguente:
\begin{equation*}
    G(u,v)=\alpha(u)\alpha(v)\sum_{x=0}^{7}\sum_{y=0}^{7}g(x,y)\cos{\left[\frac{\pi}{8}\left(x+\frac{1}{2}\right)u\right]}\cos{\left[\frac{\pi}{8}\left(y+\frac{1}{2}\right)v\right]}
\end{equation*}
dove la moltiplicazione dei coseni non dipende dai valori di $g(x,y)$ e $\alpha$ è una funzione di normalizzazione.
Inoltre, dei $64$ coefficienti ottenuti:
\begin{itemize}
    \item $G(0,0)$ è il nucleo e viene classificato come coefficiente DC;
    \item gli altri $63$ coefficienti vengono classificati come coefficienti AC.
\end{itemize}
Infine, i motivi per cui si usa la DCT e non la trasformata di Fourier sono:
\begin{itemize}
    \item la DCT è reale pura, mentre la trasformata di Fourier è complessa;
    \item la DCT presenta meno coefficienti di qualsiasi segnale;
    \item il nucleo della trasformata diretta ed inversa sono gli stessi nella DCT.
\end{itemize}
\subsection{Fase 3: quantizzazione}
Nella fase di quantizzazione avviene la perdita vera e propria di informazione della compressione. Infatti, viene diviso punto punto per una determinata
matrice e approssimato alla parte intera più piccola. Siccome la quantizzazione avviene in base ad una soglia, viene ridotto il numero di bit per
campionamento. La formula è la seguente:
\begin{equation*}
    T^{*}(u,v)= \left\lfloor \frac{T(u,v)}{Z(u,v)} \right\rfloor
\end{equation*}
dove:
\begin{itemize}
    \item $T(u,v)$ è il coefficiente trasformato;
    \item $Z(u,v)$ è il coefficiente trasformato normalizzato;
    \item $T^{*}(u,v)$ è il coefficiente sogliato e quantizzato dell'approssimazione di $T(u,v)$.
\end{itemize}
\subsection{Fase 4: pattern zig-zag}
A questo punto, vengono ordinati i coefficienti usando un pattern zig-zag, in questo modo vengono ottenute sequenze consecutive di $0$ molto più lungh
rispetto che farlo per riga.
\begin{figure}[htbp]
        \centering
        \includegraphics[width=1\textwidth]{cap6/zigzag} 
        \caption{Pattern zig-zag} 
        \label{fig:zig-zag}
\end{figure}
\subsection{Fase 5: Codifica di entropia}
A questo punto:
\begin{itemize}
    \item tutti i coefficienti DC vengono codificati con la DPCM (Differential pulse-code modulation), che non è altro che la differenza tra i
    coefficienti DC ed i coefficienti DC dell'immagine precedente;
    \item tutti i coefficienti AC vengono codificati utilizzando la RLC, siccome sono presenti molti $0$ consecutivi.
\end{itemize}
Infine, tutti i coefficienti vengono codificati in una sequenza binaria, come quella di Huffmann e quella aritmetica; ed infine viene rieseguita la
IDCT per riottenere l'immagine compressa.
\section{JPEG2000}
La seconda
\vfill
\chapter{Fondamenti della percezione della profondità}
In questo capitolo, viene trattato come avviene la percezione dello spazio tridimensionale, partendo solamente da un'immagine, che invece possiede
solamente due dimensioni. Per prima cosa, è necessiario comprende il funzionamento del nervo oculomotore.
Per prima cosa, avviene l'accomodamento, in cui il cristallino si comporta come una lente convergente di distanza focale variabile. Inoltre, il muscolo
ciliare si contrae e si allunga, variando forma e quindi la distanza focale.
Perciò, nel momento in cui si guardano oggetti lontani, il cristallino si appiattisce, mentre quando si osservano oggetti vicini diventa più spesso.
\begin{figure}[htbp]
        \centering
        \includegraphics[width=0.9\textwidth]{cap7/adattamento} 
        \caption{Processo di adattamento} 
        \label{fig:accomodamento}
\end{figure}
In contemporanea, avviene la cosiddetta vergenza, che non è altro che un movimento simultaneo degli occhi in direzioni opposte, che consentono la fusione.
In particolare:
\begin{itemize}
    \item gli occhi ruotano uno verso l'alto per guardare oggetti vicini, detta convergenza;
    \item gli occhi ruotano in direzioni opposte per guardare oggetti lontani, denominata divergenza.
\end{itemize}
\begin{figure}[htbp]
        \centering
        \includegraphics[width=0.6\textwidth]{cap7/vergenza} 
        \caption{Processo di vergenza} 
        \label{fig:vergenza}
\end{figure}
\section{Indicatori monoculari di profondità statici}
In questo paragrafo, vengono trattati i cosiddetti indicatori monoculari di profondità statici, cioè quelle caratteristica che forniscono delle
informazioni tridimensionali in immagini. Come schema, conviene analizzare il dipinto \textit{A Rainy Day in Paris}.
\begin{figure}[htbp]
        \centering
        \includegraphics[width=0.9\textwidth]{cap7/quadro} 
        \caption{\textit{A Rainy Day in Paris} che mostra le occlusioni (1), le dimensioni relative (2), le tessiture (3), la prospettiva lineare (4), la prospettiva aerea (5) e le ombre (6)} 
        \label{fig:quadro}
\end{figure}
\subsection{Occlusioni}
Il primo indicatore sono sicuramente le occlusioni, che si definiscono come gli oggetti più vicini che bloccano l'accesso visivo a quello più distanti.
Per esempio, nella figura seguente viene mostrata una ragazza che viene parzialmente coperta da un tronco di una palma.
\begin{figure}[htbp]
        \centering
        \includegraphics[width=0.8\textwidth]{cap7/occlusioni} 
        \caption{Esempio di occlusione} 
        \label{fig:occlusione}
\end{figure}
\subsection{Dimensioni relative}
Le dimensioni relative forniscono informazioni 3D, grazie ad oggetti della stessa dimensione fisica proiettano immagini retiniche di dimensione diversa
a seconda del punto del punto di osservazione. Per esempio, è molto più facile conoscere l'altezza di una statua dell'isola di Pasqua con delle persone
(che hanno un'altezza media di 1,80 m), rispetto a che non ci sono persone od altri oggetti di riferimento: infatti, anche l'esperienza gioca 
un ruolo fondamentale in questo contesto.
\begin{figure}[htbp]
        \centering
        \includegraphics[width=0.8\textwidth]{cap7/relative} 
        \caption{Esempio di dimensioni relative} 
        \label{fig:relative}
\end{figure}
\vfill
\part{Video}
\chapter{Creazione dei video}
\end{document}