\documentclass[openany]{book}
\usepackage[italian]{babel}
% ... altri pacchetti ...
\usepackage[
    top=2cm,
    bottom=2cm,
    left=2cm,
    right=2cm,
    headheight=14pt % Aggiungi questo se usi intestazioni, altrimenti LaTeX potrebbe lamentarsi.
]{geometry}
% Imposta la profondità dell'indice:
% Aumenta la porzione di pagina che può essere occupata dalle figure in alto (0.9 = 90%)
\renewcommand{\topfraction}{0.9} 
% Permette che la figura occupi almeno il 90% dello spazio superiore
\renewcommand{\textfraction}{0.1} 
% Permette che resti solo il 10% di testo nella pagina superiore
\setcounter{topnumber}{2} 
% Permette 2 figure in cima alla pagina
\usepackage{graphicx}
\usepackage{fontspec}
\usepackage{xcolor}
\usepackage{titlesec}
\usepackage{float}
\usepackage{caption}
\usepackage{mathtools}
\usepackage{xcolor}
\usepackage{pagecolor} % Necessario per \pagecolor
\usepackage{chngcntr} % <-- PACCHETTO AGGIUNTO
\usepackage{amsmath}
\usepackage{array}
\usepackage{fancyhdr}
\usepackage{rotating}
\usepackage[utf8]{inputenc}
\usepackage{eso-pic}
\usepackage{tikz}
\usepackage{geometry}
\usepackage[font=normalsize]{caption}
\usepackage{etoolbox}
\usepackage{lmodern}
\makeatletter
\patchcmd{\tableofcontents}{\clearpage}{}{}{}
\patchcmd{\listoffigures}{\clearpage}{}{}{}
\patchcmd{\listoftables}{\clearpage}{}{}{}
\makeatother
% Definisce un nuovo tipo di colonna chiamato 'C' (maiuscolo) che centra orizzontalmente e verticalmente.
\newcolumntype{C}[1]{>{\centering\arraybackslash}m{#1}}


\usepackage{makecell} 
% \renewcommand{\arraystretch}{1.5}
\tolerance=3000
% Definizione Colori
\AtBeginDocument{\fontsize{11}{12}\selectfont}

%\setmainfont{Arial} 
\graphicspath{{immagini/}}

% Modulo (PART) - Come nell'immagine 1
\titleformat{\part}[display]
  {\normalfont\Large\bfseries\centering\fontsize{30}{20}\selectfont} % Formato del titolo
  {\MakeUppercase{\partname}\ \thepart:} % Etichetta (es. MODULO 1)
  {10pt} % Spazio tra etichetta e titolo
  {\MakeUppercase} % Trasforma il titolo in maiuscolo
  [\vspace{1ex}] % Spazio aggiuntivo dopo il titolo (opzionale)

% Capitolo (CHAPTER) - Come nell'immagine 2
\titleformat{\chapter}[display]
  {\normalfont\Huge\bfseries\fontsize{18}{20}\selectfont} % Formato del titolo
  {\chaptertitlename\ \thechapter:} % Etichetta (es. Capitolo 1)
  {1em} % Spazio tra etichetta e titolo
  {} 

% Sezione (SECTION) - Come nell'immagine 2 (1.1)
\titleformat{\section}
  {\normalfont\Large\bfseries\fontsize{15}{17}\selectfont} % Formato del titolo
  {\thesection:} % Etichetta (es. 1.1)
  {1em} % Spazio orizzontale
  {}

% Sottosezione (SUBSECTION) - Come nell'immagine 2 (1.1.1)
\titleformat{\subsection}
  {\normalfont\large\bfseries\itshape\fontsize{12}{14}\selectfont} % Formato del titolo
  {\thesubsection:} % Etichetta (es. 1.1.1)
  {1em} % Spazio orizzontale
  {} % Non fare nulla al titolo
\counterwithout{figure}{chapter}
\counterwithout{chapter}{part} 
\setcounter{tocdepth}{2}

\begin{document}

% ----------------------------------------------------
% INIZIO MATERIA PRELIMINARE (Numerazione Romana)
% ----------------------------------------------------

% 1. IMPOSTA LA SEZIONE ROMANA (DEVE ESSERE PRIMA)
\frontmatter 

% 2. IMPOSTA IL CONTATORE A -1 (PER PAGINA 0)
% La pagina successiva (\begin{titlepage}) sarà la pagina -1 + 1 = 0.
%\setcounter{page}{-1} 

% 3. IMPOSTA IL COLORE DI SFONDO PER LA PAGINA 0

% --- Frontespizio Personalizzato (Pagina 0) ---
\begin{titlepage}

  \color{white}
    
    \thispagestyle{empty} % Nasconde il numero 0.
    \AddToShipoutPictureBG*{%
      \begin{tikzpicture}[remember picture, overlay]
        \node[opacity=1, inner sep=0pt] at (current page.center) {
          % 'keepaspectratio' evita che l'immagine si deformi se il libro non è A4
          % 'min width' e 'min height' assicurano che copra tutto
          \includegraphics[width=\paperwidth, height=\paperheight]{sfondo_sfocato.jpg}
        };
      \end{tikzpicture}%
    }
    
    \centering 

\includegraphics[width=11cm]{sfondo} 

\vspace{0.5cm}

{\fontsize{30pt}{36pt}\selectfont \bfseries Laboratorio di \\ multimedialità \par}

\vspace{2cm} 

{\fontsize{15pt}{18pt}\selectfont A.A. 2024-25 \par}

\vspace{1cm} 

{\fontsize{18pt}{22pt}\selectfont \textbf{Andrea Macale} \par}

\vfill
\end{titlepage}

% ----------------------------------------------------
% --- INDICE (Inizia con Pagina i) ---

% Cambia il colore di sfondo per l'indice
\setcounter{page}{1}
\pagestyle{plain} % Mostra il numero in fondo alla pagina (i, ii, iii...)
\tableofcontents
\listoffigures


% ----------------------------------------------------
% INIZIO CORPO PRINCIPALE (Numerazione Araba da 1)
\mainmatter 
\pagestyle{fancy}
\fancyhf{} % Pulisce gli header e i footer

% Intestazioni (Headers)
% Pagina PARI (Sinistra): Titolo della Sezione/Capitolo
\fancyhead[LE]{\nouppercase{\rightmark}} 
% Pagina DISPARI (Destra): Titolo della Part/Capitolo
\fancyhead[RO]{\nouppercase{\leftmark}} 

% Piè di Pagina (Footers) - Numerazione alternata
\fancyfoot[LE,RO]{\thepage} 

\renewcommand{\headrulewidth}{0.4pt} % Aggiunge una linea orizzontale
\renewcommand{\footrulewidth}{0pt} % Rimuove la linea dal piè di pagina

% Inizia la numerazione araba da 1 di default.

\part{Immagini}
\chapter{Elaborazione delle immagini}
In questo capitolo viene spiegato come avviene l’elaborazione delle immagini. Per prima cosa, è necessario capire come l’occhio umano cattura e percepisce l’immagine, soprattutto per comprendere quali sono i suoi grandi limiti. Una volta capito ciò, si può procedere all’elaborazione delle immagini.
\section{Sistema visivo umano}
L'occhio umano è racchiuso da tre membrane, dove ognuna ha una funzione rilevante per l'elaborazione delle immagini.
\begin{figure}[htbp]
        \centering
        \includegraphics[width=0.6\textwidth]{cap01/occhio} 
        \caption{Struttura dell'occhio umano} 
        \label{fig:occhio}
\end{figure}
\subsection{Cornea e coroide: due membrane come protezione}
La cornea è la membrana più esterna dell'occhio umano: infatti, essendo composta da un tessuto resistente e trasparente, è perfetta per racchiudere la superficie anteriore dell'occhio.
Inoltre, per ridurre la quantità di luce estranea che entra nell'occhio, è presente la coroide.
\subsection{Retina: la membrana per la vista}
La retina è quella membrana che fornisce il senso della vista all'essere umano. In particolare, permette di mettere a fuoco gli oggetti, grazie alla luce dell'oggetto stesso che entra nella retina.\\ Inoltre, sono presenti i cosiddetti percettori luminosi, che rendono possibile la visione a pattern. \\Il primo percettore sono i coni, che permettono la visione cromatica: infatti possono essere a lunghezza d'onda corta per il blu, a lunghezza d'onda media per il verde ed a lunghezza d'onda lunga per il rosso.\\ Per la visione acromatica, invece, sono presenti i bastoncelli, per esempio la visione scotopica e la penombra.
\begin{figure}[htbp]
        \centering
        \includegraphics[width=1\textwidth]{immagini/cap01/coni-bastoncelli-retina.jpeg} 
        \caption{Coni e bastoncelli} 
        \label{fig:coni}
\end{figure}
\subsection{Formazione dell'immagine}
A questo punto, la formazione dell'immagine avviene nel modo seguente. Le lenti dell'occhio umano sono flessibili: la sua forma è controllata delle fibre del corpo ciliare. \\Inoltre, l'abilità dell'occhio di discriminare i cambiamenti delle intensità di luce a qualsiasi livello di adattamento specifico, è descritta dalla legge di Weber.
\begin{equation*}
    k=\frac{\Delta I_C}{I}
    \label{eq:Legge di Weber}
\end{equation*}
In particolare:
\begin{itemize}
    \item $k$ è la costante di Weber, che è un valore costante e caratteristico per ogni specifica modalità sensoriale;
    \item $\Delta I_C$ è la soglia differenziale, ossia la quantità minima di cambiamento affinché il soggetto percepisca una differenza;
    \item $I$ è l'intensità di riferimento dello stimolo.
\end{itemize}
Ciò spiega che il sistema visivo tende a sottostimare od a sovrastimare i bordi delle regioni a diverse intensità. Ciò genera le illusioni: esempi sono l'illusione di dolcezza di mais e la griglia di Hermann.
\begin{figure}[H]
    \centering % Centra l'intero blocco delle immagini e della didascalia
    \begin{minipage}{0.48\textwidth} % Minipage che occupa quasi metà della larghezza del testo
        \centering % Centra l'immagine all'interno della minipage
        \includegraphics[width=\textwidth]
        {immagini/cap01/Cornsweet_illusion.png} % L'immagine occupa tutta la larghezza della minipage
        \label{fig:cornsweet}
    \end{minipage}
    \hfill
    \begin{minipage}{0.48\textwidth} % Minipage che occupa quasi metà della larghezza del testo
        \centering % Centra l'immagine all'interno della minipage
        \includegraphics[width=\textwidth]{immagini/cap01/Illusion_of_gray_dots_nevit_065.svg.png} % L'immagine occupa tutta la larghezza della minipage
        \label{fig:grigliall}
    \end{minipage}
    \caption{Illusione del mais dolce (a sinistra) e la griglia di Hermann (a destra)}
    \label{fig:confronto_immagini}
\end{figure}
\section{Rappresentazione delle immagini}
Un'immagine è una funzione bidimensionale ($f(x,y)$), definita come il prodotto tra due funzioni bidimensionali:
\begin{itemize}
    \item la luminanza ($i(x,y)$), che è la quantità di luce della sorgente incidente sulla scena osservata;
    \item la riflettanza ($r(x,y)$), che è la quantità di luce riflessa dall'oggetto nella scena.
\end{itemize}
\begin{equation*}
    f(x,y)=i(x,y)r(x,y)
    \label{eq:f}
\end{equation*}
\begin{equation*}
    0 < i(x,y) < \infty \\
    \label{eq:Luminanza}
\end{equation*}
\begin{equation*}
    0 < r(x,y) < 1 \\
    \label{eq:Riflettanza}
\end{equation*}
\begin{equation*}
    0 < f(x,y) < \infty \\
    \label{eq:Immagine}
\end{equation*}
\subsection{Intensità di un'immagine}
L'intensità di un'immagine monocromatica a $(x_0,y_0)$ è il livello di grigio $L$ in quel punto.
\begin{equation*}
    L=f(x_0,y_0)
    \label{eq:livello di grigio}
\end{equation*}
Il livello di grigio trovato appartiene ad un intervallo dove il minimo corrisponde al nero ed il massimo corrisponde al bianco.
\begin{equation*}
    L \in [L_{\text{min}},L_{\text{max}}]
    \label{eq:intervallo}
\end{equation*}
\subsection{Digitalizzazione delle immagini}
Per effettuare la digitalizzazione di un'immagine, sono necessari i seguenti parametri:
\begin{itemize}
    \item $M$, che è la larghezza dell'immagine;
    \item $N$, che è l'altezza dell'immagine;
    \item $L$, che è il numero dei livelli di grigi, che è un multiplo di $2^n$.
\end{itemize}
A questo punto, avviene la digitalizzazione, che si compone in due fasi. \\La prima fase è il campionamento, che consiste nel suddividere l'immagine in una griglia regolare di punti o celle. Ogni cella produce un pixel, che è l'unità minima di un'immagine digitale. A questo punto viene misurata la risoluzione in dpi, che non è altro che la densità della griglia: più è fitta più l'immagine sarà fedele all'originale.\\
La seconda ed ultima fase è la quantizzazione, che è il processo di discretizzazione dell'intensità. In particolare, ad ogni pixel gli viene assegnato un valore numerico discreto che ne codifica il livello di grigio. Infine, il valore numerico viene convertito in una stringa binaria.\\
Il risultato della digitalizzazione, produce una matrice di numeri reali di dimensioni $M \times N$.
\[
f(x,y) = \begin{bmatrix}
f(0,0) & \dots & f(0, N-1) \\
\vdots & \ddots & \vdots \\
f(M-1,0) & \dots & f(M-1, N-1)
\end{bmatrix}
\]
\section{Manipolazione delle immagini}
La manipolazione delle immagini si pone come obiettivo principale quello di modificare l'immagine originale che si adatta di più al contesto richiesto: come sfocare lo sfondo in un ritratto oppure rendere più luminose le vene in un'immagine catturata da un esame medico.\\
In questo paragrafo, vengono elencati una serie di possibili manipolazioni, andando a modificare la $f(x,y)$ con un operatore $T$ definito con i suoi vicini di $(x,y)$.
\begin{equation*}
    g(x,y)=T[f(x,y)]
    \label{eq:manipolazione}
\end{equation*}
\begin{figure}[H]
    \centering % Centra l'intero blocco delle immagini e della didascalia
    \begin{minipage}{0.48\textwidth} % Minipage che occupa quasi metà della larghezza del testo
        \centering % Centra l'immagine all'interno della minipage
        \includegraphics[width=\textwidth]{immagini/cap01/ritratto.jpg}
         % L'immagine occupa tutta la larghezza della minipage
        \label{fig:corn}
    \end{minipage}
    \hfill
    \begin{minipage}{0.48\textwidth} % Minipage che occupa quasi metà della larghezza del testo
        \centering % Centra l'immagine all'interno della minipage
        \includegraphics[width=\textwidth]{immagini/cap01/vasi.png} % L'immagine occupa tutta la larghezza della minipage
        \label{fig:griglia}
    \end{minipage}
    \caption{Esempi di possibili manipolazioni di immagini}
    \label{fig:confronto_immaginii}
\end{figure}
In particolare, si considera:\\
\begin{itemize}
    \item $r$ è il livello di grigio di input;
    \item $s$ è il livello di grigio di output.
\end{itemize}
\subsection{Manipolazioni principali}
Di seguito ne è riportato un grafico che ne riporta le principali tipologie di manipolazione delle immagini.
\begin{figure}[H]
        \centering
        \includegraphics[width=1\textwidth]{immagini/cap01/manipolazione.png} 
        \caption{Possibili manipolazioni delle immagini} 
        \label{fig:manipolazione}
\end{figure}
La prima ed anche la più semplice è la funzione identità, in cui il livello di grigio in entrata corrisponde a quello d'uscita, perciò l'immagine non riporta alcuna alterazione. Tale funzione è una retta che va da $(0,0)$ a $(L-1, L-1)$.
\begin{equation*}
    s=r
    \label{eq:identitàa}
\end{equation*}
La seconda è la funzione negativa, in cui inverte i livelli di grigio, creando la cosiddetta immagine negativa. Questa funzione non è altro che una retta che va da $(0,L-1)$ a $(L-1,0)$.
\begin{equation*}
    s=L-1-r
    \label{eq:identità}
\end{equation*}
Tale funzione risulta molto utile per risaltare regioni di grigio incorporate da regioni scure.
\begin{figure}[htbp]
    \centering
    \includegraphics[width=0.7\textwidth]{immagini/cap01/negativo.png} 
    \caption{Immagine originale (a sinistra) e immagine in negativo (a destra)} 
    \label{fig:negativo}
\end{figure}

La terza funzione è la funzione logaritmica, che tende ad espandere i valori dei pixel scuri ed ad comprimere i pixel chiari. Se invece si vuole ottenere l'opposto, allora si usa la funzione logaritmica inversa. La prima equazione è la funzione logaritmica, mentre la seconda è la funzione logaritmica inversa. Entrambi sfruttano la costante di scala $c \in [0,L-1]$.

% --- NESSUNA RIGA VUOTA QUI ---
\begin{align*}
    s &= c\ln{(1+r)} \\
    s &= \exp\left(\frac{r}{c}\right)-1 
\end{align*}
% --- LA RIGA VUOTA DEVE ESSERE QUI (dopo il blocco align*) ---

\begin{figure}[htbp]
    \centering
    \includegraphics[width=0.7\textwidth]{immagini/cap01/log.png} 
    \caption{Esempio di due immagini a cui sono state applicate con $r$ differenti} 
    \label{fig:log}
\end{figure}
L'ultimo gruppo di funzioni riguarda la correzione gamma, che è un'operazione non lineare usata per codificare e decodificare i valori di luminanza in un sistema di visualizzazione di immagini. Esse sfruttano le costanti $c$ e $\gamma$, entrambe positive.
\begin{equation*}
    s=cr^{\gamma}
    \label{eq:gamma}
\end{equation*}
Ciò avviene per due motivi principali:
\begin{itemize}
    \item l'occhio umano non percepisce la luminosità in modo lineare, perciò le immagini risulterebbero scure e con pochi dettagli nelle ombre;
    \item gli schermi più datati, come quelli a tubo catodico, hanno una risposta non lineare alla tensione in ingresso, seguendo una legge di potenza, e si è scelto di mantenere ciò anche per gli schermi più moderni per mantenere compatibilità e coerenza visiva per la percezione umana.
\end{itemize}
\begin{figure}[htbp]
    \centering
    \includegraphics[width=0.6\textwidth]{immagini/cap01/gamma.png} 
    \caption{Realizzazione della correzione gamma sui monitor} 
    \label{fig:gamma}
\end{figure}
Di seguito, viene riportata una figura che dimostra quando sia importante la calibrazione corretta della correzione gamma sui monitor.

\begin{figure}[htbp]
    \centering
    \includegraphics[width=0.7\textwidth]{immagini/cap01/calibrazione.png} 
    \caption{Calibrazione del gamma (azzurro la gamma dell'immagine, viola la gamma del display e la rossa la gamma complessiva)} 
    \label{fig:gamma2}
\end{figure}
Nella prima colonna non è stata applicata alcuna correzione; nella seconda l'applicazione è insufficiente, generando un'immagine troppo chiara; nella terza è l'applicazione ideale; infine nell'ultima l'applicazione è eccessiva generando un'immagine troppo scura.
\clearpage % Forza l'inizio del blocco su una nuova pagina

\subsection{Funzione di trasformazione lineare a tratti}
La funzione di trasformazione lineare a tratti ha la particolarità di non essere descritta da una singola equazione per tutto l'intervallo dei livelli di grigio, ma da più segmenti lineari, ognuno applicabile ad un intervallo di livello di grigio ben specifico. Il vantaggio sta nell'avere una forma con livello di complessità a scelta, a discapito, però, nell'avere più input dall'utente.

La prima funzione di trasformazione lineare a tratti è il contrast stretching, che si pone come obiettivo quello di aumentare il contrasto di un'immagine che appare sbiadita o con scarso intervallo dinamico. A questo punto, si scelgono due punti $(r_1,s_1)$ e $(r_2,s_2)$ in cui i livelli sotto $r_1$ (molto scuri) oppure sopra $r_2$ (molto chiari) vengono compressi, in altre parole vengono mappati rispettivamente quasi a $0$ e a $L-1$. Mentre se è compreso tra $r_1$ ed $r_2$ viene aumentato il contrasto, espandendo i livelli di grigio sull'intero intervallo di uscita.

% --- INIZIO BLOCCO FIGURA 10 (Contrast Stretching) ---
\begin{center}
    % Uso \begin{equation*} e \end{equation*} per contenere l'ambiente align*
    \begin{equation*}
        \begin{aligned}[t] 
        & s = 
            \begin{dcases}
                \frac{s_1}{r_1}r & \text{se } 0 \le r < r_1 \\[1.5ex]
                \frac{s_2-s_1}{r_2-r_1}(r-r_1)+s_1 & \text{se } r_1 \le r \le r_2 \\[1.5ex]
                \frac{L-1-s_2}{L-1-r_2}(r-r_1)+s_2 & \text{se } r_2 < r \le L-1
            \end{dcases}
        &
        \raisebox{-0.5\height}{\includegraphics[width=0.45\textwidth]{cap01/contrast.png}} 
        \end{aligned}
    \end{equation*}
    \captionof{figure}{Definizione analitica e grafico del contrast stretching} % <--- NOTA: Ho usato \captionof (richiede \usepackage{caption})
    \label{fig:contrast_stretching}
\end{center}
% --- FINE BLOCCO FIGURA 10 ---

Un caso limite si ha se avviene un'immagine binaria (bianco o nero), avendo il cosiddetto thresholding.

% --- INIZIO BLOCCO FIGURA 11 (Thresholding) ---
\begin{center}
    \begin{equation*}
        \begin{aligned}[t]
        & s = 
            \begin{cases}
                0 & \text{se } 0 \le r < T \\[1.5ex]
                L-1 & \text{se } T \le r \le L-1
            \end{cases}
        &
        % USARE IL NOME FILE CORRETTO PER IL THRESHOLDING!
        \raisebox{-0.5\height}{\includegraphics[width=0.45\textwidth]{cap01/thresholding.png}} 
        \end{aligned}
    \end{equation*}
    \captionof{figure}{Definizione analitica e grafico del thresholding}
    \label{fig:thresholding}
\end{center}
% --- FINE BLOCCO FIGURA 11 ---
La seconda funzione di trasformazione lineare a tratti è il gray-level slicing, che si pone come obiettivo quello di evidenziare un intervallo di livello di grigio ben specifico e sopprimere tutti gli altri. In particolare si fa riferimento a due casi ben specifici:
\begin{itemize}
    \item sfondo costante, in cui l'intervallo viene mappato ad un livello alto ($L-1$) ed il resto al livello $0$;
    \begin{equation*}
    s=s_{high}\text{rect}_{B-A}\left(r-\frac{A+B}{2}\right)
    \label{eq:sfondocosts}
    \end{equation*}
    \item sfondo invariato, in cui l'intervallo viene mappato ad un livello alto ($L-1$) ed il resto viene lasciato invariato.
    \begin{equation*}
    s=r+(s_{high}-r)\operatorname{rect}_{B-A}\left(r-\frac{A+B}{2}\right)
    \label{eq:sfondoinv}
    \end{equation*}
\end{itemize}
\begin{figure}[htbp]
    \centering
    \includegraphics[width=0.7\textwidth]{immagini/cap01/gray-level-slicing.png} 
    \caption{Gray-level slicing a sfondo costante (a sinistra) e a sfondo invariato (a destra)} 
    \label{fig:grayl}
\end{figure}
La terza ed ultima funzione di trasformazione lineare a tratti è il bit-plane slicing, che consiste nel mettere evidenza solamente i bit più significativi (che sono i primi bit).
\subsection{Equalizzazione dell'istogramma}
Quando si parla di un'immagine, è possibile costruire l'istogramma dell'immagine stessa, che non è altro che il numero di pixel che contengono un determinato valore di livello di grigio per ogni livello $n_k$.
\begin{center}
    % Uso \begin{equation*} e \end{equation*} per contenere l'ambiente align*
    \begin{equation*}
        \begin{aligned}[t] 
        & 
            f(x,y) = \begin{bmatrix}
            2 & 3 & 3 & 2 \\
             4 & 2 & 4 & 3 \\
             3 & 2 & 3 & 5 \\
              2 & 4 & 2 & 4
            \end{bmatrix}
        &
        \raisebox{-0.5\height}{\includegraphics[width=0.45\textwidth]{cap01/istogramma.png}} 
        \end{aligned}
    \end{equation*}
    \captionof{figure}{Esempio di istogramma di un'immagine 4x4 con livello di grigio $[0,9]$} % <--- NOTA: Ho usato \captionof (richiede \usepackage{caption})
    \label{fig:kjmkpl}
\end{center}
In termini statistici, è preferibile rappresentare l'istogramma normalizzato, ossia dividere $n_k$ con il numero totale di pixel $n$.
\begin{equation*}
    p(n_k)=\frac{n_k}{n}
\end{equation*}
A questo punto, la tecnica dell'equalizzazione dell'istogramma consiste nel cambiare l'istogramma dell'immagine in un istogramma uniforme, in cui la percentuale di ogni livello di grigio rimane sempre la stessa. Per effettuare ciò, occorre dei fondamenti di probabilità e statistica e di calcolo integrale.\\
Per prima cosa, questa operazione effettua una trasformata, perciò si può scrivere ciò come $s=T(r)$ e siccome la trasformata è reversibile, esiste anche la sua inversa $r=T^{-1}(s)$. Per semplicità, risulta molto conveniente studiare ciò nel continuo. Dati rispettivamente, $p_{in}(r)$ e $p_{out}(s)$, come la probabilità di livello di grigi di input e di outut, dalla teoria di probabilità, si ha la formula seguente (per $0\le r \le L-1$ e $0\le s \le L-1$).
\begin{equation*}
    p_{out}(s)=\left[p_{in}(r)\frac{ds}{dr}\right]_{r=T^{-1}(s)}
\end{equation*}
A questo punto, la trasformata di $p_{in}(r)$ è pari alla formula seguente:
\begin{equation*}
    s=T(r)=\int_{0}^{r} p_{in}(r') \, dr' , 0\le r \le 1
\end{equation*}
che è la funzione di distribuzione cumulativa (CDF). Perciò, dal teorema fondamentale del calcolo, si sfrutta la seguente formula.
\begin{equation*}
    p_{in}(r)=\frac{dr}{ds}
\end{equation*}
Infine, facendo dei semplici conti e sfruttando le proprietà delle derivate delle funzioni inverse, si ricava che:
\begin{equation*}
    p_{out}(s)=\left[p_{in}(r)\frac{ds}{dr}\right]_{r=T^{-1}(s)}=\left[p_{in}(r)\frac{1}{\frac{dr}{ds}}\right]_{r=T^{-1}(s)}=\left[p_{in}(r)\frac{1}{[p_{in}(r)}\right]_{r=T^{-1}(s)}=\left[1\right]_{r=T^{-1}(s)}=1, 0\le s \le 1
\end{equation*}
la densità di probabilità in uscita risulta uniforme.\\
Quindi per effettuare l'equalizzazione dell'istogramma:
\begin{enumerate}
    \item per ogni pixel si calcola il $p_{in}(r_k)$;
    \begin{equation*}
    p_{in}(r_k)=\frac{n_k}{n}, 0\le r_k \le 1 \,\, 0\le k \le L-1
    \label{eq:sfondocost}
    \end{equation*}
    \item basandosi sulla CDF, si esegue la trasformata discreta.
    \begin{equation*}
    s_k=T_{r_{k}}=\sum_{j=0}^{k}{p_{in}(r_j)},0\le k \le L-1
    \label{eq:lkl}
    \end{equation*} 
\end{enumerate}
Tornando all'esempio precedente, si effettua la CDF nel modo seguente.
\begin{center}
    % Il blocco matematico generale per l'allineamento
    \begin{equation*} 
        \begin{aligned}[t] 
        % COLONNA SINISTRA: La Matrice (usiamo bmatrix per le quadre)
        & f(x,y) = \begin{bmatrix}
            2 & 3 & 3 & 2 \\
            4 & 2 & 4 & 3 \\
            3 & 2 & 3 & 5 \\
            2 & 4 & 2 & 4
          \end{bmatrix}
        &
        % COLONNA DESTRA: La Tabella (NON USIAMO \includegraphics o \begin{table})
        % Usiamo \vcenter per l'allineamento verticale con la matrice
        \vcenter{\hbox{
            \begin{tabular}{|c|c|c|c|}
                \hline 
                $k$ & $r_k$ & $n_k$ & $p_{in}(r_k)$ \rule{0pt}{2ex}\\ 
                \hline 
                $2$ & $0$ & $6$ & $3/8$ \rule{0pt}{2ex}\\
                \hline 
                $3$ & $1/3$ & $5$ & $5/16$ \rule{0pt}{2ex}\\
                \hline 
                $4$ &$2/3$ & $4$ & $1/4$ \rule{0pt}{2ex}\\
                \hline 
                $5$ & $1$ & $1$ & $1/16$ \rule{0pt}{2ex}\\
                \hline 
            \end{tabular}
        }} % Fine \vcenter{\hbox{...
        \end{aligned}
    \end{equation*}
    
    \captionof{figure}{Calcolo dell'istogramma normalizzato}
    \label{fig:iji}
\end{center}
\begin{equation*}
    s_2=p_{in}(r_2)=\frac{3}{8}\to0
\end{equation*}
\begin{equation*}
    s_3=p_{in}(r_3)=\frac{3}{8}+\frac{5}{16}=\frac{11}{16}\to\frac{2}{3}
\end{equation*}
\begin{equation*}
    s_4=p_{in}(r_4)=\frac{3}{8}+\frac{5}{16}+\frac{1}{4}=\frac{15}{16}\to1
\end{equation*}
\begin{equation*}
    s_5=p_{in}(r_5)=\frac{3}{8}+\frac{5}{16}+\frac{1}{4}+\frac{1}{16}=1\to1
\end{equation*}
Perciò:
\begin{itemize}
    \item al livello $2$ si associa il livello $2$;
    \item al livello $3$ si associa il livello $4$;
    \item al livello $4$ si associa il livello $5$;
    \item al livello $5$ si associa il livello $5$.
\end{itemize}
A questo punto, si ottiene l'immagine equalizzata, come mostrato di seguito.
\begin{center}
    % Uso \begin{equation*} e \end{equation*} per contenere l'ambiente align*
    \begin{equation*}
        \begin{aligned}[t] 
        & 
            g(x,y) = \begin{bmatrix}
            2 & 4 & 4 & 2 \\
             5 & 2 & 5 & 4 \\
             4 & 2 & 4 & 5 \\
              2 & 5 & 2 & 5
            \end{bmatrix}
        &
        \raisebox{-0.5\height}{\includegraphics[width=0.45\textwidth]{cap01/istogramma2.png}} 
        \end{aligned}
    \end{equation*}
    \captionof{figure}{Esempio di istogramma di un'immagine equalizzata} % <--- NOTA: Ho usato \captionof (richiede \usepackage{caption})
    \label{fig:kjmkp}
\end{center}
\vfill
\chapter{Filtri nel dominio spaziale}
Nelle telecomunicazioni, per filtrare un segnale nel dominio del tempo, si usa l'operazione di convoluzione.
\begin{center}
\begin{minipage}{0.48\textwidth} % Colonna Sinistra (48% della larghezza del testo)
    \centering
    \begin{equation*}
        x(t)*h(t)=\int_{-\infty}^{\infty}h(\tau)x(t-\tau)\, d\tau
    \end{equation*}
\end{minipage}
\hfill % Spazio elastico per separare le due colonne
\begin{minipage}{0.48\textwidth} % Colonna Destra (48% della larghezza del testo)
    \centering
    \begin{equation*}
        x[n]*h[n]=\sum_{i=-\infty}^{\infty}h[i]x[n-i]
    \end{equation*}
\end{minipage}
\end{center}
Per quanto riguarda le immagini, nel dominio spaziale, la convoluzione avviene in due dimensioni (somma del prodotto elemento per elemento delle due matrici), dove:
\begin{center}
\begin{minipage}{0.48\textwidth} % Colonna Sinistra (48% della larghezza del testo)
    \centering
    \begin{itemize}
    \item l'immagine è il segnale d'ingresso;
    \item il filtro è il nucleo della convoluzione, detto maschera.
    \end{itemize}
\end{minipage}
\hfill % Spazio elastico per separare le due colonne
\begin{minipage}{0.48\textwidth} % Colonna Destra (48% della larghezza del testo)
    \centering
    \begin{equation*}
    g(x,y)=\sum_{i=-\infty}^{\infty}\sum_{j=-\infty}^{\infty}h(i,j)f(x-i,y-j)
    \end{equation*}
\end{minipage}
\end{center}
Per esempio, data l'immagine e il filtro seguente:
\begin{center}
\begin{minipage}{0.14\textwidth} % Colonna Sinistra (48% della larghezza del testo)
    \centering
    \begin{equation*}
        f(x,y) = \begin{bmatrix}
            5 & 8 & 3 & 4 \\
            3 & 2 & 1 & 1 \\
            0 & 9 & 5 & 3 \\
            4 & 2 & 7 & 2
        \end{bmatrix}
    \end{equation*}
\end{minipage}
\hfill % Spazio elastico per separare le due colonne
\begin{minipage}{0.14\textwidth} % Colonna Destra (48% della larghezza del testo)
    \centering
    \begin{equation*}
    h(x,y) = \begin{bmatrix}
            2 & 1 & 0 \\
            1 & 1 & -1 \\
            0 & -1 & -2
        \end{bmatrix}
    \end{equation*}
\end{minipage}
\hfill % Spazio elastico per separare le due colonne
\begin{minipage}{0.45\textwidth} % Colonna Destra (48% della larghezza del testo)
    \centering
    \begin{equation*}
    h_{flip}(x,y) = -\begin{bmatrix}
            2 & 1 & 0 \\
            1 & 1 & -1 \\
            0 & -1 & -2
        \end{bmatrix}=
        \begin{bmatrix}
            -2 & -1 & 0 \\
            -1 & -1 & 1 \\
            0 & 1 & 2
         \end{bmatrix}   
    \end{equation*}
\end{minipage}
\end{center}
si mette all'inizio in posizione $(0,0)$ dell'immagine e quello è il punto centrale dell'immagine. Come si può notare, sono presenti dei punti dell'immagine che vanno fuori dall'immagine. Per evitare ciò, è possibile:
\begin{itemize}
    \item ignorare i bordi, partendo dai punti in cui si ha una piena sovrapposizione dell'immagine;
    \item si assumono i valori fuori dai bordi pari a 0.
\end{itemize}
\begin{equation*}
    g(0,0)=\begin{bmatrix}
            -2 & -1 & 0 \\
            -1 & -1 & 1 \\
            0 & 1 & 2
         \end{bmatrix}*\begin{bmatrix}
            0 & 0 & 0 \\
            0 & 5 & 8 \\
            0 & 3 & 5
        \end{bmatrix}=-1\times 5 + 1 \times 8 + 1 \times 3 + 5 \times 5 = 20
\end{equation*}
L'immagine finale, ponendo i punti fuori dall'immagine pari a $0$, è dunque la seguente.
\begin{equation*}
    g(x,y)=\begin{bmatrix}
            20 & 10 & 2 & 2 \\
            18 & 1 & -8 & -7 \\
            14 & 22 & 5 & -3 \\
            6 & -4 & -16 & -18
         \end{bmatrix}
\end{equation*}
Nel caso in cui si ignorassero i bordi, si applicherebbe il filtro solamente nei punti $(1,1)$, $(1,2)$, $(2,1)$ e $(2,2)$. Come si può intuire, l'immagine di output viene tagliata.
\begin{equation*}
    g(x,y)=\begin{bmatrix}
            3 & 10\\
            -4 & 5  
         \end{bmatrix}
\end{equation*}
Infine, eseguire convoluzioni con maschere risulta essere molto versatile, poiché a seconda dei coefficienti della maschera, si possono ottenere risultati differenti: per esempio la sfocatura, il contrasto od il rilevamento dei bordi. In questo capitolo, vengono spiegati i filtri di smussamento e di nitidezza.
\section{Filtri di smussamento}
I filtri di smussamento permettono di gestire la sfocatura dell'immagine e della riduzione del rumore. In particolare, la sfocatura è un processo che serve per rimuovere piccoli dettagli e colmare piccoli scalini in linee e curve. La sfocatura accompagna la riduzione del rumore. 
\subsection{Filtri di smussamento lineari}
I filtri di smussamento lineari applicano la media dei pixel del vicinato. Essi ripiazzano il valore di ogni pixel con la media dei livello di grigio definiti dalla maschera. In particolare, a seconda dei valori della maschera si hanno tre tipologie di filtro.\\
La prima tipologia è il filtro media aritmetica, in cui la maschera contiene il prodotto scalare tra l'inverso del numero di elementi della maschera e la maschera stessa di tutti $1$.
\begin{center}
\begin{minipage}{0.48\textwidth} % Colonna Sinistra (48% della larghezza del testo)
    \centering
    \begin{equation*}
        h(x,y) = \frac{1}{9}\begin{bmatrix}
            1 & 1 & 1 \\
            1 & 1 & 1 \\
            1 & 1 & 1
        \end{bmatrix}
    \end{equation*}
\end{minipage}
\hfill % Spazio elastico per separare le due colonne
\begin{minipage}{0.48\textwidth} % Colonna Destra (48% della larghezza del testo)
    \centering
    \begin{equation*}
         h(x,y) = \frac{1}{25}\begin{bmatrix}
            1 & 1 & 1 & 1\\
            1 & 1 & 1 & 1\\
            1 & 1 & 1 & 1\\
            1 & 1 & 1 & 1
        \end{bmatrix}
    \end{equation*}
\end{minipage}
\end{center}
Tale filtro applica una sfocatura uniforme all'immagine: più la maschera contiene elementi, maggiore sarà la sfocatura.
La seconda tipologia è il filtro media ponderata, in cui si vuole dare più importanza a dei pixel rispetto ad altri, ad esempio il pixel centrale.
\begin{equation*}
        h(x,y) = \begin{bmatrix}
            3/40 & 1/8 & 3/40 \\
            1/8 & 1/5 & 1/8 \\
            3/40 & 1/8 & 3/40
        \end{bmatrix}
\end{equation*}
La terza ed ultima tipologia è il filtro gaussiano, in cui la maschera è una gaussiana a due dimensioni, in cui la deviazione standard ($\sigma$) ne determina la larghezza della campana.
\begin{equation*}
        h(x,y) = \exp{\left[\frac{-(x^2+y^2)}{2\sigma^2}\right]}
\end{equation*}
La deviazione standard controlla l'intensità della sfocatura: più è alto più sfoca. Comunque, essa sfoca molto meno brutalmente rispetto al filtro media.
\begin{figure}[htbp]
        \centering
        \includegraphics[width=0.4\textwidth]{cap02/men_gauss} 
        \caption{Filtro media e gaussiano} 
        \label{fig:f1}
\end{figure}
\subsection{Filtri di smussamento non lineari}
I filtri di smussamento non lineari non presentano un'operazione di convoluzione vera e propria, ma si basano su ordinamento statistico: in particolare, in base al ranking dei pixel. I pixel che sono considerati non rappresentativi vengono eliminati. Inoltre, vengono sostituiti i valori dei pixel con il valore della classifica. Questa tipologia viene usata per la riduzione del rumore, sfocando però l'immagine.\\
Il filtro di smussamento non lineare più comune è il filtro mediano, che consiste nel:
\begin{enumerate}
    \item prendere una porzione di un'immagine (solitamente 3x3 o 5x5);
    \begin{equation*}
        \begin{bmatrix}
            9 & 12 & 0 \\
            5 & 5 & 9 \\
            8 & 10 & 7
        \end{bmatrix}
    \end{equation*}
    \item convertire tale porzione in un array ordinato;
    \begin{equation*}
        \left\{0, 5, 5, 7, 8, 9, 9, 10, 12\right\} \to 8
    \end{equation*}
    \item sostituire la mediana dell'array nella porzione presa, sfocando l'immagine;
    \begin{equation*}
        \begin{bmatrix}
            8 & 8 & 8 \\
            8 & 8 & 8 \\
            8 & 8 & 8
        \end{bmatrix}
    \end{equation*}
    \item iterare i passaggi precedenti per tutta l'immagine.
\end{enumerate}
Questo filtro permette di rimuovere il rumore sale e pepe, molto comune nelle foto più datate, in cui l'immagine presenta dei pixel completamente bianchi e neri: infatti, tali pixel non saranno mai dei mediani: per questo motivo vengono eliminati. Inoltre, la sfocatura non è così evidente, un altro lato positivo da non sottovalutare.
\begin{figure}[htbp]
        \centering
        \includegraphics[width=0.85\textwidth]{cap02/salt} 
        \caption{Rimozione del rumore sale e pepe con un filtro mediano (sinistra originale e destra modificata)} 
        \label{fig:f2}
\end{figure}
\section{Filtri di nitidezza}
I filtri di shapening preservano i dettagli, andando ad evidenziare i bordi. Per fare ciò, è necessario trovare quella operazione che permetta di distinguire quali sono i pixel dell'immagine che sono uguali oppure molto simili. Tale operazioni sono:
\begin{itemize}
    \item la derivata prima, che esegue la differenza tra il pixel successivo ed il pixel corrente;
    \begin{equation*}
        \frac{\delta f}{\delta x}=f(x+1)-f(x)
    \end{equation*}
    \item la derivata seconda, che esegue la somma tra il pixel successivo ed il pixel precedente e infine sottrae due volte il pixel corrente.
    \begin{equation*}
        \frac{\delta^2 f}{\delta x^2}=f(x+1)+f(x-1)-2f(x)
    \end{equation*}
\end{itemize}
Per tali scopi, risulta molto conveniente usare la derivata seconda, poiché presenta una risposta più forte per i dettagli (facilmente intuibile) ed ha un'implementazione decisamente più semplice (poco intuibile).
\subsection{Il Laplaciano}
Il filtro di nitidezza più usato è il Laplaciano, che sfrutta il gradiente secondo dell'immagine: ciò è decisamente logico, dato che in più di una dimensione e con le derivate parziali, si fa riferimento ad ogni derivata parziale, perciò è necessario il gradiente.
\begin{equation*}
        \nabla^2 f = \frac{\delta^2 f}{\delta x^2}+\frac{\delta^2 f}{\delta y^2}
\end{equation*}
Con dei calcoli matematici si ricava la formula seguente:
\begin{equation*}
        \frac{\delta^2 f}{\delta x^2}=f(x+1,y)+f(x-1,y)-2f(x,y) \\
\end{equation*}
\begin{equation*}
        \frac{\delta^2 f}{\delta y^2}=f(x,y+1)+f(x,y-1)-2f(x,y) \\
\end{equation*}
\begin{equation*}
        \nabla^2 f=f(x+1,y)+f(x-1,y)+f(x,y+1)+f(x,y-1)-4f(x,y)
\end{equation*}
da cui si ricava la maschera, ricavando i coefficienti del gradiente.
\begin{equation*}
       \nabla^2 f=\begin{bmatrix}
            f(x-1,y-1) & f(x-1,y) & f(x-1,y+1) \\
           f(x,y-1) & f(x,y) & f(x,y+1) \\
            f(x+1,y-1) & f(x+1,y) & f(x+1,y+1) \\
        \end{bmatrix}=\begin{bmatrix}
            0 & 1 & 0 \\
            1 & -4 & 1 \\
            0 & 1 & 0
        \end{bmatrix}
\end{equation*}
Tuttavia, tale filtro trova solamente i bordi dell'immagine. Per questo motivo, l'immagine di output è solamente  uno dei fattori della somma algebrica con l'immagine originale.
\begin{equation*}
        g(x,y)=f(x,y)+c\nabla^2\left[(x,y)\right]
\end{equation*}
In particolare, $c=1$ se il coefficiente centrale è positivo; $c=-1$ altrimenti.
A questo punto, la nuova immagine è la seguente.
\begin{equation*}
        g(x,y)=f(x,y)-f(x+1,y)-f(x-1,y)-f(x,y+1)-f(x,y-1)+4f(x,y)
\end{equation*}
\begin{equation*}
        g(x,y)=5f(x,y)-f(x+1,y)-f(x-1,y)-f(x,y+1)-f(x,y-1)
\end{equation*}
Perciò, la maschera è la seguente.
\begin{equation*}
        h(x,y)=\begin{bmatrix}
            0 & -1 & 0 \\
            -1 & 5 & -1 \\
            0 & -1 & 0
        \end{bmatrix}
\end{equation*}
Infatti, ora si comprende il motivo della sua semplicità di implementazione. Inoltre, il Laplaciano presenta i seguenti vantaggi:
isotropico:perciò è indipendente dal punto in cui viene applicato, poiché applica le maschere circolari, quindi la rotazione resta invariata.
\begin{figure}[H]
        \centering
        \includegraphics[width=0.85\textwidth]{cap02/laplaciano.png} 
        \caption{Applicazione del Laplaciano (sinistra originale e destra modificata)} 
        \label{fig:f3}
\end{figure}
\subsection{Filtri di contrasto}
 Se serve capire una scala di dettagli da evidenziare all'immagine, piuttosto che sfruttare un'operazione fissa, conviene usare i filtri di contrasto (in inglese unsharp mask). Ciò consiste in:
 \begin{enumerate}
     \item si applica un filtro Gaussiano;
     \begin{equation*}
        \bar{f}(x,y)=f(x,y)*\exp{\left[\frac{-(x^2+y^2)}{2\sigma^2}\right]}
    \end{equation*}
    \item si sottrae l'immagine originale con l'immagine filtrata con il filtro gaussiano, ottenendo la maschera dell'immagine;
    \begin{equation*}
        g_{mask}(x,y)=f(x,y)-\bar{f}(x,y)
    \end{equation*}
    \item si aggiunge all'immagine originale la maschera dell'immagine moltiplicata per un fattore di scala di dettagli $k$.
    \begin{equation*}
        g(x,y)=f(x,y)+kg_{mask}(x,y)
    \end{equation*}
 \end{enumerate}
 
 Ciò si dimostra nel modo seguente.
 \begin{equation*}
        g(x,y)=f(x,y)+kg_{mask}(x,y)=f(x,y)+k[f(x,y)-\bar{f}(x,y)]=(1+k)f(x,y)-\bar{f}(x,y)
\end{equation*}
Siccome l'immagine mascherata è un'ottima approssimazione del Laplaciano ed il filtro guassiano si può approssimare con un filtro medio 3x3, si approssima $c \approx 1 + k$, che prende il nome di fattore di amplificazione.
\begin{equation*}
\begin{bmatrix}
            -1 & -1 & -1 \\
            -1 & c & -1 \\
            -1 & -1 & -1
        \end{bmatrix}
\end{equation*}
Ciò prende il nome di high-boost filtering. 
\begin{figure}[H]
        \centering
        \includegraphics[width=0.9\textwidth]{cap02/nitidezza.png} 
        \caption{Originale, Laplaciano filtro di contrasto e high-boost filtering da in alto a sinistra in senso orario} 
        \label{fig:f4k}
\end{figure}
\subsection{Filtri di rilevamento dei bordi}
Per rilevare i bordi, sfruttare la derivata seconda è molto semplice da implementare. Tuttavia, presenta un problema da non sottovalutare: è molto sensibile al rumore. Per questo motivo, se l'obiettivo è quello di rilevare i bordi, conviene invece usare la derivata prima. Inoltre, in matematica, la derivata prima rileva i picchi: per questo motivo, è l'operazione che permette di rilevare i bordi. Tuttavia, va eseguito sia per le righe che per le colonne.\\
Il primo nucleo prende il nome di operatori di Roberts, in cui si approssima attraverso l'equivalenza digitale della derivata del primo ordine, prendendo in considerazione la differenza tra il pixel successivo e il pixel precedente lungo la riga e lungo la colonna.
\begin{center}
\begin{minipage}{0.48\textwidth} % Colonna Sinistra (48% della larghezza del testo)
    \centering
    \begin{equation*}
        h_x(x,y) \approx f(x+1,y) - f(x-1,y)
    \end{equation*}
\end{minipage}
\hfill % Spazio elastico per separare le due colonne
\begin{minipage}{0.48\textwidth} % Colonna Destra (48% della larghezza del testo)
    \centering
    \begin{equation*}
        h_y(x,y) \approx f(x,y+1) - f(x,y-1)
    \end{equation*}
\end{minipage}
\end{center}
Ciò si può rappresentare anche con due matrici 2x2.
\begin{center}
\begin{minipage}{0.48\textwidth} % Colonna Sinistra (48% della larghezza del testo)
    \centering
    \begin{equation*}
    h_x=\begin{bmatrix}
        0 & -1 \\
        1 & 0
    \end{bmatrix}
\end{equation*}
\end{minipage}
\hfill % Spazio elastico per separare le due colonne
\begin{minipage}{0.48\textwidth} % Colonna Destra (48% della larghezza del testo)
    \centering
    \begin{equation*}
    h_y=\begin{bmatrix}
        -1 & 0 \\
        0 & 1
    \end{bmatrix}
\end{equation*}
\end{minipage}
\end{center}
Di seguito, è riportato un esempio con l'immagine 3x3 seguente.
\begin{equation*}
    f(x,y)=\begin{bmatrix}
        10 & 20 & 30\\
        40 & 50 & 60\\
        70 & 80 & 90
    \end{bmatrix}
\end{equation*}
Adesso, si applicano le due convoluzioni: viene mostrato solo un prodotto per riga e per colonna.
\begin{equation*}
    h_x(1,1)=\begin{bmatrix}
        0 & -1 \\
        1 & 0
    \end{bmatrix}*\begin{bmatrix}
        10 & 20 \\
        40 & 50
    \end{bmatrix}=0 \times 10 + (-1) \times 20 + 1 \times 40 + 0 \times 50 = 20
\end{equation*}
\begin{equation*}
    h_y(1,1)=\begin{bmatrix}
        -1 & 0 \\
        0 & 1
    \end{bmatrix}*\begin{bmatrix}
        10 & 20 \\
        40 & 50
    \end{bmatrix}= -1 \times 10 + 0 \times 20 + 0 \times 40 + 1 \times 50 = 40
\end{equation*}
Tuttavia, si preferisce usare la coppia di kernel di Prewitt e di Sobel, riportati di seguito rispettivamente.
\begin{center}
\begin{minipage}{0.11\textwidth} % Colonna Sinistra (48% della larghezza del testo)
    \centering
    Prewitt
\end{minipage}
\hfill
\begin{minipage}{0.31\textwidth} % Colonna Sinistra (48% della larghezza del testo)
    \centering
    \begin{equation*}
    h_x=\begin{bmatrix}
        -1 & 0 & 1\\
        -1 & 0 & 1\\
        -1 & 0 & 1
    \end{bmatrix}
\end{equation*}
\end{minipage}
\hfill % Spazio elastico per separare le due colonne
\begin{minipage}{0.31\textwidth} % Colonna Destra (48% della larghezza del testo)
    \centering
    \begin{equation*}
    h_y=\begin{bmatrix}
        -1 & -1 & -1\\
        0 & 0 & 0\\
        1 & 1 & 1
    \end{bmatrix}
\end{equation*}
\end{minipage}
\end{center}
\begin{center}
\begin{minipage}{0.11\textwidth} % Colonna Sinistra (48% della larghezza del testo)
    \centering
    Sobel
\end{minipage}
\hfill
\begin{minipage}{0.31\textwidth} % Colonna Sinistra (48% della larghezza del testo)
    \centering
    \begin{equation*}
    h_x=\begin{bmatrix}
        -1 & 0 & 1\\
        -2 & 0 & 2\\
        -1 & 0 & 1
    \end{bmatrix}
\end{equation*}
\end{minipage}
\hfill % Spazio elastico per separare le due colonne
\begin{minipage}{0.31\textwidth} % Colonna Destra (48% della larghezza del testo)
    \centering
    \begin{equation*}
    h_y=\begin{bmatrix}
        -1 & -2 & -1\\
        0 & 0 & 0\\
        1 & 2 & 1
    \end{bmatrix}
\end{equation*}
\end{minipage}
\end{center}
Tali nuclei condividono le seguenti due proprietà:
\begin{itemize}
    \item hanno i coefficienti opposti per avere un'alta risposta nella regione dell'immagine con molta variazione d'intensità, in cui la probabilità che ci sia un bordo è molto alta;
    \item la somma dei coefficienti è pari a $0$: ciò significa che quando è applicato ad un'immagine con una regione completamente omogenea, il risultato è $0$.
\end{itemize}
Una volta fatto ciò, si calcolano:
\begin{itemize}
    \item la magnitudine, che misura quanto è forte il bordo;
    \begin{equation*}
        h=\sqrt{h_x^2+h_y^2}
    \end{equation*}
    \item la direzione, che misura l'orientamento del bordo.
    \begin{equation*}
        \theta =  \operatorname{atan2}(h_y, h_x)=
        \begin{dcases}
               \arctan\left(\frac{h_y}{h_x}\right) & \text{se } h_x > 0 \\[1.5ex]
                \arctan\left(\frac{h_y}{h_x}\right)+\pi & \text{se } h_x < 0 \land h_y \ge 0 \\[1.5ex]
                \arctan\left(\frac{h_y}{h_x}\right)-\pi & \text{se } h_x < 0 \land h_y < 0 \\[1.5ex]
                \frac{\pi}{2} & \text{se } h_x = 0 \land h_y > 0 \\[1.5ex]
                -\frac{\pi}{2} & \text{se } h_x = 0 \land h_y < 0 \\[1.5ex]    0 & \text{se } h_x = 0 \land h_y = 0 
            \end{dcases}
    \end{equation*}
\end{itemize}
Combinando questi due risultati, è possibile comprendere bene la forza effettiva del bordo.\\
A seconda del kernel utilizzato, è possibile applicare dei filtri di rilevamento dei bordi. In questo corso ne vengono analizzati di due tipologie.\\
Il primo, che è anche quello più potente disponibile ad oggi, è il rilevatore di bordi di Canny. L'algoritmo consiste nei seguenti passaggi:\\
\begin{enumerate}
    \item si applica un filtro gaussiano per ridurre il rumore;
    \begin{equation*}
        \bar{f}(x,y)=f(x,y)*\exp{\left[\frac{-(x^2+y^2)}{2\sigma^2}\right]}
    \end{equation*}
    \item si calcolano i gradienti locali con la coppia di kernel di Sobel;
    \begin{center}
    \begin{minipage}{0.46\textwidth} % Colonna Sinistra (48% della larghezza del testo)
        \centering
        \begin{equation*}
        h_x=\begin{bmatrix}
            -1 & 0 & 1\\
            -2 & 0 & 2\\
            -1 & 0 & 1
        \end{bmatrix}
    \end{equation*}
    \end{minipage}
    \hfill % Spazio elastico per separare le due colonne
    \begin{minipage}{0.46\textwidth} % Colonna Destra (48% della larghezza del testo)
        \centering
        \begin{equation*}
        h_y=\begin{bmatrix}
            -1 & -2 & -1\\
            0 & 0 & 0\\
            1 & 2 & 1
        \end{bmatrix}
    \end{equation*}
    \end{minipage}
    \end{center}
     \begin{center}
    \begin{minipage}{0.46\textwidth} % Colonna Sinistra (48% della larghezza del testo)
        \centering
        \begin{equation*}
        h=\sqrt{h_x^2+h_y^2}
    \end{equation*}
    \end{minipage}
    \hfill % Spazio elastico per separare le due colonne
    \begin{minipage}{0.46\textwidth} % Colonna Destra (48% della larghezza del testo)
        \centering
        \begin{equation*}
        \theta =  \operatorname{atan2}(h_y, h_x)
    \end{equation*}
    \end{minipage}
    \end{center}
    \item avviene la cosiddetta soppressione dei non-massimi, in cui si guarda la direzione e si confronta la magnitudine calcolata con i suoi vicini, che viene mantenuta solamente se è un massimo locale;
    \item si effettua un'ulteriore processo decisionale a soglia, per classificare un pixel come bordo forte ($h(p) > T_{alta}$), bordo debole ($h(p) < T_{bassa}$) oppure un non bordo ($T_{bassa} \le h(p) \le T_{alta}$);
    \item si prendono solo i bordi forti oppure i bordi deboli adiacenti (anche in diagonale) con un bordo forte, il resto dei bordi deboli vengono scartati.
\end{enumerate}
\begin{figure}[htbp]
        \centering
        \includegraphics[width=0.95\textwidth]{cap02/bordi} 
        \caption{Immagine originale (a sinistra) e immagine con i bordi rilevati con il rilevatore di Canny ( a destra)} 
        \label{fig:f4}
\end{figure}
Infine, la seconda ed ultima tipologia di rilevamento dei bordi è la trasformata di Hough, che sfrutta un metodo matematico per rilevare linee in un'immagine. Ciò, può essere usata per rilevare bordi potenzialmente sparsi, rotti od isolati; in linee utili, corrispondenti ai bordi dell'immagine. Tale processo consiste nel prendere un punto di coordinata $(x,y)$ e memorizzare tutti i punti appartenenti alla retta $y=ax+b$. Ad esempio data il punto $(x,y)=(1,3)$ si ottengono i punti alle coordinate $3=a+b$, perciò a $b=3-a$, come $(x,y)=(0,3)$.
Ripetendo ciò per gli altri punti, si ottengono una serie di punti che intersecano varie linee: i punti che hanno più intersezioni corrispondono alle linee più lunghe dell'immagine.\\
Tuttavia, le linee verticali hanno gradiente infinito: per risolvere ciò si usano le coordinate normali.
\begin{equation*}
    \rho = x \cos{\theta}+y\sin{\theta}
\end{equation*}
A questo punto, le linee più rilevanti dell'immagine, corrispondono ai valori più elevati di $(\rho, \theta)$.
\vfill

\chapter{Filtri nel dominio della frequenza}
\input{capitoli/cap03}
\chapter{Trasformata Wavelet}
\input{capitoli/cap04}
\chapter{Processo di compressione delle immagini}
Il processo di compressione di un'immagine, consiste nel ridurre drasticamente la quantità di dati presenti nell'immagine, poiché trasmettere un'immagine
non compressa richiede molto tempo e molta larghezza di banda, come mostrato nell'immagine seguente.
\begin{figure}[htbp]
        \centering
        \includegraphics[width=0.9\textwidth]{cap05/compressione} 
        \caption{Processo di trasmissione di un'immagine compressa}
        \label{fig:hansform}
\end{figure}
Basti pensare che già un'immagine $1024\times1024$ in scala di grigi con risoluzione bassa ($8 \, \text{b/p}$), per essere trasmessa in 2G ($v=50 \, \text{kb/s}$), sono necessari:
\begin{center}
\begin{minipage}{0.48\textwidth} % Colonna Sinistra (48% della larghezza del testo)
    \centering
    \begin{equation*}
    c=1024\times1024\times8=8388608 \, \text{b}
\end{equation*}
\end{minipage}
\hfill % Spazio elastico per separare le due colonne
\begin{minipage}{0.48\textwidth} % Colonna Destra (48% della larghezza del testo)
    \centering
        \begin{equation*}
    b=\frac{c}{v}=\frac{8388608}{50000} = 167,77216 \, \text{s} \approx 3'
    \end{equation*}
\end{minipage}
\end{center}
che è decisamente troppo elevato.
\section{Ridondanza dei dati}
Per prima cosa è necessario ribadire che dato ed informazione non sono assolutamente sinonimi. Infatti:
\begin{itemize}
    \item un'informazione è ciò che si vuole trasmettere;
    \item un dato rappresenta i modi di trasmettere l'informazione.
\end{itemize}
La ridondanza dei dati si può quantificare matematicamente come entità. Infatti, si considerano le seguenti grandezze:
\begin{itemize}
    \item rapporto di compressione ($C_R$), che è il rapporto tra due insiemi di dati che contengono la stessa informazione;
    \begin{equation*}
        C_{R} = \frac{n_1}{n_2}
    \end{equation*}
    \item ridondanza relativa dei dati ($R_D$), che indica la quantità dei dati che sono ridondanti, spesso indicati in percentuale.
    \begin{equation*}
        R_D = 1-\frac{1}{C_R}
    \end{equation*}
\end{itemize}
In questo paragrafo, verranno trattate le diverse tipologie di ridondanza.
\subsection{Ridondanza di codifica}
La ridondanza di codifica si ha nel momento in cui si sceglie un sistema di codifica non efficiente. Per calcolare l'efficienza di un
algoritmo di compressione, dipende dal livello medio di grigio, che dipende dalla probabilità che ogni livello di grigio dell'immagine
sia presente in tale immagine e quanti bit occupa ($l(r_k)$).
\begin{center}
\begin{minipage}{0.48\textwidth} % Colonna Sinistra (48% della larghezza del testo)
    \centering
    \begin{equation*}
        p_r(r_k)=\frac{n_k}{MN}
    \end{equation*}
\end{minipage}
\hfill % Spazio elastico per separare le due colonne
\begin{minipage}{0.48\textwidth} % Colonna Destra (48% della larghezza del testo)
    \centering
    \begin{equation*}
        L_{avg}=\sum_{0}^{L-1}l(r_k)p_r(r_k)
    \end{equation*}
\end{minipage}
\end{center}
Per esempio data un'immagine $256 \times 256$ con le seguenti caratteristiche:
\begin{table}[H] % Usa [H] per renderla non flottante
    \centering
    \begin{tabular}{| C{5cm} | C{5cm} |} 
            \hline
            
            % Usiamo \thead per le intestazioni:
            \textbf{$r_k$} & \textbf{$p_k(r_k)$} \\ 
            \hline % Linea spessa sotto le intestazioni
            $r_{87}=87$ & $3/10$ \\
            \hline % Linea spessa sotto le intestazioni
            $r_{128}=128$ & $1/2$ \\
            \hline % Linea spessa sotto le intestazioni
            $r_{186}=186$ & $1/10$ \\
            \hline % Linea spessa sotto le intestazioni
            $r_{255}=255$ & $1/10$ \\
            \hline % Linea spessa sotto le intestazioni
            $r_{k} \, se \, k \neq 87, 128, 186, 255$ & ${0}$ \\
            \hline
    \end{tabular}
\end{table}
per esempio si usa per ogni livello di grigio il numero massimo rappresentabile del livello di grigio più alto ($255$), che è $8$ ($\log_{2}{(255)}$): Il
livello medio di grigio è $L_{avg_{1}}=8\, b$. A questo punto conviene assegnare il numero di bit a seconda della probabilità: maggiore è la probabilità,
meno bit deve avere quel livello di grigio, così lo spazio occupato sarebbe decisamente minore.
\begin{table}[H] % Usa [H] per renderla non flottante
    \centering
    \begin{tabular}{| C{5cm} | C{5cm} | C{5cm}|} 
            \hline
            
            % Usiamo \thead per le intestazioni:
            \textbf{$r_k$} & \textbf{$p_k(r_k)$} & \textbf{$l_k(r_k)$} \\ 
            \hline % Linea spessa sotto le intestazioni
            $r_{87}=87$ & $3/10$ & $2$\\
            \hline % Linea spessa sotto le intestazioni
            $r_{128}=128$ & $1/2$ & $1$\\
            \hline % Linea spessa sotto le intestazioni
            $r_{186}=186$ & $1/10$ & $3$\\
            \hline % Linea spessa sotto le intestazioni
            $r_{255}=255$ & $1/10$ & $3$ \\
            \hline % Linea spessa sotto le intestazioni
    \end{tabular}
\end{table}
\begin{equation*}
    L_{avg_{2}}=\frac{3}{10}\times 2 + \frac{1}{2}\times 1 + \frac{1}{10} \times 3 + \frac{1}{10} \times 3 = \frac{17}{10} \, b = 1,7\, b
\end{equation*}
Infine, si calcolano le due grandezze.
\begin{center}
\begin{minipage}{0.48\textwidth} % Colonna Sinistra (48% della larghezza del testo)
    \centering
    \begin{equation*}
        C_R = \frac{8\times10}{17}=\frac{80}{17}
    \end{equation*}
\end{minipage}
\hfill % Spazio elastico per separare le due colonne
\begin{minipage}{0.48\textwidth} % Colonna Destra (48% della larghezza del testo)
    \centering
    \begin{equation*}
        R_D = 1 - \frac{17}{80} = \frac{63}{80} = 78,75\%
    \end{equation*}
\end{minipage}
\end{center}
\subsection{Ridondanza interpixel}
La ridondanza interpixel implica che qualsiasi valore di un pixel può essere predetto dai sui vicini, grazie alla correlazione. Per ridurre ciò,
i dati dovrebberero essere mappati.
\begin{figure}[htbp]
        \centering
        \includegraphics[width=0.4\textwidth]{cap05/interpixel} 
        \caption{Esempio di ridondanza di interpixel}
        \label{fig:interpixel}
\end{figure}
\subsection{Ridondanza psicovisiva}
La ridondanza psicovisiva sta nel fatto che alcuni pixel sono talmente simili che l'occhio umano non li percepisce, dato che il sistema visivo umano
non percepisce tutta l'informazione visiva con la stessa intensità: ma cerca solo caratteristiche importanti, come angoli e texture. Il classico esempio
è un immagine a tinta unita.
\begin{figure}[htbp]
        \centering
        \includegraphics[width=0.9\textwidth]{cap05/psico} 
        \caption{Esempio di ridondanza psicovisiva}
        \label{fig:psicovisiva}
\end{figure}
\section{Teoria dell'informazione}
Per effettuare una codifica efficiente, è necessario conoscere dei cenni di teoria dell'informazione.
\begin{figure}[htbp]
        \centering
        \includegraphics[width=0.9\textwidth]{cap05/teoria} 
        \caption{Esempio di rappresentazione della teoria dell'informazione} 
        \label{fig:teoria}
\end{figure}
\subsection{Concetti base di informazione}
Per prima cosa è necessario comprendere i concetti base dell'informazione:
\begin{itemize}
    \item più la probabilità di un evento è bassa più il contenuto informativo è alto;
    \item l'informazione di più messaggi indipendenti sono la somma delle informazione corrispottive.
\end{itemize}
Detto ciò, sia $X$ una sorgente che genera un messaggio $x_i$ con probabilità $p(x_i)$, l'informazione si calcola nel modo seguente:
\begin{equation*}
    I(x_i) = \log_{c}{\left[\frac{1}{p(x_i)}\right]}
\end{equation*}
dove $c$ è la base del logiritmo, che dipende dal tipo di codifica. Le più comuni sono il bit ($c=2$) ed il nat ($c=e$).
\subsection{Entropia}
Un altro concetto fondamentale è l'entropia dell'informazione, che misura l'incertezza media e l'informazione attesa della sorgente.
\begin{equation*}
    H(X)=-\sum_{i=1}^{N}p(x_i)\log_{c}{[p(x_i)]}
\end{equation*}
L'entropia dell'informazione è protagonista del teorema della codifica di sorgente di Shannon, che afferma che è impossibile comprimere i dati
che presentano come velocità di codifica, cioè il numero di simboli trasmessi per secondo, minore dell'entropia senza perdere del contenuto informativo.
\subsection{Criteri di fedeltà}
Infine, l'ultima parte analizzata della teoria dell'informazione è quella dedicata ai criteri di fedeltà, che quantifica quanta informazione è possibile
perdere mantenendo ancora l'informazione accettata oppure no. In particolare, si considerano due tipologie di criteri di fedeltà.\\
Il primo criterio è quello soggettivo, che si basano solamente sulla comparazione visiva tra due immagini, classificando le immagini in rankig da eccellente ad inutilizzabile.\\
Il secondo ed ultimo criterio è quello oggettivo, che si basa in base a delle metriche matematiche, che dipendono dall'errore ($\epsilon(x,y)$) e dall'errore assoluto ($\epsilon$),
calcolati in base all'immagine originale ($f(x,y)$) ed all'immagine compressa ($\hat{f}(x,y)$).
\begin{center}
\begin{minipage}{0.48\textwidth} % Colonna Sinistra (48% della larghezza del testo)
    \centering
    \begin{equation*}
        \epsilon(x,y) = \hat{f}(x,y)-f(x,y)
    \end{equation*}
\end{minipage}
\hfill % Spazio elastico per separare le due colonne
\begin{minipage}{0.48\textwidth} % Colonna Destra (48% della larghezza del testo)
    \centering
    \begin{equation*}
        \epsilon = \sum_{x=0}^{M-1}\sum_{y=0}^{N-1}\left[\hat{f}(x,y)-f(x,y)\right]
    \end{equation*}
\end{minipage}
\end{center}
A questo punto, i criteri di fedeltà sono dettati dalle metriche di errore quadratico medio ($RMSE$), il rapporto segnale-rumore quadrico medio ($SNR_{ms}$)
ed il rapporto segnale-rumore di picco ($PSNR$), dove questi ultimi due si misurano in Decibel (dB), che è una scala logaritmica.
\begin{center}
\begin{minipage}{0.52\textwidth} % Colonna Sinistra (48% della larghezza del testo)
    \centering
    \begin{equation*}
        RMSE=\sqrt{\frac{1}{MN}\sum_{x=0}^{M-1}\sum_{y=0}^{N-1}\left[\hat{f}(x,y)-f(x,y)\right]^2}
\end{equation*}
 \begin{equation*}
        SNR_{ms}=\sum_{y=0}^{N-1}\hat{f}^2(x,y)\left\{\sum_{x=0}^{M-1}\sum_{y=0}^{N-1}\left[\hat{f}(x,y)-f(x,y)\right]^2\right\}^{-1}
\end{equation*}
\begin{equation*}
        PSNR=L_{max}\left\{\sum_{x=0}^{M-1}\sum_{y=0}^{N-1}\left[\hat{f}(x,y)-f(x,y)\right]^2\right\}^{-1}
\end{equation*}
\end{minipage}
\hfill % Spazio elastico per separare le due colonne
\begin{minipage}[c]{0.44\textwidth} % Colonna Destra (48% della larghezza del testo)
    \centering
        \includegraphics[width=0.7\textwidth]{cap05/psnr} 
        \captionof{figure}{Misura del PSNR in un'immagine} 
        \label{fig:psnr}
\end{minipage}
\end{center}
\section{Modello di compressione delle immagini}
Il modello di compressione delle immagini consistono nei seguenti operazioni:
\begin{enumerate}
    \item mappatura, che consiste nel prendere l'immagine ed eseguire una mappatura dei dati, riducendo la ridondanza interpixel (operazione reversibile);
    \item quantizzazione, che esegue l'operazione di quantizzazione, riducendo la ridondanza psicovisiva (operazione irreversibile);
    \item codifica di simbolo, che esegue un sistema di codifica dell'immagine (operazione reversibile) e lo invia al canale;
    \item decodifica di simbolo, che dal canale prende l'immagine codifica e la ritrasforma prima della codifica;
    \item mappatura inversa, che la riporta come matrice di pixel, leggermente diversa a causa del quantizzatore.
\end{enumerate}
\begin{figure}[htbp]
        \centering
        \includegraphics[width=0.9\textwidth]{cap05/modello} 
        \caption{Schema del modello di compressione di un'immagine} 
        \label{fig:modello}
\end{figure}
A seconda dell'errore, la tipologia di compressione di un'immagine si suddividono in due grandi categorie:
\begin{itemize}
    \item compressione lossless, in cui l'errore è nullo, perciò non viene perso alcuna informazione, sfruttando la ridondanza di codifica ed interpixel;
    \item compressione lossy, invece, sfrutta ogni tipologia di ridondanza (anche quella psicovisiva), tollerando qualche errore o qualche
        perdita d'informazione.
\end{itemize}
\begin{center}
\begin{minipage}{0.48\textwidth} % Colonna Sinistra (48% della larghezza del testo)
    \centering
        \includegraphics[width=0.7\textwidth]{cap05/satellite}     
\end{minipage}
\hfill % Spazio elastico per separare le due colonne
\begin{minipage}{0.48\textwidth} % Colonna Destra (48% della larghezza del testo)
    \centering
        \includegraphics[width=0.7\textwidth]{cap05/telefono} 
\end{minipage}
\captionof{figure}{Esempio di tipologia di immagine da comprimere lossless (a sinistra) od anche lossy (a destra)}
\label{fig:compressione}
\end{center}
In questo paragrafo, sono elencate la maggior parte dei sistemi di codifica più utilizzati.
\subsection{Codifica di Huffman}
La codifica di Huffman è un sistema di codifica lossless, che usa gli alberi binari come struttura dati per costruire la codifica, la cui costruzione si
basa sulla probabilità di simbolo. Infatti, essa consiste in:
\begin{enumerate}
    \item ordinare i simboli per probabilità;
    \item combinare le due probabilità minori;
    \item ripetere ciò, finché rimangono due probabilità.
\end{enumerate}
Ad esempio, dato l'insieme $X={A,B,C,D}$ con probabilità $p(X)=\left\{\frac{1}{8},\frac{1}{2},\frac{1}{4},\frac{1}{8}\right\}$, si ordina 
$X={B,C,A,D}$ e si costruisce a partire da $B$ e poi $CAD$, che a sua volta si divide in $C$ e $AD$, che infine si divide in $A$ e $D$. In particolare,
Ogni volta che si va a sinistra si assegna uno $0$, mentre $1$ quando si va a destra.
\begin{figure}[htbp]
        \centering
        \includegraphics[width=0.35\textwidth]{cap05/Huffmann} 
        \caption{Esempio di una codifica di Huffman} 
        \label{fig:Huffman}
\end{figure}
In particolare, nessun simbolo possiede il prefisso di un altro simbolo. Infatti:
\begin{itemize}
    \item $A \rightarrow 110$;
    \item $B \rightarrow 0$;
    \item $C \rightarrow 10$;
    \item $D \rightarrow 111$.
\end{itemize}
\subsection{Codifica aritmetica}
La codifica aritmetica è una codifica lossless in cui non c'è una corrispondenza uno ad uno tra sorgente e codice e non è presente nessuna ipotesi che
la codifica di simbolo avvenga uno alla volta. A questo punto viene generato un intervallo $[0;1[$, in cui inizialmente viene suddiviso in base alla
probabilità di simbolo. Poi, più il messaggio diventa lungo, minore sarà la distanza tra i due estremi dell'intervallo, in base alla probabilità.\\
Ad esempio, dati i seguenti simboli e le probabilità seguenti: $X=\left\{A,B,C\right\}$ $p(X)=\left\{0,4;0,5;0,1\right\}$. Nel momento in cui si vuole ricostruire il messaggio
$BBC$, si considerano i seguenti simboli:
\begin{enumerate}
    \item simbolo $B$ (intervallo di riferimento $[0;1[$):
    \begin{itemize}
        \item $A \rightarrow [0; 0,4[$
        \item $A \rightarrow [0,4; 0,9[$ (intervallo da considerare)
        \item $C \rightarrow [0,9; 1[$
    \end{itemize}
    \item simbolo $B$ (intervallo di riferimento $[0,4; 0,9[$, $\Delta=0,9-0,4=0,5$):
    \begin{itemize}
        \item $A \rightarrow 0,4\times0,5=0,2 \rightarrow [0,4; 0,4+0,2[ \rightarrow [0,4; 0,6[$
        \item $A \rightarrow 0,5\times0,5=0,25 \rightarrow [0,6; 0,6+0,25[ \rightarrow [0,6; 0,85[$ (intervallo da considerare)
        \item $C \rightarrow 0,1\times0,5=0,05 \rightarrow [0,85; 0,85+0,05[ \rightarrow [0,85; 0,9[$
    \end{itemize}
    \item simbolo $C$ (intervallo di riferimento $[0,6; 0,85[$, $\Delta=0,85-0,6=0,25$):
    \begin{itemize}
        \item $A \rightarrow 0,4\times0,25=0,1 \rightarrow [0,6; 0,6+0,1[ \rightarrow [0,6; 0,7[$
        \item $A \rightarrow 0,5\times0,25=0,125 \rightarrow [0,7; 0,7+0,125[ \rightarrow [0,6; 0,825[$ 
        \item $C \rightarrow 0,1\times0,25=0,025 \rightarrow [0,825; 0,825+0,025[ \rightarrow [0,825; 0,85[$ (intervallo da considerare)
    \end{itemize}
    \item fine: $[0,825; 0,85[$.
\end{enumerate}
\subsection{Codifica RLC}
La codifica RLC (Run-length coding) è una codifica lossless molto semplice in cui concatena ogni simbolo con il numero di volte che si ripete
consecutivamente. Ad esempio $AAABBAC$ diventa $(A, 3)(B, 2)(A, 1), (C, 1)$. Tuttavia, tale sistema di codifica non è molto efficiente, tuttavia è
veramente semplice da implementare.
\subsection{Codifica di Lempel Ziv}
La codifica di Lempel Ziv è una codifica lossless che è un sistema di codifica che è composto da tre elementi: $<P,L,C>$. Dove:
\begin{itemize}
    \item $P$ indica quanti passi indietro bisognerebbe bloccare il testo decodificato;
    \item $L$ è la lunghezza della stringa;
    \item $C$ è il prossimo carattere della stringa.
\end{itemize}
A questo punto, l'algoritmo consiste in:
\begin{enumerate}
    \item si decompone la sequenza d'ingresso in stringhe;
    \item ogni volta che un blocco differisce dal precedente, viene inserito nel dizionario;
    \item tutte le stringhe incluse nel dizionario, vengono associate ad una posizione;
    \item nella procedura di codifica, l'algoritmo registra ogni stringa nuova nel dizionario e la sua posizione.
\end{enumerate}
Per esempio, data la stringa $xyxxyxyxxyy$, si procede come segue:
\begin{enumerate}
    \item legge $x$: non disponibile nel dizionario, quindi aggiunge $<0, 0, x>$;
    \item (buffer $[x]$) legge $y$: non disponibile nel dizionario, quindi aggiunge $<0, 0, y>$;
    \item (buffer $[xy]$) legge $x$, che si trova a $2$ posizioni indietro, si trova un match lungo $1$ (il carattere prima della $x$ è un'altra $x$), $<2, 1, x>$;
    \item (buffer $[xyxx]$) legge $y$, che si trova a $3$ posizioni indietro, si trova un match lungo $2$ ($yx$ c'è, $yxy$ non c'è), $<3, 2, y>$;
    \item (buffer $[xyxxyxy]$) legge $x$, si cerca il match più lungo, che è $xxy$ ($3$), che si trova $5$ posizioni indietro, $<5, 3, y>$;
    \item fine: $<0, 0, x>$ $<0, 0, y>$ $<2, 1, x>$ $<3, 2, y>$ $<5, 3, y>$.
\end{enumerate}
\subsection{Bit-plane coding}
Il bit-plane coding consiste nella composizione multivello dell'immagine in una serie di immagini binarie e comprimere ogni immagine binaria con
qualsiasi altra codifica lossless, ottenendo una compressione lossless.
\begin{figure}[htbp]
        \centering
        \includegraphics[width=0.35\textwidth]{cap05/bit} 
        \caption{Rappresentazione del bit-plane coding} 
        \label{fig:bit}
\end{figure}
\subsection{Standard di compressione lossy}
Molto spesso, le immagini vengono compresse in maniera lossy, altrimenti occuperebbero troppo spazio, perciò molto più difficili da trasmettere. Inoltre,
tali compressioni non presentano delle perdite significative nella maggior parte degli ambiti in cui si usano, grazie a molti standard. Lo standard più
usato e conosciuto è il JPEG, a cui gli viene dedicato l'intero capitolo successivo.
\vfill
\chapter{Formato JPEG}
Il formato JPEG (Joint Photographic Expert Group) è sicuramente il formato delle immagini più conosciuto ed usato, poiché ottiene dei risultati molto
notevoli nelle foto (sia in bianco e nero che a colori), tuttavia non per i cartoni animati e per le immagini generate al computer, nonostante implementi
un modello di compressione lossy. In particolare, ne esistono due versioni, entrambe trattate in questo capitolo.
\section{Prima versione di JPEG}
Di seguito è mostrato uno schema che riassume i passi eseguiti nel formato di compressione della prima versione di JPEG.
\begin{figure}[htbp]
        \centering
        \includegraphics[width=1\textwidth]{cap06/schema1} 
        \caption{Schema del modello di compressione JPEG} 
        \label{fig:schemaJPEG1}
\end{figure}
\subsection{Fase 1: conversione da RGB a YCbCr}
Un'immagine a colori viene rappresentata nel formato RGB, che non sono altre tre matrici dove R contiene i livelli di rosso, G di verde e B di blu. tuttavia
tale rappresentazione non risulta efficace nella compressione, dato che sono ridondanti. Per questo motivo, si convertono tali matrici nel formato YCbCr, dove Y è luminanza e CbCr
rappresentano la crominanza, che insieme non sono affatto ridondanti. Infatti, l'occhio umano percepisci la variazione di luminosità che di colore, dato
dal fatto che l'occhio umano ha molti più bastoncelli che coni. Ciò, verrà trattato nei capitoli successivi. \\
Comunque, la risoluzione delle componenti cromatiche vengono ridotte di un fattore $2$. In particolare:
\begin{itemize}
    \item $4:4:4$ non si ha nessun sottocampionamento;
    \item $4:2:2$ si ha una riduzione solamente nella direzione orizzontale;
    \item $4:2:0$ si ha una riduzione sia nella direzione orizzontale sia in quella verticale.
\end{itemize}
\subsection{Fase 2: trasformata DCT}
A questo punto, viene suddivisa l'immagine in blocchi da $8x8$, poiché è la dimensione che permette la migliore qualità rispetto alla dimensione del file.
Per ogni blocco, viene eseguita la cosiddetta DCT (Discete Cosine Transform), che lavora nell'intervallo da $-128$ a $128$, perciò il blocco deve essere
prima traslato negativamente di $128$. A questo punto, la formula della DCT è la seguente:
\begin{equation*}
    G(u,v)=\alpha(u)\alpha(v)\sum_{x=0}^{7}\sum_{y=0}^{7}g(x,y)\cos{\left[\frac{\pi}{8}\left(x+\frac{1}{2}\right)u\right]}\cos{\left[\frac{\pi}{8}\left(y+\frac{1}{2}\right)v\right]}
\end{equation*}
dove la moltiplicazione dei coseni non dipende dai valori di $g(x,y)$ e $\alpha$ è una funzione di normalizzazione.
Inoltre, dei $64$ coefficienti ottenuti:
\begin{itemize}
    \item $G(0,0)$ è il nucleo e viene classificato come coefficiente DC;
    \item gli altri $63$ coefficienti vengono classificati come coefficienti AC.
\end{itemize}
Infine, i motivi per cui si usa la DCT e non la trasformata di Fourier sono:
\begin{itemize}
    \item la DCT è reale pura, mentre la trasformata di Fourier è complessa;
    \item la DCT presenta meno coefficienti di qualsiasi segnale;
    \item il nucleo della trasformata diretta ed inversa sono gli stessi nella DCT.
\end{itemize}
\subsection{Fase 3: quantizzazione}
Nella fase di quantizzazione avviene la perdita vera e propria di informazione della compressione. Infatti, viene diviso punto punto per una determinata
matrice e approssimato alla parte intera più piccola. Siccome la quantizzazione avviene in base ad una soglia, viene ridotto il numero di bit per
campionamento. La formula è la seguente:
\begin{equation*}
    T^{*}(u,v)= \left\lfloor \frac{T(u,v)}{Z(u,v)} \right\rfloor
\end{equation*}
dove:
\begin{itemize}
    \item $T(u,v)$ è il coefficiente trasformato;
    \item $Z(u,v)$ è il coefficiente trasformato normalizzato;
    \item $T^{*}(u,v)$ è il coefficiente sogliato e quantizzato dell'approssimazione di $T(u,v)$.
\end{itemize}
\subsection{Fase 4: pattern zig-zag}
A questo punto, vengono ordinati i coefficienti usando un pattern zig-zag, in questo modo vengono ottenute sequenze consecutive di $0$ molto più lungh
rispetto che farlo per riga.
\begin{figure}[htbp]
        \centering
        \includegraphics[width=1\textwidth]{cap06/zigzag} 
        \caption{Pattern zig-zag} 
        \label{fig:zig-zag}
\end{figure}
\subsection{Fase 5: Codifica di entropia}
A questo punto:
\begin{itemize}
    \item tutti i coefficienti DC vengono codificati con la DPCM (Differential pulse-code modulation), che non è altro che la differenza tra i
    coefficienti DC ed i coefficienti DC dell'immagine precedente;
    \item tutti i coefficienti AC vengono codificati utilizzando la RLC, siccome sono presenti molti $0$ consecutivi.
\end{itemize}
Infine, tutti i coefficienti vengono codificati in una sequenza binaria, come quella di Huffmann e quella aritmetica; ed infine viene rieseguita la
IDCT per riottenere l'immagine compressa.
\section{JPEG2000}
La seconda versione di JPEG, denominata JPEG2000, è una versione molto migliorata rispetto alla precedente, sebbene purtroppo sia ad oggi poco utilizzata
nell'ambito comune, per la scarsa implementazione nei browser principali. Tuttavia, trova largo utilizzo in ambiti più tecnici, come nella medicina, dove
è necessaria una risoluzione molto più alta.\\
Questo standard presenta:
\begin{itemize}
    \item migliori performance di velocità/distorsione;
    \item possibilità di definire delle regioni di interesse (ROI, dall'inglese regions of interest), per avere maggiore qualità;
    \item funzionamento anche con immagini veramente molto grandi.
\end{itemize}
Tuttavia, il vantaggio più importante di JPEG2000 è sicuramente la scalabilità dei seguenti quattro tipi:
\begin{itemize}
    \item scalabilità di risoluzione, ossia è possibile ottenere un flusso progressivo di immagini, partendo dalla risoluzione più bassa;
    \item scalabilità di distorsione, cioè è possibile ottenere un flusso progressivo di SNR basso;
    \item scalabilità spazioale, ovvero è possibile ottenere un flusso progressivo di una specifica regione, partendo da una piccola ridotta;
    \item scalabilità dei componenti, dove possibile ottenere un flusso progressivo, partendo da una scala di grigi per arrivare ad un'immagine a colori.
\end{itemize}
\begin{figure}[htbp]
        \centering
        \includegraphics[width=0.9\textwidth]{cap06/roi} 
        \caption{Esempio di applicazione ROI dall'immagine originale (a sinistra) all'immagine compressa (a destra)} 
        \label{fig:roi}
\end{figure}
Di seguito, è elencato uno schema riassuntivo del processo di compressione di JPEG2000.
\begin{figure}[htbp]
        \centering
        \includegraphics[width=1\textwidth]{cap06/schema2} 
        \caption{Schema del modello di compressione JPEG2000} 
        \label{fig:schemaJPEG2000}
\end{figure}
\subsection{Fase 1: preprocessing}
Per prima cosa, si esegue il cosiddetto preprocessing, che si compone in tre sottofasi, di cui alcune sono molto simili alla prima versione di JPEG.\\
Innanzittutto, si effettua l'estrazione delle componenti e la trasformazione, in cui si mappano le componenti RGB e si trasformano in YCbCr, per ridurre
la correlazione tra le componenti, conducendo a due tipologie di trasformazioni:
\begin{itemize}
    \item trasformazione delle componenti irreversibile (ICT, irreversibile component transfomation);
    \item trasformazione delle componenti reversibile (RCT, reversibile component transfomation).
\end{itemize}
Successivamente, si esegue la cosiddetta piastrellatura dell'immagine (in inglese image tiling), in cui avviene una partizione dell'immagine in blocchi
rettangolari non sovrapposti, in modo tale da poter essere compressi in maniera indipendente.\\
Infine, avviene la traslazione dei coefficienti di $-128$, passanda dall'intervallo $[0,255]$ a $[-128,128]$, per avere la trasformata centrata in $0$.
\subsection{Fase 2: discrete Wavelet transform}
A questo punto, ad ogni blocco viene effettuata la DWT (Discrete Wavelet Transform), poiché:
\begin{itemize}
    \item permette la decomposizione multirisoluzione;
    \item ammette la scalatura di risoluzione;
    \item ogni livello di bassa risoluzione, viene suddiviso in sottobande di dimensioni dimezzate.
\end{itemize}
In particolare, JPEG2000 supporta da $0$ a $32$ stadi: solitamente per le immagini naturali, se ne usano tra $4$ ed $8$.
\subsection{Fase 3: codifica di blocchi}
Nella fase di codifica dei blocchi, ogni sottobanda viene partizionata in codifiche di blocchi, tipicamente di dimensione 32x32 o 64x64, in cui
ogni blocco viene codificato separatamente. In questo modo, si ha la scalatura di risoluzione e spaziale.
\begin{figure}[htbp]
        \centering
        \includegraphics[width=0.3\textwidth]{cap06/code} 
        \caption{Schema della fase di codifica di blocchi} 
        \label{fig:codeblocks}
\end{figure}
\subsection{Fase 4: quantizzazione}
Nella fase di quantizzazione, i coefficienti wavelet sono quantizzati usando un quantizzatore uniforme, in cui per ogni sottobanda $b$ e la larghezza
del gradino del quantizzatore di base $\Delta_{b}$, si quantizza ogni coefficiente, sfruttando la formula seguente:
\begin{equation*}
    q=\operatorname{sign}(y) \left\lfloor \frac{|y|}{\Delta_{b}} \right\rfloor
\end{equation*}
Di seguito, è riportato un esempio con $y=-21,7$ e $\Delta_{b}=10$.
\begin{center}
\begin{minipage}{0.28\textwidth} % Colonna Sinistra (48% della larghezza del testo)
    \centering
    \begin{equation*}
    q=\operatorname{sign}(-21,7) \left\lfloor \frac{|-21,7|}{10} \right\rfloor=-2
\end{equation*}
\end{minipage}
\hfill % Spazio elastico per separare le due colonne
\begin{minipage}{0.68\textwidth} % Colonna Destra (48% della larghezza del testo)
    \centering
        \includegraphics[width=0.9\textwidth]{cap06/quantizzazione}
        \captionof{figure}{Esempio di quantizzazione JPEG2000}
        \label{fig:quantizzazioneq}
\end{minipage}
\end{center}
\subsection{Fase 5: EBCOT}
La fase di EBCOT (Embedded Block Coding with Optimized Truncation) consiste nel prendere un blocco e codificarlo con il bit-plane dal bit 
più significativo a quello meno significativo. Inoltre, se alcuni piani più significativi non contengono $1$, allora il piano viene impostato sul piano 
di bit più in alto, con almeno un $1$.\\
In questo modo, si ha una classificazione dei coefficienti, partendo da quelli più insignificanti, classificati come $0$, finché non viene codificato
il primo non-zero, divendando significativo, codificando il suo segno e tutta la sua sottosequenza di bit. Ad esempio, i dati con la riduzione di
distorsione più alta dovrebbere essere codificati per primi per media di rappresentazione di bit compressi. Per essere codificato, il coefficiente deve superare:
\begin{itemize}
    \item la propagazione di significatività, in cui se è un bit è insignificante, ma almeno uno dei suo vicini lo è, allora viene codificato, oppure allora
    stesso tempo è significante, e il suo flag è $1$, allora il segno del simbolo è codificato;
    \item la raffinatezza di magnitudine, che codifica campioni significativi che non sono stati codificati nel passaggio precedente;
    \item il passaggio di pulizia, in cui codifica tutti i bit che sono stati passati attraverso i due passaggi di codifica precedenti.
\end{itemize}
\begin{figure}[htbp]
        \centering
        \includegraphics[width=0.6\textwidth]{cap06/ebcot} 
        \caption{Fase di EBCOT}
        \label{fig:ebcot}
\end{figure}
\subsection{Fase 6: codifica aritmetica adattiva}
Ora, per ridurre la ridondanza sostanziale tra piani di bit successivi, si usa la codifica aritmetica adattiva, in cui è in grado di cambiare tra ben
$18$ modelli di probabilità adattiva. In particolare:
\begin{itemize}
    \item seleziona i modelli basati sulla bit codificati precedentemente dal piano corrette a quelli precedenti;
    \item ogni modello stima la sua distribuzione di probabilità;
    \item in questo contesto usa la probabilità condizionale.
\end{itemize}
\subsection{Fase 7: strati di qualità}
I flussi di bit risultanti per ogni blocco di codice sono organizzati in livelli di qualità per ottenere:
\begin{itemize}
    \item scalabilità della risoluzione, cioè l'eliminazione dei componenti dalle sottobande ad alta risoluzione;
    \item scalabilità della distorsione, ossia l'eliminazione degli bit meno significativi dalla quantizzazione incorporata;
\end{itemize}
I livelli di qualità consentono la gestione di questa struttura di scalabilità:
\begin{itemize}
    \item specificando come ogni blocco deve essere troncato rispetto agli altri;
    \item fornendo ad ogni livello nel flusso incorporato un miglioramento progressivo;
    \item ottimizzando la struttura dei livelli durante la compressione, per un embedding ottimale.
\end{itemize}
\begin{figure}[htbp]
        \centering
        \includegraphics[width=0.9\textwidth]{cap06/qualita} 
        \caption{Strati di qualità}
        \label{fig:stratiqualitaJPEG2000}
\end{figure}
\vfill
\chapter{Fondamenti della percezione della profondità}
In questo capitolo, viene trattato come avviene la percezione dello spazio tridimensionale, partendo solamente da un'immagine, che invece possiede
solamente due dimensioni. Per prima cosa, è necessiario comprende il funzionamento del nervo oculomotore.
Per prima cosa, avviene l'accomodamento, in cui il cristallino si comporta come una lente convergente di distanza focale variabile. Inoltre, il muscolo
ciliare si contrae e si allunga, variando forma e quindi la distanza focale.
Perciò, nel momento in cui si guardano oggetti lontani, il cristallino si appiattisce, mentre quando si osservano oggetti vicini diventa più spesso.
\begin{figure}[htbp]
        \centering
        \includegraphics[width=0.9\textwidth]{cap07/adattamento} 
        \caption{Processo di adattamento} 
        \label{fig:accomodamento}
\end{figure}
In contemporanea, avviene la cosiddetta vergenza, che non è altro che un movimento simultaneo degli occhi in direzioni opposte, che consentono la fusione.
In particolare:
\begin{itemize}
    \item gli occhi ruotano uno verso l'alto per guardare oggetti vicini, detta convergenza;
    \item gli occhi ruotano in direzioni opposte per guardare oggetti lontani, denominata divergenza.
\end{itemize}
\begin{figure}[htbp]
        \centering
        \includegraphics[width=0.6\textwidth]{cap07/vergenza} 
        \caption{Processo di vergenza} 
        \label{fig:vergenza}
\end{figure}
\section{Indicatori monoculari di profondità statici}
In questo paragrafo, vengono trattati i cosiddetti indicatori monoculari di profondità statici, cioè quelle caratteristica che forniscono delle
informazioni tridimensionali in immagini. Come schema, conviene analizzare il dipinto \textit{A Rainy Day in Paris}.
\begin{figure}[htbp]
        \centering
        \includegraphics[width=0.9\textwidth]{cap07/quadro} 
        \caption{\textit{A Rainy Day in Paris} che mostra le occlusioni (1), le dimensioni relative (2), le tessiture (3), la prospettiva lineare (4), la prospettiva aerea (5) e le ombre (6)} 
        \label{fig:quadro}
\end{figure}
\subsection{Occlusioni}
Il primo indicatore sono sicuramente le occlusioni, che si definiscono come gli oggetti più vicini che bloccano l'accesso visivo a quello più distanti.
Per esempio, nella figura seguente viene mostrata una ragazza che viene parzialmente coperta da un tronco di una palma.
\begin{figure}[htbp]
        \centering
        \includegraphics[width=0.8\textwidth]{cap07/occlusioni} 
        \caption{Esempio di occlusione} 
        \label{fig:occlusione}
\end{figure}
\subsection{Dimensioni relative}
Le dimensioni relative forniscono informazioni 3D, grazie ad oggetti della stessa dimensione fisica proiettano immagini retiniche di dimensione diversa
a seconda del punto del punto di osservazione. Per esempio, è molto più facile conoscere l'altezza di una statua dell'isola di Pasqua con delle persone
(che hanno un'altezza media di 1,80 m), rispetto a che non ci sono persone od altri oggetti di riferimento: infatti, anche l'esperienza gioca 
un ruolo fondamentale in questo contesto.
\begin{figure}[htbp]
        \centering
        \includegraphics[width=0.9\textwidth]{cap07/relative} 
        \caption{Esempio di dimensioni relative} 
        \label{fig:relative}
\end{figure}
\subsection{Tessitura}
Per tessitura consiste in una superficie molto estesa con una tessitura non uniforme proietterà sulla retina un'immagine, in cui la frequenza della tessitura
aumenta all'aumentare della distanza dal punto di osservazione.
\begin{figure}[htbp]
        \centering
        \includegraphics[width=0.9\textwidth]{cap07/tessitura} 
        \caption{Esempio di tessitura} 
        \label{fig:tessitura}
\end{figure}
\subsection{Prospettiva lineare}
La prospettiva lineare consiste in linee parallele sul piano visivo convergono verso il punti di fuga, all'aumentare della distanza di osservazione: ciò
fornisce informazioni sulle distanze nello spazio tridimensionale. Un classico esempio sono i binari del treno: l'informazione aggiunta ce l'hai grazie
ai bastoncini tra le due rotaie (due linee parallele).
\begin{figure}[htbp]
        \centering
        \includegraphics[width=0.9\textwidth]{cap07/lineare} 
        \caption{Esempio di prospettiva lineare} 
        \label{fig:lineare}
\end{figure}
\subsection{Prospettiva aerea}
La prospettiva aerea descrive il fatto che la luce degli oggetti più lontani, prima di raggiungere la nostra retina, deve attraversare più parti dell'
atmosfera rispetto a quella degli oggetti più vicini.
\begin{figure}[htbp]
        \centering
        \includegraphics[width=0.8\textwidth]{cap07/aerea} 
        \caption{Esempio di prospettiva aerea} 
        \label{fig:aerea}
\end{figure}
\subsection{Ombre}
L'occlusione delle fonte di luce crea delle ombre, permettono di avere informazioni sulla struttura tridimensionale della scena.
\begin{figure}[htbp]
        \centering
        \includegraphics[width=0.8\textwidth]{cap07/ombre} 
        \caption{Esempio di ombre} 
        \label{fig:ombre}
\end{figure}
\section{Visione binoculare}
Per avere la visione binoculare, è necessario che si verificano tre fasi fondamentali, i cosiddetti gradi di Worth:
\begin{itemize}
        \item percezione simultanea;
        \item fusione;
        \item stereopsi.
\end{itemize}
Tuttavia, il concetto fondamentale da comprendere sta nel fatto che, ogni fenomeno è di livello superiore al precedente e la presenza del grado più
elevato. Infine, la stereopsi prevede la presenza dei due precedenti. In questo paragrafo, verranno trattati nel dettaglio.
\subsection{Percezione simultanea}
La percezione simultanea si definisce come la capacità di percepire contemporaneamente le immagini sui due occhi. Infatti, entrambi gli occhi cooperano
per formare un'unica percezione. I campi visivi dei due occhi sono legati reciprocamente. Difatti, un area della retina dell'occhio destro corrisponde
ad un'altra dell'occhio sinistro e viceversa. I principali punti corrispondenti prendono il nome di fovee.
\subsection{Fusione}
Per quanto riguarda la fusione, si fa riferimento a due tipologie diverse.
La prima è la fusione motoria, che contribuisce semplicemente a mantenere la fusione di due immagini sfruttando i muscoli oculari.\\
La seconda tipologia, ed anche la più significativa, è la funsione sensioriale, che rappresenta la capacità intrinseca del cervello di elaborare e integrare le informazioni visive grezze provenienti dai due occhi, unendole in un'unica percezione visiva (immagine singola).\\
A questo punto, si fa riferimento alla legge della corrispondenza sensoriale, che afferma che:
\begin{itemize}
        \item punti retinici corrispondenti sono dei punti che hanno in comune una direzione visiva soggettiva, che vengono fuse se vengono visti nella
                stessa direzione visiva e se sono sufficientemente simili;
        \item le fovee sono punti retinici corrispondenti, cioè hanno lo valore spaziale nullo, che corrisponde alla direzione visiva principale.
\end{itemize}
Inoltre, un soggetto ha corrispondenza retinica normale quando la stimolazione di entrambi le fovee o di punti retinici corrispondenti dà origine 
ad un percetto unitario. Il processo di fusione, invece, risulta faticoso se gli oggetti sono troppo vicini oppure se la loro profondità 
cambia rapidamente.
\begin{figure}[H]
        \centering
        \includegraphics[width=0.8\textwidth]{cap07/crn} 
        \caption{Se $\alpha_{BL}$ e $\alpha_{BR}$ sono uguali, allora $BR$ e $BL$ sono punti retinici corrispondenti} 
        \label{fig:crn}
\end{figure}
 Infine, si fa riferimento a:
 \begin{itemize}
        \item oroptero, che è il luogo di tutti i punti nello spazio in cui le immagini cadono su punti retinici corrispondenti per un angolo di convergenza 
        $\beta$ ed i punti all'esterno della cosiddetta area di Panum, generano immagine doppie;
        \item diplopia fisiologica, che è lo sdoppiamento fisiologico dei punti che giacciono fuori dall'oroptero, perciò $BR$ e $BL$ sono punti retici disparati,
                perciò hanno due direzioni visive soggettive diverse. 
 \end{itemize}
 \begin{center}
\begin{minipage}{0.48\textwidth} % Colonna Sinistra (48% della larghezza del testo)
    \centering
        \includegraphics[width=0.7\textwidth]{cap07/oroptero}     
\end{minipage}
\hfill % Spazio elastico per separare le due colonne
\begin{minipage}{0.48\textwidth} % Colonna Destra (48% della larghezza del testo)
    \centering
        \includegraphics[width=0.7\textwidth]{cap07/diplopia} 
\end{minipage}
\captionof{figure}{Descrizione dell'oroptero (a sinistra) e della diplopia fisiologica (a destra)}
\label{fig:dcdkjcddjdcjd}
\end{center}
\subsection{Stereopsi}
La stereopsi è la capacità del cervello di fondere due immagini provienti dagli occhi per costruire una visione tridimensionale di un oggetto osservato.
Inoltre, esiste un occhio dominante, che è quello che prefisce un input visivo rispetto all'altro. Per verificare qual è il proprio occhio dominate,
si esegue il test di Miles.
\section{Applicazione ad immagini statiche}
A questo punto, non resta che capire la perceziona della profondità a partire da un'immagine. Il cervello umano percepisce la profondità, grazie a due
viste provenienti da due occhi, che sono leggermente traslati, la cui distanza di traslazione prende il nome di disparità. A questo punto, la disparità
dipende dalla disparità e a indicatori monoculari. Per questo motivo, sono nati gli stereogrammi, di cui ne sono mostrati i seguenti tre:
\begin{itemize}
        \item stereoscopio di Wheatstone, il cui funzionamento si basa su due specchi posizionati ad angolo, in cui ogni occhio vede una singola immagine
        ma il cervello le vede come se venissero da un unico punto e le fonde, creando una percezione di tridimensionalità;
        \item coppie stereo d'autore, che non sono altro che due dipinti quasi identici, che affiancati il cervello li affianca per creare un'immagine
        singola, creando un'illusione di profondità;
        \item anaglifi, le cui due immagini corrispondenti alla vista sono stampate sullo stesso supporto, ma una ha un supporto in blu e l'altra in rosso,
                creando profondità grazie alla scarsa informazione cromatica.
\end{itemize}
\begin{center}
\begin{minipage}{0.30\textwidth} % Colonna Sinistra (48% della larghezza del testo)
    \centering
        \includegraphics[width=0.7\textwidth]{cap07/stereo}     
\end{minipage}
\hfill % Spazio elastico per separare le due colonne
\begin{minipage}{0.30\textwidth} % Colonna Destra (48% della larghezza del testo)
    \centering
        \includegraphics[width=0.7\textwidth]{cap07/autore} 
\end{minipage}
\hfill % Spazio elastico per separare le due colonne
\begin{minipage}{0.30\textwidth} % Colonna Destra (48% della larghezza del testo)
    \centering
        \includegraphics[width=0.7\textwidth]{cap07/anaglifo} 
\end{minipage}
\captionof{figure}{Dalla prima immagine a sinistra: stereoscopio di Wheatstone, coppie stereo d'autore e anaglifo}
\label{fig:dcdkjcddjdcjds}
\end{center}
\vfill
\part{Video}
\chapter{Creazione e caratteristiche dei video}
La creazione di un video avviene mediante una sequenza di immagini 2D, che rappresentano la proiezione di una scena 3D in movimento sul piano immagine
della videocamera.
\begin{figure}[H]
        \centering
        \includegraphics[width=0.9\textwidth]{cap08/primo} 
        \caption{Primo video mai realizzato} 
        \label{fig:primo}
\end{figure}
In questo capitolo, verranno analizzate le caretteristiche del video.
\section{Proporzioni}
La prima caratteristica fondamentale del video è la proporzione, che non è altro che il rapporto tra l'larghezza e l'altezza. Tale rapporto è fisso:
infatti risulta tre valori standard:
\begin{itemize}
    \item 4:3, usato nelle vecchie TV, per i canali SD, poiché risulta troppo squadrato;
    \item 16:9, usato nella maggior parte delle TV, per i canali DC;
    \item 21:9, usato nella maggior parte dei video, ma invece in pochissime TV.
\end{itemize}
Nel caso in cui il video e la TV hanno proporzioni diversi, verranno mostrati degli spazi vuoti, come mostrato nell'immagine di seguito.
\begin{figure}[H]
        \centering
        \includegraphics[width=0.9\textwidth]{cap08/rapporti}
        \caption{Video mostrati in varie TV} 
        \label{fig:rapporti}
\end{figure}
\section{Frame rate}
Per dare la sensazione di movimento, le immagini devono essere ricaricate, in generale a frequenza di $60 Hz$. Il frame rate non è altro che il tempo
impiegato tra due immagini successive, espresso in frame per seconds (fps). Solitamente, il frame rate per essere considerato accettabile deve essere
almeno 30 fps, mentre è buono a 60 fps, eccellente sopra i 120 fps.
\begin{figure}[htbp]
        \centering
        \includegraphics[width=0.8\textwidth]{cap08/fps} 
        \caption{Come varia il frame rate in un video} 
        \label{fig:fps}
\end{figure}
\section{Modalità di trasmissione}
Una caratteristica molto importante è come avviene la trasmissione di un video, a seconda della qualità richiesta.\\
Il metodo di trasmissione più semplice, ma che richiede molto più risoluzione e larghezza di banda è il metodo progressivo, in cui ogni immagine viene
trasmessa esattamente come viene mostrata.\\
Se invece, non si possiede molta larghezza di banda oppure il framerate non è molto alto, allora si opta per il metodo interlacciato, che consiste nel
trasmettere prima le linee pari e poi quelle dispari, creando l'illusione del frame rate doppio rispetto a quello reale. Tuttavia, è necessario fornire
le seguenti considerazioni:
\begin{itemize}
        \item è necessario un tempo di ritracciamento, che crea degli impulsi neri;
        \item occorrono dei segnali di sincronizzazione, per garantire che l'immagine inizi in alto a sinistra;
        \item per evitare lo sfarfallio, si deve effettuare un aggiornamento completo a $50 \, \text{Hz}$.
\end{itemize}
\section{Risoluzione}
Analogamente per le immagini, anche per quanto riguarda i video la risoluzione non è altro che il numero di pixel presenti. Tuttavia, qui è cruciale
anche le proporzioni, il frame rate e la modalità di trasmissione. In particolare:
\begin{itemize}
        \item formato PAL (Phase Alternate Line), sviluppato principalmente in Europa (Francia esclusa ad esempio), in 4:3 a $25 \, \text{fps}$ ad una risoluzione 
              massima a 720x576px;
        \item HD ready, con risoluzione a 720p (progressivo) a $50 \, \text{fps}$ (720p/50);
        \item HDTV (1080i/25);
        \item Full HDTV (1080p/50);
        \item Ultra HD, composto dal 2K, dal 4K e dall'8K.
\end{itemize}
\begin{figure}[H]
        \centering
        \includegraphics[width=0.7\textwidth]{cap08/risoluzione} 
        \caption{La differenza tra le varie risoluzioni} 
        \label{fig:risoluzionevideo}
\end{figure}
\section{Rappresentazione dei colori}
Come accennato all'inizio del corso, i coni presenti nell'occhio umano rispondono a tre lunghezze d'onda fondamentali:
\begin{itemize}
        \item rosso ($700 \, \text{nm}$);
        \item verde ($546,1 \, \text{nm}$);
        \item blu ($435,8 \, \text{nm}$).
\end{itemize}
Tale lunghezze d'onda fanno parte dello spettro visibile.
\begin{figure}[htbp]
        \centering
        \includegraphics[width=0.9\textwidth]{cap08/spettro} 
        \caption{Spettro della luce visibile} 
        \label{fig:spettrocolori}
\end{figure}
\subsection{Teorie della tipologia di colore}
Per formare ogni colore, si considerano due teorie, a seconda del comportamento dell'assorbimento della luce.\\
La prima è la teoria sottrattiva, in cui il nero assorbe più luce ed il bianco riflette più luce. A questo punto, tutti i colori, esclusi quelli del
pigmento, vengono assorbiti. In questo caso, i colori primari sono il ciano, il magenta, il giallo ed il nero. Tale teoria trova largo utilizzo nelle 
stampanti.
Tuttavia, nei video si usa l'altra teoria, ossia la teoria additiva, in cui il nero riflette più luce, mentre il bianco asssorbe più luce. A questo
punto, i colori primari sono gli stessi della luce visibile, perciò il rosso, il verde ed il blu.
\begin{center}
\begin{minipage}{0.48\textwidth} % Colonna Sinistra (48% della larghezza del testo)
    \centering
        \includegraphics[width=0.7\textwidth]{cap08/pigmento}     
\end{minipage}
\hfill % Spazio elastico per separare le due colonne
\begin{minipage}{0.48\textwidth} % Colonna Destra (48% della larghezza del testo)
    \centering
        \includegraphics[width=0.7\textwidth]{cap08/luce} 
\end{minipage}
\end{center}
\captionof{figure}{Teoria additiva (a sinistra) e sottrattiva (a destra)}
\label{fig:dcdksjcddjdcjds}
\subsection{Rappresentazione dei colori}
Per quanto riguarda la rappresentazione dei colori, ciò può avvenire semplicimente con la ruota dei colori, detto HSV, dove:
\begin{itemize}
        \item H sta per hue, che indica la rotazione sulla ruota dei colori;
        \item S sta per saturation, che indica la saturazione;
        \item V sta per value, che indica il valore di luminosità.
\end{itemize}
Un'altra rappresentazione è RGB, in cui vengono rappresentati i tre colori su degli assi cartesiani in tre dimensioni. In particolare, sull'asse $y$
si rappresenta il verde, poiché presenta una maggiore risposta alla luce; il rosso ed il blu rispettivamente sull'asse $x$ e $z$, come mostrato nella
figura seguente.
\begin{center}
\begin{minipage}{0.56\textwidth} % Colonna Sinistra (48% della larghezza del testo)
    \centering
        \includegraphics[width=0.95\textwidth]{cap08/rgb}     
\end{minipage}
\hfill % Spazio elastico per separare le due colonne
\begin{minipage}{0.40\textwidth} % Colonna Destra (48% della larghezza del testo)
     \begin{table}[H] % Usa [H] per renderla non flottante
    \centering
    \begin{tabular}{| C{2cm} | C{2cm} |} 
            \hline
            % Usiamo \thead per le intestazioni:
            \textbf{Punto} & \textbf{Colore} \\ 
            \hline % Linea spessa sotto le intestazioni
            $(0,0,0)$ & Nero \\
            \hline % Linea spessa sotto le intestazioni
            $(0,0,1)$ & Blu \\
            \hline % Linea spessa sotto le intestazioni
            $(0,1,0)$ & Verde \\
            \hline % Linea spessa sotto le intestazioni
            $(0,1,1)$ & Ciano \\
            \hline % Linea spessa sotto le intestazioni
            $(1,0,0)$ & Rosso \\
            \hline
            $(1,0,1)$ & Viola \\
            \hline
            $(1,1,0)$ & Giallo \\
            \hline
            $(1,1,1)$ & Bianco \\
            \hline
    \end{tabular}
\end{table}
\end{minipage}
\end{center}
\captionof{figure}{Rappresentazione cartesiana di RGB}
\label{fig:dcdkssjcddjdcjds}
Come già visto nella compressione JPEG, se si separano le tre componenti RGB, esse sono molto correlate, poiché rappresentano più o meno lo stesso
contenuto informativo, cambiando solo leggermente i livelli di grigio. Per questo motivo, si sceglie la rappresentazione YCbCr, poiché è più compatibile
con il sistema visivo umano.
\begin{center}
\begin{minipage}{0.30\textwidth} % Colonna Sinistra (48% della larghezza del testo)
    \centering
        \begin{equation*}
        Y=K_rR+K_bB+(1-K_r-K_b)G
\end{equation*}    
\end{minipage}
\hfill % Spazio elastico per separare le due colonne
\begin{minipage}{0.30\textwidth} % Colonna Destra (48% della larghezza del testo)
    \centering
\begin{equation*}
        C_b=\frac{0,5(B-y)}{1-K_b}
\end{equation*}
\end{minipage}
\hfill % Spazio elastico per separare le due colonne
\begin{minipage}{0.30\textwidth} % Colonna Destra (48% della larghezza del testo)
    \centering
\begin{equation*}
        C_r=\frac{0,5(R-y)}{1-K_r}
\end{equation*}
\end{minipage}
\end{center}
\begin{figure}[htbp]
        \centering
        \includegraphics[width=0.5\textwidth]{cap08/analisi_colore_risultato} 
        \caption{Divisione immagine in RGB e YCbCr} 
        \label{fig:analis_colore_risultato}
\end{figure}
\subsection{Acquisizione dei colori}
Nella fase di acquisizione, la luce viene catturata tramite il CCD (Charge Coupled Device), che non è altro che un array di celle che contengono colori.
Nella maggior parte delle camere possiedono un chip singolo: tuttavia, la soluzione migliore sarebbe possederne tre diversi. Per l'occhio umano, la
distribuzione migliore è $50 \%$ di verde, $25 \%$ di rosso e $25 \%$ di blu. In particolare:
\begin{itemize}
        \item i sensori verdi sono elementi sensibili alla luminanza;
        \item i sensori rossi e blu sono elementi sensibili alla crominanza.
\end{itemize}
\begin{figure}[htbp]
        \centering
        \includegraphics[width=0.9\textwidth]{cap08/camera} 
        \caption{Modalità di acquisizione di immagini} 
        \label{fig:camere}
\end{figure}
\subsection{Quantizzazione}
Per quanto riguarda la quantizzazione, devono essere definiti il numero di livelli, tipicamente sono $8$ ($8 b/p$). Questo numero viene definito dalla
legge di Weber, di meno (già con $6$) si comincia ad avere sfocatura. Con $8 b/p$ si ha l'LDR (Low Dynamic Range), che è il rapporto tra il valore
massimo ed il valore minimo di una misura fisica. Per esempio:
\begin{itemize}
        \item per una scena tra la parte più chiara e quella più scura;
        \item per un display tra il massimo e minimo dell'intensità emessa dallo schermo.
\end{itemize}
Infine, con $32 b/p$ si ha l'HDR (High Dynamic Range), che posso acquisire immagini con esposizioni molto diverse, per avere scene più dinamiche.
\vfill
\chapter{Video stereoscopico e tecnologie 3D}
In questo capitolo, viene ripreso il concetto di stereoscopia, applicato però ai video: perciò è necessario avere delle competenze dello stereoscopia
applicate alle immagini.
\section{Indicatori monoculari di profondità dinamici}
Gli indicatori monoculari di profondità dinamici sono segnali visivi che il nostro sistema percettivo utilizza con un solo occhio per percepire la
profondità e la distanza degli oggetti; richiedendo un movimento od un cambiamento di scena.
\subsection{Parallasse di movimento}
Il parallasse di movimento afferma che oggetti che si trovano:
\begin{itemize}
    \item più in lontanza sembrano che si muovono nella stessa direzione dell'osservatore;
    \item più vicini sembrano che si muovono nella direzione opposta dell'osservatore.
\end{itemize}
Per esempio, mentre si osserva dalla finestrino di un treno un paesaggio di montagna:
\begin{itemize}
    \item se si guarda un'albero a pochi metri di distanza, sembra muoversi nella direzione opposta;
    \item se si guardano le montagne sullo sfondo, sembrano muoversi nella stessa direzione.
\end{itemize}
\begin{figure}[htbp]
        \centering
        \includegraphics[width=0.6\textwidth]{cap09/montagna} 
        \caption{Esempio pratico del funzionamento del parallasse di movimento} 
        \label{fig:montagne}
\end{figure}
\subsection{Velocità angolare relativa}
La velocità angolare relativa descrive quanto velocemente un corpo ruota, misurata da un osservatore che sta ruotando anche lui. In particolare:
\begin{itemize}
    \item l'osservatore fisso misura la rotazione totale del corpo;
    \item l'osservatore girevole vede solamente la rotazione del corpo in eccesso od in difetto rispetto alla sua rotazione.
\end{itemize}
Un classico esempio è un genitore che vede il proprio figlio girare sull'altalena girevole: il genitore, che è l'osservatore fisso, vede la velocità angolare assoluta del figlio,
mentre il figlio, che è l'osservatore girevole, vede gli oggetti immobile che girano attorno a lui, mentre la giostra immobile.
\begin{figure}[htbp]
        \centering
        \includegraphics[width=0.5\textwidth]{cap09/giostra} 
        \caption{Esempio pratico del funzionamento della velocità angolare relativa} 
        \label{fig:giostra}
\end{figure}
\subsection{Espansione radiale}
L'espansione radiale è l'effetto che si verifica quando un osservatore si muove in avanti oppure al contrario quando un oggetto si muove verso 
l'osservatore, le immagini degli oggetti sulla retina si allargano o si espandono in modo centrifugo a partire dal punto centrale, eccetto per un punto
centrale, detto centro di espansione, in cui non si muove nulla dalla retina.\\
Un esempio pratico è quando si guida in autostrada di notte: le strisce bianche sembrano nascere dal centro di espansione, mentre gli altri oggetti sembrano allontanarsi dal centro di espansione.
\begin{figure}[htbp]
        \centering
        \includegraphics[width=0.6\textwidth]{cap09/autostrada} 
        \caption{Esempio pratico del funzionamento dell'espansione radiale} 
        \label{fig:autostrada}
\end{figure}
\subsection{Movimento delle ombre}
Per quanto riguarda il movimento delle ombre, è una combinazione di semplici principi fisici e fenomeni astronomici. Un'ombra si muove perché la 
sorgente luminosa, l'oggetto che la proietta oppure la superficie che la riceve sono in movimento. Tale movimento può aiutare il cervello a 
processare la direzione del movimento dell'oggetto o della sorgente luminosa.
\vfill
\chapter{Compressione dei video e standard MPEG}
I video non compressi contengono una quantità di dati veramente immensa. Ad esempio, per elaborare un segnale video HDTV-R a 720p a $60 \, \text{fps}$:
\begin{equation*}
    c = \left(720 \times 1280 \, \frac{\text{px}}{\text{frame}}\right) \times \left(60 \, \frac{\text{frame}}{\text{s}}\right) \times
    \left(3 \,  \frac{\text{color}}{\text{px}}\right) \times \left(8 \,  \frac{\text{b}}{\text{color}}\right) =
    1,3264 \, \text{Gb/s}
\end{equation*}
dove la banda del canale è solamente $20 \, \text{Mb/s}$: da qui si rende necessario comprimere i video.
\section{Codifica dei video}
La codifica video consiste nel ridurre la ridondanza spaziale con i frame, similmente a come avviene in JPEG, e temporale.
\subsection{Differenza tra frame}
Un video non è altro che è una sequenza di frame, in cui gli oggetti appiaiono, si muovono e scompaiono e i pixel di sfondo rimangono gli stessi. A
questo punto, se si sottraggono due immagini:
\begin{itemize}
    \item il cambiamento di sfondo è solamente rumore;
    \item i bordi degli oggetti presentano dei cambiamenti significativi.
\end{itemize}
Questa operazione prende il nome di DPCM (Difference Pulse-Code Modulation), in cui il frame 0 è il fermo immagine ed il resto sono la differenza
tra il frame corrente e quello precedente: per esempio la differenza frame 1 è la differenza tra il frame 1 ed il frame 0, la differenza frame 2 è la 
differenza tra il frame 2 ed il frame 1 e così via. Perciò:
\begin{itemize}
    \item se la scena è priva di movimento, allora tutte le differenze di frame sono nulle, potendo comprimere molto contenuto informativo;
    \item se nella scena è presente movimento, si può vedere la differenza tra le due immagini.
\end{itemize}
La differenza tra due frame può essere causata:
\begin{itemize}
    \item il movimento della camera di contorni sullo sfondo o di oggetti fissi si possono vedere dalla immagine differenziale;
    \item il movimento degli oggetti si può vedere grazie ai contorni degli oggetti in movimento;
    \item i cambi di illuminazione, come fari e lampioni;
    \item tagli di sceni, l'immagine differenziale presenti dei cambiamenti netti;
    \item rumore.
\end{itemize}
A questo punto, se la differenza tra due frame è solamente rumore, allora si preferisce avere differenza nulla. Invece, se si riesce a vedere qualcosa
nell'immagine differenziale ed anche a riconoscerla, significa che è presente correlazione nell'immagine differenziale. L'obiettivo è rimuovere la
correlazione, compensando il movimento.
\begin{center}
\begin{minipage}{0.30\textwidth} % Colonna Sinistra (48% della larghezza del testo)
    \centering
        \includegraphics[width=0.9\textwidth]{cap10/f0}     
\end{minipage}
\hfill % Spazio elastico per separare le due colonne
\begin{minipage}{0.30\textwidth} % Colonna Destra (48% della larghezza del testo)
    \centering
        \includegraphics[width=0.9\textwidth]{cap10/f1} 
\end{minipage}
\hfill % Spazio elastico per separare le due colonne
\begin{minipage}{0.30\textwidth} % Colonna Destra (48% della larghezza del testo)
    \centering
        \includegraphics[width=0.9\textwidth]{cap10/fd} 
\end{minipage}
\captionof{figure}{Differenza tra $f_0$ (a sinistra) a $f_1$ (al centro) mostrato a destra}
\label{fig:euhfeihfeihfeifhieeihei}
\end{center}
Nella codifica video, si definiscono due tipologie di frame:
\begin{itemize}
    \item key frame, se la compressione è basata sul contenuto del frame;
    \item delta frame, se la compressione è basata sul contenuto dell'ultimo key frame.
\end{itemize}
\begin{figure}[H]
        \centering
        \includegraphics[width=0.9\textwidth]{cap10/frame} 
        \caption{Sequenza di key frame e delta frame} 
        \label{fig:seq_key_delta}
\end{figure}
Infine, la quantità di dati da codificare può essere ridotto significativamente se il frame precedente è sottratto dal frame corrente: 
in questo modo si ottiene la cosiddetta immagine residua.
\subsection{Motion JPEG e sfruttamento della ridondanza temporale}
Nella fase di Motion JPEG, avviene la compressione JPEG ad ogni fotogramma del video, effettuando una compressione puramente spaziale con velocità di
compressione da 2:1 a 12:1 con perdita di dati fino a 5:1, detta qualità broadcast. A ciò si aggiunge lo sfruttamento della ridondanza temporale, infatti
esistono tre tipologie di movimento:
\begin{itemize}
    \item traslazionella fase, il movimento tipico degli oggetti rigidi;
    \item rotazione attraverso un asse;
    \item zoom.
\end{itemize}
Ciò, permette di comprendere i parametri che descrivono il movimento. In particolare, si prendono alcune porzioni di frame e si stima il moto tra due
frame: il frame corrente ed il frame di riferimento. Il problema è capire bene quali sono le porzioni.\\
Infine, si esegue la codifica ed avviene attraverso i vettori di movimento $dx$ e $dy$. L'idea generale sta nel trovare una regione $P_C$ nel frame
corrente, trovare una regione $P_R$ nella finestra di ricerca nel frame di riferimento, per minimizzare l'errore tra la regione $P_R$ e $P_C$.
\begin{figure}[H]
        \centering
        \includegraphics[width=0.9\textwidth]{cap10/regioni} 
        \caption{Stima del movimento} 
        \label{fig:regioni}
\end{figure}
\subsection{Block matching}
A questo punto, gli algoritmi di codifica dei video solitamente contengono due tipologie di schemi di codifica:
\begin{itemize}
    \item la codifca intraframe, che è simile alla codifica di immagini fisse, poiché non sfrutta la correlazione di frame adiacenti;
    \item la codifica interframe, che invece include la stima e la componensazione del movimento, per ridurre la ridondanza temporale.
\end{itemize}
La tecnica del block matching consiste nel dividere il frame corrente in blocchi, per trovare un blocco candidato nella regione di ricerca, che ha la
correlazione maggiore con il blocco sorgente. In particolare, la differenza tra il blocco sorgente ed il blocco candidato prende il nome di vettori di
movimento. Infatti, quest'ultimo descrive la distanza tra la posizione del blocco in fase di codifica e la posizione del blocco con la corrispondenza 
migliore nel frame di riferimento. Infine, l'immagine residua dell'errore contiene le differenze tra l'immagine desiderata e quella prevista. Ciò, si
calcola con l'MSE o l'MAE.
\begin{center}
\begin{minipage}{0.48\textwidth} % Colonna Sinistra (48% della larghezza del testo)
    \centering
        \begin{equation*}
            MSE = \sum\left[ b\left(B_{ref}\right) - b(\left(B_{curr}\right) \right]^2
        \end{equation*}     
\end{minipage}
\hfill % Spazio elastico per separare le due colonne
\begin{minipage}{0.48\textwidth} % Colonna Destra (48% della larghezza del testo)
    \begin{equation*}
            MAE = \sum\left| b\left(B_{ref}\right) - b(\left(B_{curr}\right)\right|
    \end{equation*}
\end{minipage}
\end{center}
\section{Processo di compressione video}
In questo paragrafo si entra nel cuore del capitolo: vengono spiegate le varie tecniche della compressione video.
\subsection{Codifica basata sul block matching}

\vfill
\part{Audio}
\chapter{Segnali audio}
In questo capitolo, vengono spiegato il suono, introducendo però prima come funziona l'udito umano.
\section{Orecchio umano}
In questo paragrafo, viene spiegato prima la struttura dell'orecchio umano, per poi della trasmissione del suono.
\subsection{Struttura dell'orecchio umano}
L'orecchio umano è divisa in tre componenti principali:
\begin{itemize}
    \item orecchio esterno, che include i padiglioni auricolari ed il condotto uditivo;
    \item orecchio medio, che contiene il timpano e gli ossicini;
    \item orecchio interno, che contiene la finestra ovale e le vie nervose.
\end{itemize}
\begin{figure}[H]
        \centering
        \includegraphics[width=0.9\textwidth]{cap11/orecchio} 
        \caption{Struttura dell'orecchio umano} 
        \label{fig:orecchio}
\end{figure}
\subsection{Trasmissione del suono}
La trasmissione del suono avviene nel modo seguente:
\begin{enumerate}
    \item le onde sonore colpiscono il timpano facendolo vibrare;
    \item gli ossicini vibrano all'unisono;
    \item la staffa si muove dentro e fuori dalla finestra ovale;
    \item le onde sonore vengono trasmesse attraverso la perilinfa;
    \item le onde ad alta frequenza causano basse vibrazioni alla base dell'orecchio interno, mentre quelle a bassa frequenza causano basse vibrazioni all'apice dell'orecchio interno;
    \item l'onda viene trasmessa dal dotto cocleare alla scala timpanica;
    \item onda descendente;
    \item l'impatto dell'onda causa il movimento della membrana timpanica secondaria.
\end{enumerate}
Grazie alla membrana basilare, le vibrazioni sono regolate in base alla frequenza del suono.
\begin{figure}[H]
        \centering
        \includegraphics[width=0.9\textwidth]{cap11/trasmissione} 
        \caption{Trasmissione del suono verso l'orecchio} 
        \label{fig:trasmissione}
\end{figure}
\section{Definizione e caratteristiche del suono}
Secondo il Merriam-Webstar Dictionary definisce il suono come:  
\begin{itemize}
    \item un'impressione uditiva particolare;
    \item una sensazione percepita dal senso dell'udito;
    \item l'energia meccanica radiante trasmessa dalle onde di pressione longitudinali in un mezzo materiale, come l'aria, che è la causa oggettiva dell'udito.
\end{itemize}
La formula analitica del suono è la seguente:
\begin{equation*}
    x(t) = A \cos{\left(\omega t + \varphi\right)}
\end{equation*}
\begin{figure}[H]
        \centering
        \includegraphics[width=0.7\textwidth]{cap11/suono} 
        \caption{Rappresentazione del suono} 
        \label{fig:suono1}
\end{figure}
A questo punto, si elencano le caratteristiche del suono.
\subsection{Dimensioni fisiche del suono}
Dalla formula analitica, si ricavano le seguenti dimensioni fisiche:
\begin{itemize}
    \item ampiezza ($A$), che indica l'altezza di un ciclo, che è correlata alla percezione della sonorità;
    \item lunghezza d'onda ($\lambda$), che è la distanza tra i picchi di un'onda;
    \item frequenza ($\nu$), che il numero di cicli al secondo, correlata alla percezione dell'altezza.
\end{itemize}
\begin{figure}[htbp]
        \centering
        \includegraphics[width=0.7\textwidth]{cap11/suono2} 
        \caption{Rappresentazione delle dimensioni fisiche del suono} 
        \label{fig:suono2}
\end{figure}
\subsection{Dimensioni psicologiche del suono}
La psicoacustica come lo studio della correlazione tra la fisica degli stimoli acustici e le sensazioni uditive, definendo le cosiddette dimensioni
psicologiche. \\
La prima dimensione è il pitch, in cui descrive la sensibilità dell'orecchio umano a percepire suoni a frequenze diverse. Le frequenza in cui 
l'orecchio umano riesce a percepire da $20\,\text{Hz}\,-\,20\,\text{kHz}$, tuttavia è più sensibile alle frequenze medie ($0,5-5\,\text{kHz}$).\\
Un'altra dimensione psicologica è l'intensità ($I$), che si definisce come il rapporto tra l'energia sonora ($P$) e l'area perpendicolare alla propagazione
($S$), che si misura in $\text{W}/\text{m}^2$. La soglia di udibilità è pari a $I_0=10^{-12} \, \text{W}/\text{m}^2$.
\begin{equation*}
    I=\frac{P}{S}
\end{equation*}
Tuttavia, per convenzione si usa la scala decibel, che è una scala logaritmica definita nel modo seguente.
\begin{equation*}
    L=10\log{\left(\frac{I}{I_0}\right)}
\end{equation*}
Inoltre, a $120\,\text{dB}$ ($1\,\text{W}/\text{m}^2$), si ha la cosiddetta soglia del dolore.
\begin{figure}[H]
        \centering
        \includegraphics[width=0.9\textwidth]{cap11/dB} 
        \caption{Misura dell'intensità più comuni} 
        \label{fig:intensità}
\end{figure}
Inoltre, un'altra dimensione psicologica fondamentale è il timbro, che non è altro che lo spettro di un suono a cui vengono aggiunti dei modelli complessi. Da qui:
\begin{itemize}
    \item multipli della frequenza fondamentale generano musica;
    \item multipli di frequenze non correlate generano rumore.
\end{itemize}
Infine, l'ultima dimensione psicologica è il mascheramento, che si verifica nel momento in cui la percezione di un suono interferisce con un altro. In
particolare, ne esistono di due tipologie:
\begin{itemize}
    \item mascheramento di frequenza, in cui i suoni più forti a bassa frequenza tendono a mascherare quelli più deboli ad alta frequenza;
    \item mascheramento temporale, che protegge l'orecchio umano da suoni forti, contraendosi leggermente.
\end{itemize}
\vfill
\chapter{Multimedia}
ffe
\section{Titolo}
\subsection{1}
\subsection{2}
\section{Titolo 2}
\vfill

\begin{titlepage}
    
  \thispagestyle{empty} % Nasconde il numero della pagina.
        
  % === Immagine di Sfondo (Non ruotata) ===
  \AddToShipoutPictureBG*{%
      \begin{tikzpicture}[remember picture, overlay]
        \node[opacity=1, inner sep=0pt] at (current page.center) {
          \includegraphics[width=\paperwidth, height=\paperheight]{sfondo_sfocato.jpg}
        };
      \end{tikzpicture}%
  }
\null
\vfill
\end{titlepage}
\end{document}