In questo capitolo, vengono spiegato il suono, introducendo però prima come funziona l'udito umano.
\section{Orecchio umano}
In questo paragrafo, viene spiegato prima la struttura dell'orecchio umano, per poi della trasmissione del suono.
\subsection{Struttura dell'orecchio umano}
L'orecchio umano è divisa in tre componenti principali:
\begin{itemize}
    \item orecchio esterno, che include i padiglioni auricolari ed il condotto uditivo;
    \item orecchio medio, che contiene il timpano e gli ossicini;
    \item orecchio interno, che contiene la finestra ovale e le vie nervose.
\end{itemize}
\begin{figure}[H]
        \centering
        \includegraphics[width=0.9\textwidth]{cap11/orecchio} 
        \caption{Struttura dell'orecchio umano} 
        \label{fig:orecchio}
\end{figure}
\subsection{Trasmissione del suono}
La trasmissione del suono avviene nel modo seguente:
\begin{enumerate}
    \item le onde sonore colpiscono il timpano facendolo vibrare;
    \item gli ossicini vibrano all'unisono;
    \item la staffa si muove dentro e fuori dalla finestra ovale;
    \item le onde sonore vengono trasmesse attraverso la perilinfa;
    \item le onde ad alta frequenza causano basse vibrazioni alla base dell'orecchio interno, mentre quelle a bassa frequenza causano basse vibrazioni all'apice dell'orecchio interno;
    \item l'onda viene trasmessa dal dotto cocleare alla scala timpanica;
    \item onda descendente;
    \item l'impatto dell'onda causa il movimento della membrana timpanica secondaria.
\end{enumerate}
Grazie alla membrana basilare, le vibrazioni sono regolate in base alla frequenza del suono.
\begin{figure}[H]
        \centering
        \includegraphics[width=0.9\textwidth]{cap11/trasmissione} 
        \caption{Trasmissione del suono verso l'orecchio} 
        \label{fig:trasmissione}
\end{figure}
\section{Definizione e caratteristiche del suono}
Secondo il Merriam-Webstar Dictionary definisce il suono come:  
\begin{itemize}
    \item un'impressione uditiva particolare;
    \item una sensazione percepita dal senso dell'udito;
    \item l'energia meccanica radiante trasmessa dalle onde di pressione longitudinali in un mezzo materiale, come l'aria, che è la causa oggettiva dell'udito.
\end{itemize}
La formula analitica del suono è la seguente:
\begin{equation*}
    x(t) = A \cos{\left(\omega t + \varphi\right)}
\end{equation*}
\begin{figure}[H]
        \centering
        \includegraphics[width=0.7\textwidth]{cap11/suono} 
        \caption{Rappresentazione del suono} 
        \label{fig:suono1}
\end{figure}
A questo punto, si elencano le caratteristiche del suono.
\subsection{Dimensioni fisiche del suono}
Dalla formula analitica, si ricavano le seguenti dimensioni fisiche:
\begin{itemize}
    \item ampiezza ($A$), che indica l'altezza di un ciclo, che è correlata alla percezione della sonorità;
    \item lunghezza d'onda ($\lambda$), che è la distanza tra i picchi di un'onda;
    \item frequenza ($\nu$), che il numero di cicli al secondo, correlata alla percezione dell'altezza.
\end{itemize}
\begin{figure}[htbp]
        \centering
        \includegraphics[width=0.7\textwidth]{cap11/suono2} 
        \caption{Rappresentazione delle dimensioni fisiche del suono} 
        \label{fig:suono2}
\end{figure}
\subsection{Dimensioni psicologiche del suono}
La psicoacustica come lo studio della correlazione tra la fisica degli stimoli acustici e le sensazioni uditive, definendo le cosiddette dimensioni
psicologiche. \\
La prima dimensione è il pitch, in cui descrive la sensibilità dell'orecchio umano a percepire suoni a frequenze diverse. Le frequenza in cui 
l'orecchio umano riesce a percepire da $20\,\text{Hz}\,-\,20\,\text{kHz}$, tuttavia è più sensibile alle frequenze medie ($0,5-5\,\text{kHz}$).\\
Un'altra dimensione psicologica è l'intensità ($I$), che si definisce come il rapporto tra l'energia sonora ($P$) e l'area perpendicolare alla propagazione
($S$), che si misura in $\text{W}/\text{m}^2$. La soglia di udibilità è pari a $I_0=10^{-12} \, \text{W}/\text{m}^2$.
\begin{equation*}
    I=\frac{P}{S}
\end{equation*}
Tuttavia, per convenzione si usa la scala decibel, che è una scala logaritmica definita nel modo seguente.
\begin{equation*}
    L=10\log{\left(\frac{I}{I_0}\right)}
\end{equation*}
Inoltre, a $120\,\text{dB}$ ($1\,\text{W}/\text{m}^2$), si ha la cosiddetta soglia del dolore.
\begin{figure}[H]
        \centering
        \includegraphics[width=0.9\textwidth]{cap11/dB} 
        \caption{Misura dell'intensità più comuni} 
        \label{fig:intensità}
\end{figure}
Inoltre, un'altra dimensione psicologica fondamentale è il timbro, che non è altro che lo spettro di un suono a cui vengono aggiunti dei modelli complessi. Da qui:
\begin{itemize}
    \item multipli della frequenza fondamentale generano musica;
    \item multipli di frequenze non correlate generano rumore.
\end{itemize}
Infine, l'ultima dimensione psicologica è il mascheramento, che si verifica nel momento in cui la percezione di un suono interferisce con un altro. In
particolare, ne esistono di due tipologie:
\begin{itemize}
    \item mascheramento di frequenza, in cui i suoni più forti a bassa frequenza tendono a mascherare quelli più deboli ad alta frequenza;
    \item mascheramento temporale, che protegge l'orecchio umano da suoni forti, contraendosi leggermente.
\end{itemize}
\vfill