Il formato JPEG (Joint Photographic Expert Group) è sicuramente il formato delle immagini più conosciuto ed usato, poiché ottiene dei risultati molto
notevoli nelle foto (sia in bianco e nero che a colori), tuttavia non per i cartoni animati e per le immagini generate al computer, nonostante implementi
un modello di compressione lossy. In particolare, ne esistono due versioni, entrambe trattate in questo capitolo.
\section{Prima versione di JPEG}
Di seguito è mostrato uno schema che riassume i passi eseguiti nel formato di compressione della prima versione di JPEG.
\begin{figure}[htbp]
        \centering
        \includegraphics[width=1\textwidth]{cap6/schema1} 
        \caption{Schema del modello di compressione JPEG} 
        \label{fig:schemaJPEG1}
\end{figure}
\subsection{Fase 1: conversione da RGB a YCbCr}
Un'immagine a colori viene rappresentata nel formato RGB, che non sono altre tre matrici dove R contiene i livelli di rosso, G di verde e B di blu. tuttavia
tale rappresentazione non risulta efficace nella compressione, dato che sono ridondanti. Per questo motivo, si convertono tali matrici nel formato YCbCr, dove Y è luminanza e CbCr
rappresentano la crominanza, che insieme non sono affatto ridondanti. Infatti, l'occhio umano percepisci la variazione di luminosità che di colore, dato
dal fatto che l'occhio umano ha molti più bastoncelli che coni. La formula di conversione è la seguente:
\begin{equation*}
    \begin{bmatrix}
        Y\\
        Cb\\
        Cr
    \end{bmatrix}=
    \begin{bmatrix}
        0,299 & 0,587 & 0,114\\
        -0,1687 & -0,3313 & 0,5\\
        0,5 & -0,4187 & -0,0813
    \end{bmatrix} \begin{bmatrix}
        R\\
        G\\
        B
    \end{bmatrix}+ \begin{bmatrix}
        0\\
        128\\
        128
    \end{bmatrix}
\end{equation*}
Inoltre, la risoluzione delle componenti cromatiche vengono ridotte di un fattore $2$. In particolare:
\begin{itemize}
    \item $4:4:4$ non si ha nessun sottocampionamento;
    \item $4:2:2$ si ha una riduzione solamente nella direzione orizzontale;
    \item $4:2:0$ si ha una riduzione sia nella direzione orizzontale sia in quella verticale.
\end{itemize}
\subsection{Fase 2: trasformata DCT}
A questo punto, viene suddivisa l'immagine in blocchi da $8x8$, poiché è la dimensione che permette la migliore qualità rispetto alla dimensione del file.
Per ogni blocco, viene eseguita la cosiddetta DCT (Discete Cosine Transform), che lavora nell'intervallo da $-128$ a $128$, perciò il blocco deve essere
prima traslato negativamente di $128$. A questo punto, la formula della DCT è la seguente:
\begin{equation*}
    G(u,v)=\alpha(u)\alpha(v)\sum_{x=0}^{7}\sum_{y=0}^{7}g(x,y)\cos{\left[\frac{\pi}{8}\left(x+\frac{1}{2}\right)u\right]}\cos{\left[\frac{\pi}{8}\left(y+\frac{1}{2}\right)v\right]}
\end{equation*}
dove la moltiplicazione dei coseni non dipende dai valori di $g(x,y)$ e $\alpha$ è una funzione di normalizzazione.
Inoltre, dei $64$ coefficienti ottenuti:
\begin{itemize}
    \item $G(0,0)$ è il nucleo e viene classificato come coefficiente DC;
    \item gli altri $63$ coefficienti vengono classificati come coefficienti AC.
\end{itemize}
Infine, i motivi per cui si usa la DCT e non la trasformata di Fourier sono:
\begin{itemize}
    \item la DCT è reale pura, mentre la trasformata di Fourier è complessa;
    \item la DCT presenta meno coefficienti di qualsiasi segnale;
    \item il nucleo della trasformata diretta ed inversa sono gli stessi nella DCT.
\end{itemize}
\subsection{Fase 3: quantizzazione}
Nella fase di quantizzazione avviene la perdita vera e propria di informazione della compressione. Infatti, viene diviso punto punto per una determinata
matrice e approssimato alla parte intera più piccola. Siccome la quantizzazione avviene in base ad una soglia, viene ridotto il numero di bit per
campionamento. La formula è la seguente:
\begin{equation*}
    T^{*}(u,v)= \left\lfloor \frac{T(u,v)}{Z(u,v)} \right\rfloor
\end{equation*}
dove:
\begin{itemize}
    \item $T(u,v)$ è il coefficiente trasformato;
    \item $Z(u,v)$ è il coefficiente trasformato normalizzato;
    \item $T^{*}(u,v)$ è il coefficiente sogliato e quantizzato dell'approssimazione di $T(u,v)$.
\end{itemize}
\subsection{Fase 4: pattern zig-zag}
A questo punto, vengono ordinati i coefficienti usando un pattern zig-zag, in questo modo vengono ottenute sequenze consecutive di $0$ molto più lungh
rispetto che farlo per riga.
\begin{figure}[htbp]
        \centering
        \includegraphics[width=1\textwidth]{cap6/zigzag} 
        \caption{Pattern zig-zag} 
        \label{fig:zig-zag}
\end{figure}
\subsection{Fase 5: Codifica di entropia}
A questo punto:
\begin{itemize}
    \item tutti i coefficienti DC vengono codificati con la DPCM (Differential pulse-code modulation), che non è altro che la differenza tra i
    coefficienti DC ed i coefficienti DC dell'immagine precedente;
    \item tutti i coefficienti AC vengono codificati utilizzando la RLC, siccome sono presenti molti $0$ consecutivi.
\end{itemize}
Infine, tutti i coefficienti vengono codificati in una sequenza binaria, come quella di Huffmann e quella aritmetica; ed infine viene rieseguita la
IDCT per riottenere l'immagine compressa.
\section{JPEG2000}
La seconda
\vfill