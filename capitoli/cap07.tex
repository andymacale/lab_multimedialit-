In questo capitolo, viene trattato come avviene la percezione dello spazio tridimensionale, partendo solamente da un'immagine, che invece possiede
solamente due dimensioni. Per prima cosa, è necessiario comprende il funzionamento del nervo oculomotore.
Per prima cosa, avviene l'accomodamento, in cui il cristallino si comporta come una lente convergente di distanza focale variabile. Inoltre, il muscolo
ciliare si contrae e si allunga, variando forma e quindi la distanza focale.
Perciò, nel momento in cui si guardano oggetti lontani, il cristallino si appiattisce, mentre quando si osservano oggetti vicini diventa più spesso.
\begin{figure}[htbp]
        \centering
        \includegraphics[width=0.9\textwidth]{cap07/adattamento} 
        \caption{Processo di adattamento} 
        \label{fig:accomodamento}
\end{figure}
In contemporanea, avviene la cosiddetta vergenza, che non è altro che un movimento simultaneo degli occhi in direzioni opposte, che consentono la fusione.
In particolare:
\begin{itemize}
    \item gli occhi ruotano uno verso l'alto per guardare oggetti vicini, detta convergenza;
    \item gli occhi ruotano in direzioni opposte per guardare oggetti lontani, denominata divergenza.
\end{itemize}
\begin{figure}[htbp]
        \centering
        \includegraphics[width=0.6\textwidth]{cap07/vergenza} 
        \caption{Processo di vergenza} 
        \label{fig:vergenza}
\end{figure}
\section{Indicatori monoculari di profondità statici}
In questo paragrafo, vengono trattati i cosiddetti indicatori monoculari di profondità statici, cioè quelle caratteristica che forniscono delle
informazioni tridimensionali in immagini. Come schema, conviene analizzare il dipinto \textit{A Rainy Day in Paris}.
\begin{figure}[htbp]
        \centering
        \includegraphics[width=0.9\textwidth]{cap07/quadro} 
        \caption{\textit{A Rainy Day in Paris} che mostra le occlusioni (1), le dimensioni relative (2), le tessiture (3), la prospettiva lineare (4), la prospettiva aerea (5) e le ombre (6)} 
        \label{fig:quadro}
\end{figure}
\subsection{Occlusioni}
Il primo indicatore sono sicuramente le occlusioni, che si definiscono come gli oggetti più vicini che bloccano l'accesso visivo a quello più distanti.
Per esempio, nella figura seguente viene mostrata una ragazza che viene parzialmente coperta da un tronco di una palma.
\begin{figure}[htbp]
        \centering
        \includegraphics[width=0.8\textwidth]{cap07/occlusioni} 
        \caption{Esempio di occlusione} 
        \label{fig:occlusione}
\end{figure}
\subsection{Dimensioni relative}
Le dimensioni relative forniscono informazioni 3D, grazie ad oggetti della stessa dimensione fisica proiettano immagini retiniche di dimensione diversa
a seconda del punto del punto di osservazione. Per esempio, è molto più facile conoscere l'altezza di una statua dell'isola di Pasqua con delle persone
(che hanno un'altezza media di 1,80 m), rispetto a che non ci sono persone od altri oggetti di riferimento: infatti, anche l'esperienza gioca 
un ruolo fondamentale in questo contesto.
\begin{figure}[htbp]
        \centering
        \includegraphics[width=0.9\textwidth]{cap07/relative} 
        \caption{Esempio di dimensioni relative} 
        \label{fig:relative}
\end{figure}
\subsection{Tessitura}
Per tessitura consiste in una superficie molto estesa con una tessitura non uniforme proietterà sulla retina un'immagine, in cui la frequenza della tessitura
aumenta all'aumentare della distanza dal punto di osservazione.
\begin{figure}[htbp]
        \centering
        \includegraphics[width=0.9\textwidth]{cap07/tessitura} 
        \caption{Esempio di tessitura} 
        \label{fig:tessitura}
\end{figure}
\subsection{Prospettiva lineare}
La prospettiva lineare consiste in linee parallele sul piano visivo convergono verso il punti di fuga, all'aumentare della distanza di osservazione: ciò
fornisce informazioni sulle distanze nello spazio tridimensionale. Un classico esempio sono i binari del treno: l'informazione aggiunta ce l'hai grazie
ai bastoncini tra le due rotaie (due linee parallele).
\begin{figure}[htbp]
        \centering
        \includegraphics[width=0.9\textwidth]{cap07/lineare} 
        \caption{Esempio di prospettiva lineare} 
        \label{fig:lineare}
\end{figure}
\subsection{Prospettiva aerea}
La prospettiva aerea descrive il fatto che la luce degli oggetti più lontani, prima di raggiungere la nostra retina, deve attraversare più parti dell'
atmosfera rispetto a quella degli oggetti più vicini.
\begin{figure}[htbp]
        \centering
        \includegraphics[width=0.8\textwidth]{cap07/aerea} 
        \caption{Esempio di prospettiva aerea} 
        \label{fig:aerea}
\end{figure}
\subsection{Ombre}
L'occlusione delle fonte di luce crea delle ombre, permettono di avere informazioni sulla struttura tridimensionale della scena.
\begin{figure}[htbp]
        \centering
        \includegraphics[width=0.8\textwidth]{cap07/ombre} 
        \caption{Esempio di ombre} 
        \label{fig:ombre}
\end{figure}
\section{Visione binoculare}
Per avere la visione binoculare, è necessario che si verificano tre fasi fondamentali, i cosiddetti gradi di Worth:
\begin{itemize}
        \item percezione simultanea;
        \item fusione;
        \item stereopsi.
\end{itemize}
Tuttavia, il concetto fondamentale da comprendere sta nel fatto che, ogni fenomeno è di livello superiore al precedente e la presenza del grado più
elevato. Infine, la stereopsi prevede la presenza dei due precedenti. In questo paragrafo, verranno trattati nel dettaglio.
\subsection{Percezione simultanea}
La percezione simultanea si definisce come la capacità di percepire contemporaneamente le immagini sui due occhi. Infatti, entrambi gli occhi cooperano
per formare un'unica percezione. I campi visivi dei due occhi sono legati reciprocamente. Difatti, un area della retina dell'occhio destro corrisponde
ad un'altra dell'occhio sinistro e viceversa. I principali punti corrispondenti prendono il nome di fovee.
\subsection{Fusione}
Per quanto riguarda la fusione, si fa riferimento a due tipologie diverse.
La prima è la fusione motoria, che contribuisce semplicemente a mantenere la fusione di due immagini sfruttando i muscoli oculari.\\
La seconda tipologia, ed anche la più significativa, è la funsione sensioriale, che rappresenta la capacità intrinseca del cervello di elaborare e integrare le informazioni visive grezze provenienti dai due occhi, unendole in un'unica percezione visiva (immagine singola).\\
A questo punto, si fa riferimento alla legge della corrispondenza sensoriale, che afferma che:
\begin{itemize}
        \item punti retinici corrispondenti sono dei punti che hanno in comune una direzione visiva soggettiva, che vengono fuse se vengono visti nella
                stessa direzione visiva e se sono sufficientemente simili;
        \item le fovee sono punti retinici corrispondenti, cioè hanno lo valore spaziale nullo, che corrisponde alla direzione visiva principale.
\end{itemize}
Inoltre, un soggetto ha corrispondenza retinica normale quando la stimolazione di entrambi le fovee o di punti retinici corrispondenti dà origine 
ad un percetto unitario. Il processo di fusione, invece, risulta faticoso se gli oggetti sono troppo vicini oppure se la loro profondità 
cambia rapidamente.
\begin{figure}[H]
        \centering
        \includegraphics[width=0.8\textwidth]{cap07/crn} 
        \caption{Se $\alpha_{BL}$ e $\alpha_{BR}$ sono uguali, allora $BR$ e $BL$ sono punti retinici corrispondenti} 
        \label{fig:crn}
\end{figure}
 Infine, si fa riferimento a:
 \begin{itemize}
        \item oroptero, che è il luogo di tutti i punti nello spazio in cui le immagini cadono su punti retinici corrispondenti per un angolo di convergenza 
        $\beta$ ed i punti all'esterno della cosiddetta area di Panum, generano immagine doppie;
        \item diplopia fisiologica, che è lo sdoppiamento fisiologico dei punti che giacciono fuori dall'oroptero, perciò $BR$ e $BL$ sono punti retici disparati,
                perciò hanno due direzioni visive soggettive diverse. 
 \end{itemize}
 \begin{center}
\begin{minipage}{0.48\textwidth} % Colonna Sinistra (48% della larghezza del testo)
    \centering
        \includegraphics[width=0.7\textwidth]{cap07/oroptero}     
\end{minipage}
\hfill % Spazio elastico per separare le due colonne
\begin{minipage}{0.48\textwidth} % Colonna Destra (48% della larghezza del testo)
    \centering
        \includegraphics[width=0.7\textwidth]{cap07/diplopia} 
\end{minipage}
\captionof{figure}{Descrizione dell'oroptero (a sinistra) e della diplopia fisiologica (a destra)}
\label{fig:dcdkjcddjdcjd}
\end{center}
\subsection{Stereopsi}
La stereopsi è la capacità del cervello di fondere due immagini provienti dagli occhi per costruire una visione tridimensionale di un oggetto osservato.
Inoltre, esiste un occhio dominante, che è quello che prefisce un input visivo rispetto all'altro. Per verificare qual è il proprio occhio dominate,
si esegue il test di Miles.
\section{Applicazione ad immagini statiche}
A questo punto, non resta che capire la perceziona della profondità a partire da un'immagine. Il cervello umano percepisce la profondità, grazie a due
viste provenienti da due occhi, che sono leggermente traslati, la cui distanza di traslazione prende il nome di disparità. A questo punto, la disparità
dipende dalla disparità e a indicatori monoculari. Per questo motivo, sono nati gli stereogrammi, di cui ne sono mostrati i seguenti tre:
\begin{itemize}
        \item stereoscopio di Wheatstone, il cui funzionamento si basa su due specchi posizionati ad angolo, in cui ogni occhio vede una singola immagine
        ma il cervello le vede come se venissero da un unico punto e le fonde, creando una percezione di tridimensionalità;
        \item coppie stereo d'autore, che non sono altro che due dipinti quasi identici, che affiancati il cervello li affianca per creare un'immagine
        singola, creando un'illusione di profondità;
        \item anaglifi, le cui due immagini corrispondenti alla vista sono stampate sullo stesso supporto, ma una ha un supporto in blu e l'altra in rosso,
                creando profondità grazie alla scarsa informazione cromatica.
\end{itemize}
\begin{center}
\begin{minipage}{0.30\textwidth} % Colonna Sinistra (48% della larghezza del testo)
    \centering
        \includegraphics[width=0.7\textwidth]{cap07/stereo}     
\end{minipage}
\hfill % Spazio elastico per separare le due colonne
\begin{minipage}{0.30\textwidth} % Colonna Destra (48% della larghezza del testo)
    \centering
        \includegraphics[width=0.7\textwidth]{cap07/autore} 
\end{minipage}
\hfill % Spazio elastico per separare le due colonne
\begin{minipage}{0.30\textwidth} % Colonna Destra (48% della larghezza del testo)
    \centering
        \includegraphics[width=0.7\textwidth]{cap07/anaglifo} 
\end{minipage}
\captionof{figure}{Dalla prima immagine a sinistra: stereoscopio di Wheatstone, coppie stereo d'autore e anaglifo}
\label{fig:dcdkjcddjdcjds}
\end{center}
\vfill