La creazione di un video avviene mediante una sequenza di immagini 2D, che rappresentano la proiezione di una scena 3D in movimento sul piano immagine
della videocamera.
\begin{figure}[H]
        \centering
        \includegraphics[width=0.9\textwidth]{cap8/primo} 
        \caption{Primo video mai realizzato} 
        \label{fig:primo}
\end{figure}
In questo capitolo, verranno analizzate le caretteristiche del video.
\section{Proporzioni}
La prima caratteristica fondamentale del video è la proporzione, che non è altro che il rapporto tra l'larghezza e l'altezza. Tale rapporto è fisso:
infatti risulta tre valori standard:
\begin{itemize}
    \item 4:3, usato nelle vecchie TV, per i canali SD, poiché risulta troppo squadrato;
    \item 16:9, usato nella maggior parte delle TV, per i canali DC;
    \item 21:9, usato nella maggior parte dei video, ma invece in pochissime TV.
\end{itemize}
Nel caso in cui il video e la TV hanno proporzioni diversi, verranno mostrati degli spazi vuoti, come mostrato nell'immagine di seguito.
\begin{figure}[H]
        \centering
        \includegraphics[width=0.9\textwidth]{cap8/rapporti}
        \caption{Video mostrati in varie TV} 
        \label{fig:rapporti}
\end{figure}
\section{Frame rate}
Per dare la sensazione di movimento, le immagini devono essere ricaricate, in generale a frequenza di $60 Hz$. Il frame rate non è altro che il tempo
impiegato tra due immagini successive, espresso in frame per seconds (fps). Solitamente, il frame rate per essere considerato accettabile deve essere
almeno 30 fps, mentre è buono a 60 fps, eccellente sopra i 120 fps.
\begin{figure}[htbp]
        \centering
        \includegraphics[width=0.8\textwidth]{cap8/fps} 
        \caption{Come varia il frame rate in un video} 
        \label{fig:fps}
\end{figure}
\section{Modalità di trasmissione}
Una caratteristica molto importante è come avviene la trasmissione di un video, a seconda della qualità richiesta.\\
Il metodo di trasmissione più semplice, ma che richiede molto più risoluzione e larghezza di banda è il metodo progressivo, in cui ogni immagine viene
trasmessa esattamente come viene mostrata.\\
Se invece, non si possiede molta larghezza di banda oppure il framerate non è molto alto, allora si opta per il metodo interlacciato, che consiste nel
trasmettere prima le linee pari e poi quelle dispari, creando l'illusione del frame rate doppio rispetto a quello reale. Tuttavia, è necessario fornire
le seguenti considerazioni:
\begin{itemize}
        \item è necessario un tempo di ritracciamento, che crea degli impulsi neri;
        \item occorrono dei segnali di sincronizzazione, per garantire che l'immagine inizi in alto a sinistra;
        \item per evitare lo sfarfallio, si deve effettuare un aggiornamento completo a $50 Hz$.
\end{itemize}
\section{Risoluzione}
Analogamente per le immagini, anche per quanto riguarda i video la risoluzione non è altro che il numero di pixel presenti. Tuttavia, qui è cruciale
anche le proporzioni, il frame rate e la modalità di trasmissione. In particolare:
\begin{itemize}
        \item formato PAL (Phase Alternate Line), sviluppato principalmente in Europa (Francia esclusa ad esempio), in 4:3 a 25 fps ad una risoluzione 
              massima a 720x576px;
        \item HD ready, con risoluzione a 720p (progressivo) a 50 fps (720p/50);
        \item HDTV (1080i/25);
        \item Full HDTV (1080p/50);
        \item Ultra HD, composto dal 2K, dal 4K e dall'8K.
\end{itemize}
\begin{figure}[H]
        \centering
        \includegraphics[width=0.7\textwidth]{cap8/risoluzione} 
        \caption{La differenza tra le varie risoluzioni} 
        \label{fig:risoluzionevideo}
\end{figure}
\section{Rappresentazione dei colori}
Come accennato all'inizio del corso, i coni presenti nell'occhio umano rispondono a tre lunghezze d'onda fondamentali:
\begin{itemize}
        \item rosso ($700 nm$);
        \item verde ($546,1 nm$);
        \item blu ($435,8 nm$).
\end{itemize}
Tale lunghezze d'onda fanno parte dello spettro visibile.
\begin{figure}[htbp]
        \centering
        \includegraphics[width=0.9\textwidth]{cap8/spettro} 
        \caption{Spettro della luce visibile} 
        \label{fig:spettrocolori}
\end{figure}
\subsection{Teorie della tipologia di colore}
Per formare ogni colore, si considerano due teorie, a seconda del comportamento dell'assorbimento della luce.\\
La prima è la teoria sottrattiva, in cui il nero assorbe più luce ed il bianco riflette più luce. A questo punto, tutti i colori, esclusi quelli del
pigmento, vengono assorbiti. In questo caso, i colori primari sono il ciano, il magenta, il giallo ed il nero. Tale teoria trova largo utilizzo nelle 
stampanti.
Tuttavia, nei video si usa l'altra teoria, ossia la teoria additiva, in cui il nero riflette più luce, mentre il bianco asssorbe più luce. A questo
punto, i colori primari sono gli stessi della luce visibile, perciò il rosso, il verde ed il blu.
\begin{center}
\begin{minipage}{0.48\textwidth} % Colonna Sinistra (48% della larghezza del testo)
    \centering
        \includegraphics[width=0.7\textwidth]{cap8/pigmento}     
\end{minipage}
\hfill % Spazio elastico per separare le due colonne
\begin{minipage}{0.48\textwidth} % Colonna Destra (48% della larghezza del testo)
    \centering
        \includegraphics[width=0.7\textwidth]{cap8/luce} 
\end{minipage}
\end{center}
\captionof{figure}{Teoria sottrattiva (a sinistra) e teoria additiva (a destra)}
\label{fig:dcdksjcddjdcjds}
\subsection{Rappresentazione dei colori}
Per quanto riguarda la rappresentazione dei colori, ciò può avvenire semplicimente con la ruota dei colori, detto HSV, dove:
\begin{itemize}
        \item H sta per hue, che indica la rotazione sulla ruota dei colori;
        \item S sta per saturation, che indica la saturazione;
        \item V sta per value, che indica il valore di luminosità.
\end{itemize}
Un'altra rappresentazione è RGB, che non 
\vfill