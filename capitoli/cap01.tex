In questo capitolo viene spiegato come avviene l’elaborazione delle immagini. Per prima cosa, è necessario capire come l’occhio umano cattura e percepisce l’immagine, soprattutto per comprendere quali sono i suoi grandi limiti. Una volta capito ciò, si può procedere all’elaborazione delle immagini.
\section{Sistema visivo umano}
L'occhio umano è racchiuso da tre membrane, dove ognuna ha una funzione rilevante per l'elaborazione delle immagini.
\begin{figure}[htbp]
        \centering
        \includegraphics[width=0.6\textwidth]{cap01/occhio} 
        \caption{Struttura dell'occhio umano} 
        \label{fig:occhio}
\end{figure}
\subsection{Cornea e coroide: due membrane come protezione}
La cornea è la membrana più esterna dell'occhio umano: infatti, essendo composta da un tessuto resistente e trasparente, è perfetta per racchiudere la superficie anteriore dell'occhio.
Inoltre, per ridurre la quantità di luce estranea che entra nell'occhio, è presente la coroide.
\subsection{Retina: la membrana per la vista}
La retina è quella membrana che fornisce il senso della vista all'essere umano. In particolare, permette di mettere a fuoco gli oggetti, grazie alla luce dell'oggetto stesso che entra nella retina.\\ Inoltre, sono presenti i cosiddetti percettori luminosi, che rendono possibile la visione a pattern. \\Il primo percettore sono i coni, che permettono la visione cromatica: infatti possono essere a lunghezza d'onda corta per il blu, a lunghezza d'onda media per il verde ed a lunghezza d'onda lunga per il rosso.\\ Per la visione acromatica, invece, sono presenti i bastoncelli, per esempio la visione scotopica e la penombra.
\begin{figure}[htbp]
        \centering
        \includegraphics[width=1\textwidth]{immagini/cap01/coni-bastoncelli-retina.jpeg} 
        \caption{Coni e bastoncelli} 
        \label{fig:coni}
\end{figure}
\subsection{Formazione dell'immagine}
A questo punto, la formazione dell'immagine avviene nel modo seguente. Le lenti dell'occhio umano sono flessibili: la sua forma è controllata delle fibre del corpo ciliare. \\Inoltre, l'abilità dell'occhio di discriminare i cambiamenti delle intensità di luce a qualsiasi livello di adattamento specifico, è descritta dalla legge di Weber.
\begin{equation*}
    k=\frac{\Delta I_C}{I}
    \label{eq:Legge di Weber}
\end{equation*}
In particolare:
\begin{itemize}
    \item $k$ è la costante di Weber, che è un valore costante e caratteristico per ogni specifica modalità sensoriale;
    \item $\Delta I_C$ è la soglia differenziale, ossia la quantità minima di cambiamento affinché il soggetto percepisca una differenza;
    \item $I$ è l'intensità di riferimento dello stimolo.
\end{itemize}
Ciò spiega che il sistema visivo tende a sottostimare od a sovrastimare i bordi delle regioni a diverse intensità. Ciò genera le illusioni: esempi sono l'illusione di dolcezza di mais e la griglia di Hermann.
\begin{figure}[H]
    \centering % Centra l'intero blocco delle immagini e della didascalia
    \begin{minipage}{0.48\textwidth} % Minipage che occupa quasi metà della larghezza del testo
        \centering % Centra l'immagine all'interno della minipage
        \includegraphics[width=\textwidth]
        {immagini/cap01/Cornsweet_illusion.png} % L'immagine occupa tutta la larghezza della minipage
        \label{fig:cornsweet}
    \end{minipage}
    \hfill
    \begin{minipage}{0.48\textwidth} % Minipage che occupa quasi metà della larghezza del testo
        \centering % Centra l'immagine all'interno della minipage
        \includegraphics[width=\textwidth]{immagini/cap01/Illusion_of_gray_dots_nevit_065.svg.png} % L'immagine occupa tutta la larghezza della minipage
        \label{fig:grigliall}
    \end{minipage}
    \caption{Illusione del mais dolce (a sinistra) e la griglia di Hermann (a destra)}
    \label{fig:confronto_immagini}
\end{figure}
\section{Rappresentazione delle immagini}
Un'immagine è una funzione bidimensionale ($f(x,y)$), definita come il prodotto tra due funzioni bidimensionali:
\begin{itemize}
    \item la luminanza ($i(x,y)$), che è la quantità di luce della sorgente incidente sulla scena osservata;
    \item la riflettanza ($r(x,y)$), che è la quantità di luce riflessa dall'oggetto nella scena.
\end{itemize}
\begin{equation*}
    f(x,y)=i(x,y)r(x,y)
    \label{eq:f}
\end{equation*}
\begin{equation*}
    0 < i(x,y) < \infty \\
    \label{eq:Luminanza}
\end{equation*}
\begin{equation*}
    0 < r(x,y) < 1 \\
    \label{eq:Riflettanza}
\end{equation*}
\begin{equation*}
    0 < f(x,y) < \infty \\
    \label{eq:Immagine}
\end{equation*}
\subsection{Intensità di un'immagine}
L'intensità di un'immagine monocromatica a $(x_0,y_0)$ è il livello di grigio $L$ in quel punto.
\begin{equation*}
    L=f(x_0,y_0)
    \label{eq:livello di grigio}
\end{equation*}
Il livello di grigio trovato appartiene ad un intervallo dove il minimo corrisponde al nero ed il massimo corrisponde al bianco.
\begin{equation*}
    L \in [L_{\text{min}},L_{\text{max}}]
    \label{eq:intervallo}
\end{equation*}
\subsection{Digitalizzazione delle immagini}
Per effettuare la digitalizzazione di un'immagine, sono necessari i seguenti parametri:
\begin{itemize}
    \item $M$, che è la larghezza dell'immagine;
    \item $N$, che è l'altezza dell'immagine;
    \item $L$, che è il numero dei livelli di grigi, che è un multiplo di $2^n$.
\end{itemize}
A questo punto, avviene la digitalizzazione, che si compone in due fasi. \\La prima fase è il campionamento, che consiste nel suddividere l'immagine in una griglia regolare di punti o celle. Ogni cella produce un pixel, che è l'unità minima di un'immagine digitale. A questo punto viene misurata la risoluzione in dpi, che non è altro che la densità della griglia: più è fitta più l'immagine sarà fedele all'originale.\\
La seconda ed ultima fase è la quantizzazione, che è il processo di discretizzazione dell'intensità. In particolare, ad ogni pixel gli viene assegnato un valore numerico discreto che ne codifica il livello di grigio. Infine, il valore numerico viene convertito in una stringa binaria.\\
Il risultato della digitalizzazione, produce una matrice di numeri reali di dimensioni $M \times N$.
\[
f(x,y) = \begin{bmatrix}
f(0,0) & \dots & f(0, N-1) \\
\vdots & \ddots & \vdots \\
f(M-1,0) & \dots & f(M-1, N-1)
\end{bmatrix}
\]
\section{Manipolazione delle immagini}
La manipolazione delle immagini si pone come obiettivo principale quello di modificare l'immagine originale che si adatta di più al contesto richiesto: come sfocare lo sfondo in un ritratto oppure rendere più luminose le vene in un'immagine catturata da un esame medico.\\
In questo paragrafo, vengono elencati una serie di possibili manipolazioni, andando a modificare la $f(x,y)$ con un operatore $T$ definito con i suoi vicini di $(x,y)$.
\begin{equation*}
    g(x,y)=T[f(x,y)]
    \label{eq:manipolazione}
\end{equation*}
\begin{figure}[H]
    \centering % Centra l'intero blocco delle immagini e della didascalia
    \begin{minipage}{0.48\textwidth} % Minipage che occupa quasi metà della larghezza del testo
        \centering % Centra l'immagine all'interno della minipage
        \includegraphics[width=\textwidth]{immagini/cap01/ritratto.jpg}
         % L'immagine occupa tutta la larghezza della minipage
        \label{fig:corn}
    \end{minipage}
    \hfill
    \begin{minipage}{0.48\textwidth} % Minipage che occupa quasi metà della larghezza del testo
        \centering % Centra l'immagine all'interno della minipage
        \includegraphics[width=\textwidth]{immagini/cap01/vasi.png} % L'immagine occupa tutta la larghezza della minipage
        \label{fig:griglia}
    \end{minipage}
    \caption{Esempi di possibili manipolazioni di immagini}
    \label{fig:confronto_immaginii}
\end{figure}
In particolare, si considera:\\
\begin{itemize}
    \item $r$ è il livello di grigio di input;
    \item $s$ è il livello di grigio di output.
\end{itemize}
\subsection{Manipolazioni principali}
Di seguito ne è riportato un grafico che ne riporta le principali tipologie di manipolazione delle immagini.
\begin{figure}[H]
        \centering
        \includegraphics[width=1\textwidth]{immagini/cap01/manipolazione.png} 
        \caption{Possibili manipolazioni delle immagini} 
        \label{fig:manipolazione}
\end{figure}
La prima ed anche la più semplice è la funzione identità, in cui il livello di grigio in entrata corrisponde a quello d'uscita, perciò l'immagine non riporta alcuna alterazione. Tale funzione è una retta che va da $(0,0)$ a $(L-1, L-1)$.
\begin{equation*}
    s=r
    \label{eq:identitàa}
\end{equation*}
La seconda è la funzione negativa, in cui inverte i livelli di grigio, creando la cosiddetta immagine negativa. Questa funzione non è altro che una retta che va da $(0,L-1)$ a $(L-1,0)$.
\begin{equation*}
    s=L-1-r
    \label{eq:identità}
\end{equation*}
Tale funzione risulta molto utile per risaltare regioni di grigio incorporate da regioni scure.
\begin{figure}[htbp]
    \centering
    \includegraphics[width=0.7\textwidth]{immagini/cap01/negativo.png} 
    \caption{Immagine originale (a sinistra) e immagine in negativo (a destra)} 
    \label{fig:negativo}
\end{figure}

La terza funzione è la funzione logaritmica, che tende ad espandere i valori dei pixel scuri ed ad comprimere i pixel chiari. Se invece si vuole ottenere l'opposto, allora si usa la funzione logaritmica inversa. La prima equazione è la funzione logaritmica, mentre la seconda è la funzione logaritmica inversa. Entrambi sfruttano la costante di scala $c \in [0,L-1]$.

% --- NESSUNA RIGA VUOTA QUI ---
\begin{align*}
    s &= c\ln{(1+r)} \\
    s &= \exp\left(\frac{r}{c}\right)-1 
\end{align*}
% --- LA RIGA VUOTA DEVE ESSERE QUI (dopo il blocco align*) ---

\begin{figure}[htbp]
    \centering
    \includegraphics[width=0.7\textwidth]{immagini/cap01/log.png} 
    \caption{Esempio di due immagini a cui sono state applicate con $r$ differenti} 
    \label{fig:log}
\end{figure}
L'ultimo gruppo di funzioni riguarda la correzione gamma, che è un'operazione non lineare usata per codificare e decodificare i valori di luminanza in un sistema di visualizzazione di immagini. Esse sfruttano le costanti $c$ e $\gamma$, entrambe positive.
\begin{equation*}
    s=cr^{\gamma}
    \label{eq:gamma}
\end{equation*}
Ciò avviene per due motivi principali:
\begin{itemize}
    \item l'occhio umano non percepisce la luminosità in modo lineare, perciò le immagini risulterebbero scure e con pochi dettagli nelle ombre;
    \item gli schermi più datati, come quelli a tubo catodico, hanno una risposta non lineare alla tensione in ingresso, seguendo una legge di potenza, e si è scelto di mantenere ciò anche per gli schermi più moderni per mantenere compatibilità e coerenza visiva per la percezione umana.
\end{itemize}
\begin{figure}[htbp]
    \centering
    \includegraphics[width=0.6\textwidth]{immagini/cap01/gamma.png} 
    \caption{Realizzazione della correzione gamma sui monitor} 
    \label{fig:gamma}
\end{figure}
Di seguito, viene riportata una figura che dimostra quando sia importante la calibrazione corretta della correzione gamma sui monitor.

\begin{figure}[htbp]
    \centering
    \includegraphics[width=0.7\textwidth]{immagini/cap01/calibrazione.png} 
    \caption{Calibrazione del gamma (azzurro la gamma dell'immagine, viola la gamma del display e la rossa la gamma complessiva)} 
    \label{fig:gamma2}
\end{figure}
Nella prima colonna non è stata applicata alcuna correzione; nella seconda l'applicazione è insufficiente, generando un'immagine troppo chiara; nella terza è l'applicazione ideale; infine nell'ultima l'applicazione è eccessiva generando un'immagine troppo scura.
\clearpage % Forza l'inizio del blocco su una nuova pagina

\subsection{Funzione di trasformazione lineare a tratti}
La funzione di trasformazione lineare a tratti ha la particolarità di non essere descritta da una singola equazione per tutto l'intervallo dei livelli di grigio, ma da più segmenti lineari, ognuno applicabile ad un intervallo di livello di grigio ben specifico. Il vantaggio sta nell'avere una forma con livello di complessità a scelta, a discapito, però, nell'avere più input dall'utente.

La prima funzione di trasformazione lineare a tratti è il contrast stretching, che si pone come obiettivo quello di aumentare il contrasto di un'immagine che appare sbiadita o con scarso intervallo dinamico. A questo punto, si scelgono due punti $(r_1,s_1)$ e $(r_2,s_2)$ in cui i livelli sotto $r_1$ (molto scuri) oppure sopra $r_2$ (molto chiari) vengono compressi, in altre parole vengono mappati rispettivamente quasi a $0$ e a $L-1$. Mentre se è compreso tra $r_1$ ed $r_2$ viene aumentato il contrasto, espandendo i livelli di grigio sull'intero intervallo di uscita.

% --- INIZIO BLOCCO FIGURA 10 (Contrast Stretching) ---
\begin{center}
    % Uso \begin{equation*} e \end{equation*} per contenere l'ambiente align*
    \begin{equation*}
        \begin{aligned}[t] 
        & s = 
            \begin{dcases}
                \frac{s_1}{r_1}r & \text{se } 0 \le r < r_1 \\[1.5ex]
                \frac{s_2-s_1}{r_2-r_1}(r-r_1)+s_1 & \text{se } r_1 \le r \le r_2 \\[1.5ex]
                \frac{L-1-s_2}{L-1-r_2}(r-r_1)+s_2 & \text{se } r_2 < r \le L-1
            \end{dcases}
        &
        \raisebox{-0.5\height}{\includegraphics[width=0.45\textwidth]{cap01/contrast.png}} 
        \end{aligned}
    \end{equation*}
    \captionof{figure}{Definizione analitica e grafico del contrast stretching} % <--- NOTA: Ho usato \captionof (richiede \usepackage{caption})
    \label{fig:contrast_stretching}
\end{center}
% --- FINE BLOCCO FIGURA 10 ---

Un caso limite si ha se avviene un'immagine binaria (bianco o nero), avendo il cosiddetto thresholding.

% --- INIZIO BLOCCO FIGURA 11 (Thresholding) ---
\begin{center}
    \begin{equation*}
        \begin{aligned}[t]
        & s = 
            \begin{cases}
                0 & \text{se } 0 \le r < T \\[1.5ex]
                L-1 & \text{se } T \le r \le L-1
            \end{cases}
        &
        % USARE IL NOME FILE CORRETTO PER IL THRESHOLDING!
        \raisebox{-0.5\height}{\includegraphics[width=0.45\textwidth]{cap01/thresholding.png}} 
        \end{aligned}
    \end{equation*}
    \captionof{figure}{Definizione analitica e grafico del thresholding}
    \label{fig:thresholding}
\end{center}
% --- FINE BLOCCO FIGURA 11 ---
La seconda funzione di trasformazione lineare a tratti è il gray-level slicing, che si pone come obiettivo quello di evidenziare un intervallo di livello di grigio ben specifico e sopprimere tutti gli altri. In particolare si fa riferimento a due casi ben specifici:
\begin{itemize}
    \item sfondo costante, in cui l'intervallo viene mappato ad un livello alto ($L-1$) ed il resto al livello $0$;
    \begin{equation*}
    s=s_{high}\text{rect}_{B-A}\left(r-\frac{A+B}{2}\right)
    \label{eq:sfondocosts}
    \end{equation*}
    \item sfondo invariato, in cui l'intervallo viene mappato ad un livello alto ($L-1$) ed il resto viene lasciato invariato.
    \begin{equation*}
    s=r+(s_{high}-r)\operatorname{rect}_{B-A}\left(r-\frac{A+B}{2}\right)
    \label{eq:sfondoinv}
    \end{equation*}
\end{itemize}
\begin{figure}[htbp]
    \centering
    \includegraphics[width=0.7\textwidth]{immagini/cap01/gray-level-slicing.png} 
    \caption{Gray-level slicing a sfondo costante (a sinistra) e a sfondo invariato (a destra)} 
    \label{fig:grayl}
\end{figure}
La terza ed ultima funzione di trasformazione lineare a tratti è il bit-plane slicing, che consiste nel mettere evidenza solamente i bit più significativi (che sono i primi bit).
\subsection{Equalizzazione dell'istogramma}
Quando si parla di un'immagine, è possibile costruire l'istogramma dell'immagine stessa, che non è altro che il numero di pixel che contengono un determinato valore di livello di grigio per ogni livello $n_k$.
\begin{center}
    % Uso \begin{equation*} e \end{equation*} per contenere l'ambiente align*
    \begin{equation*}
        \begin{aligned}[t] 
        & 
            f(x,y) = \begin{bmatrix}
            2 & 3 & 3 & 2 \\
             4 & 2 & 4 & 3 \\
             3 & 2 & 3 & 5 \\
              2 & 4 & 2 & 4
            \end{bmatrix}
        &
        \raisebox{-0.5\height}{\includegraphics[width=0.45\textwidth]{cap01/istogramma.png}} 
        \end{aligned}
    \end{equation*}
    \captionof{figure}{Esempio di istogramma di un'immagine 4x4 con livello di grigio $[0,9]$} % <--- NOTA: Ho usato \captionof (richiede \usepackage{caption})
    \label{fig:kjmkpl}
\end{center}
In termini statistici, è preferibile rappresentare l'istogramma normalizzato, ossia dividere $n_k$ con il numero totale di pixel $n$.
\begin{equation*}
    p(n_k)=\frac{n_k}{n}
\end{equation*}
A questo punto, la tecnica dell'equalizzazione dell'istogramma consiste nel cambiare l'istogramma dell'immagine in un istogramma uniforme, in cui la percentuale di ogni livello di grigio rimane sempre la stessa. Per effettuare ciò, occorre dei fondamenti di probabilità e statistica e di calcolo integrale.\\
Per prima cosa, questa operazione effettua una trasformata, perciò si può scrivere ciò come $s=T(r)$ e siccome la trasformata è reversibile, esiste anche la sua inversa $r=T^{-1}(s)$. Per semplicità, risulta molto conveniente studiare ciò nel continuo. Dati rispettivamente, $p_{in}(r)$ e $p_{out}(s)$, come la probabilità di livello di grigi di input e di outut, dalla teoria di probabilità, si ha la formula seguente (per $0\le r \le L-1$ e $0\le s \le L-1$).
\begin{equation*}
    p_{out}(s)=\left[p_{in}(r)\frac{ds}{dr}\right]_{r=T^{-1}(s)}
\end{equation*}
A questo punto, la trasformata di $p_{in}(r)$ è pari alla formula seguente:
\begin{equation*}
    s=T(r)=\int_{0}^{r} p_{in}(r') \, dr' , 0\le r \le 1
\end{equation*}
che è la funzione di distribuzione cumulativa (CDF). Perciò, dal teorema fondamentale del calcolo, si sfrutta la seguente formula.
\begin{equation*}
    p_{in}(r)=\frac{dr}{ds}
\end{equation*}
Infine, facendo dei semplici conti e sfruttando le proprietà delle derivate delle funzioni inverse, si ricava che:
\begin{equation*}
    p_{out}(s)=\left[p_{in}(r)\frac{ds}{dr}\right]_{r=T^{-1}(s)}=\left[p_{in}(r)\frac{1}{\frac{dr}{ds}}\right]_{r=T^{-1}(s)}=\left[p_{in}(r)\frac{1}{[p_{in}(r)}\right]_{r=T^{-1}(s)}=\left[1\right]_{r=T^{-1}(s)}=1, 0\le s \le 1
\end{equation*}
la densità di probabilità in uscita risulta uniforme.\\
Quindi per effettuare l'equalizzazione dell'istogramma:
\begin{enumerate}
    \item per ogni pixel si calcola il $p_{in}(r_k)$;
    \begin{equation*}
    p_{in}(r_k)=\frac{n_k}{n}, 0\le r_k \le 1 \,\, 0\le k \le L-1
    \label{eq:sfondocost}
    \end{equation*}
    \item basandosi sulla CDF, si esegue la trasformata discreta.
    \begin{equation*}
    s_k=T_{r_{k}}=\sum_{j=0}^{k}{p_{in}(r_j)},0\le k \le L-1
    \label{eq:lkl}
    \end{equation*} 
\end{enumerate}
Tornando all'esempio precedente, si effettua la CDF nel modo seguente.
\begin{center}
    % Il blocco matematico generale per l'allineamento
    \begin{equation*} 
        \begin{aligned}[t] 
        % COLONNA SINISTRA: La Matrice (usiamo bmatrix per le quadre)
        & f(x,y) = \begin{bmatrix}
            2 & 3 & 3 & 2 \\
            4 & 2 & 4 & 3 \\
            3 & 2 & 3 & 5 \\
            2 & 4 & 2 & 4
          \end{bmatrix}
        &
        % COLONNA DESTRA: La Tabella (NON USIAMO \includegraphics o \begin{table})
        % Usiamo \vcenter per l'allineamento verticale con la matrice
        \vcenter{\hbox{
            \begin{tabular}{|c|c|c|c|}
                \hline 
                $k$ & $r_k$ & $n_k$ & $p_{in}(r_k)$ \rule{0pt}{2ex}\\ 
                \hline 
                $2$ & $0$ & $6$ & $3/8$ \rule{0pt}{2ex}\\
                \hline 
                $3$ & $1/3$ & $5$ & $5/16$ \rule{0pt}{2ex}\\
                \hline 
                $4$ &$2/3$ & $4$ & $1/4$ \rule{0pt}{2ex}\\
                \hline 
                $5$ & $1$ & $1$ & $1/16$ \rule{0pt}{2ex}\\
                \hline 
            \end{tabular}
        }} % Fine \vcenter{\hbox{...
        \end{aligned}
    \end{equation*}
    
    \captionof{figure}{Calcolo dell'istogramma normalizzato}
    \label{fig:iji}
\end{center}
\begin{equation*}
    s_2=p_{in}(r_2)=\frac{3}{8}\to0
\end{equation*}
\begin{equation*}
    s_3=p_{in}(r_3)=\frac{3}{8}+\frac{5}{16}=\frac{11}{16}\to\frac{2}{3}
\end{equation*}
\begin{equation*}
    s_4=p_{in}(r_4)=\frac{3}{8}+\frac{5}{16}+\frac{1}{4}=\frac{15}{16}\to1
\end{equation*}
\begin{equation*}
    s_5=p_{in}(r_5)=\frac{3}{8}+\frac{5}{16}+\frac{1}{4}+\frac{1}{16}=1\to1
\end{equation*}
Perciò:
\begin{itemize}
    \item al livello $2$ si associa il livello $2$;
    \item al livello $3$ si associa il livello $4$;
    \item al livello $4$ si associa il livello $5$;
    \item al livello $5$ si associa il livello $5$.
\end{itemize}
A questo punto, si ottiene l'immagine equalizzata, come mostrato di seguito.
\begin{center}
    % Uso \begin{equation*} e \end{equation*} per contenere l'ambiente align*
    \begin{equation*}
        \begin{aligned}[t] 
        & 
            g(x,y) = \begin{bmatrix}
            2 & 4 & 4 & 2 \\
             5 & 2 & 5 & 4 \\
             4 & 2 & 4 & 5 \\
              2 & 5 & 2 & 5
            \end{bmatrix}
        &
        \raisebox{-0.5\height}{\includegraphics[width=0.45\textwidth]{cap01/istogramma2.png}} 
        \end{aligned}
    \end{equation*}
    \captionof{figure}{Esempio di istogramma di un'immagine equalizzata} % <--- NOTA: Ho usato \captionof (richiede \usepackage{caption})
    \label{fig:kjmkp}
\end{center}
\vfill