In questo capitolo, viene trattato come avviene la percezione dello spazio tridimensionale, partendo solamente da un'immagine, che invece possiede
solamente due dimensioni. Per prima cosa, è necessiario comprende il funzionamento del nervo oculomotore.
Per prima cosa, avviene l'accomodamento, in cui il cristallino si comporta come una lente convergente di distanza focale variabile. Inoltre, il muscolo
ciliare si contrae e si allunga, variando forma e quindi la distanza focale.
Perciò, nel momento in cui si guardano oggetti lontani, il cristallino si appiattisce, mentre quando si osservano oggetti vicini diventa più spesso.
\begin{figure}[htbp]
        \centering
        \includegraphics[width=0.9\textwidth]{cap7/adattamento} 
        \caption{Processo di adattamento} 
        \label{fig:accomodamento}
\end{figure}
In contemporanea, avviene la cosiddetta vergenza, che non è altro che un movimento simultaneo degli occhi in direzioni opposte, che consentono la fusione.
In particolare:
\begin{itemize}
    \item gli occhi ruotano uno verso l'alto per guardare oggetti vicini, detta convergenza;
    \item gli occhi ruotano in direzioni opposte per guardare oggetti lontani, denominata divergenza.
\end{itemize}
\begin{figure}[htbp]
        \centering
        \includegraphics[width=0.6\textwidth]{cap7/vergenza} 
        \caption{Processo di vergenza} 
        \label{fig:vergenza}
\end{figure}
\section{Indicatori monoculari di profondità statici}
In questo paragrafo, vengono trattati i cosiddetti indicatori monoculari di profondità statici, cioè quelle caratteristica che forniscono delle
informazioni tridimensionali in immagini. Come schema, conviene analizzare il dipinto \textit{A Rainy Day in Paris}.
\begin{figure}[htbp]
        \centering
        \includegraphics[width=0.9\textwidth]{cap7/quadro} 
        \caption{\textit{A Rainy Day in Paris} che mostra le occlusioni (1), le dimensioni relative (2), le tessiture (3), la prospettiva lineare (4), la prospettiva aerea (5) e le ombre (6)} 
        \label{fig:quadro}
\end{figure}
\subsection{Occlusioni}
Il primo indicatore sono sicuramente le occlusioni, che si definiscono come gli oggetti più vicini che bloccano l'accesso visivo a quello più distanti.
Per esempio, nella figura seguente viene mostrata una ragazza che viene parzialmente coperta da un tronco di una palma.
\begin{figure}[htbp]
        \centering
        \includegraphics[width=0.8\textwidth]{cap7/occlusioni} 
        \caption{Esempio di occlusione} 
        \label{fig:occlusione}
\end{figure}
\subsection{Dimensioni relative}
Le dimensioni relative forniscono informazioni 3D, grazie ad oggetti della stessa dimensione fisica proiettano immagini retiniche di dimensione diversa
a seconda del punto del punto di osservazione. Per esempio, è molto più facile conoscere l'altezza di una statua dell'isola di Pasqua con delle persone
(che hanno un'altezza media di 1,80 m), rispetto a che non ci sono persone od altri oggetti di riferimento: infatti, anche l'esperienza gioca 
un ruolo fondamentale in questo contesto.
\begin{figure}[htbp]
        \centering
        \includegraphics[width=0.8\textwidth]{cap7/relative} 
        \caption{Esempio di dimensioni relative} 
        \label{fig:relative}
\end{figure}
\vfill