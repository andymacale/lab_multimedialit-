Il processo di compressione di un'immagine, consiste nel ridurre drasticamente la quantità di dati presenti nell'immagine, poiché trasmettere un'immagine
non compressa richiede molto tempo e molta larghezza di banda, come mostrato nell'immagine seguente.
\begin{figure}[htbp]
        \centering
        \includegraphics[width=0.9\textwidth]{cap5/compressione} 
        \caption{Processo di trasmissione di un'immagine compressa}
        \label{fig:hansform}
\end{figure}
Basti pensare che già un'immagine $1024\times1024$ in scala di grigi con risoluzione bassa ($8 \, b/p$), per essere trasmessa in 2G ($v=50 \, kb/s $), sono necessari:
\begin{center}
\begin{minipage}{0.48\textwidth} % Colonna Sinistra (48% della larghezza del testo)
    \centering
    \begin{equation*}
    c=1024\times1024\times8=8388608 \, b
\end{equation*}
\end{minipage}
\hfill % Spazio elastico per separare le due colonne
\begin{minipage}{0.48\textwidth} % Colonna Destra (48% della larghezza del testo)
    \centering
        \begin{equation*}
    b=\frac{c}{v}=\frac{8388608}{50000} = 167,77216 \, s \approx 3'
    \end{equation*}
\end{minipage}
\end{center}
che è decisamente troppo elevato.
\section{Ridondanza dei dati}
Per prima cosa è necessario ribadire che dato ed informazione non sono assolutamente sinonimi. Infatti:
\begin{itemize}
    \item un'informazione è ciò che si vuole trasmettere;
    \item un dato rappresenta i modi di trasmettere l'informazione.
\end{itemize}
La ridondanza dei dati si può quantificare matematicamente come entità. Infatti, si considerano le seguenti grandezze:
\begin{itemize}
    \item rapporto di compressione ($C_R$), che è il rapporto tra due insiemi di dati che contengono la stessa informazione;
    \begin{equation*}
        C_{R} = \frac{n_1}{n_2}
    \end{equation*}
    \item ridondanza relativa dei dati ($R_D$), che indica la quantità dei dati che sono ridondanti, spesso indicati in percentuale.
    \begin{equation*}
        R_D = 1-\frac{1}{C_R}
    \end{equation*}
\end{itemize}
In questo paragrafo, verranno trattate le diverse tipologie di rindondanza.
\subsection{Ridondanza di codifica}
La ridondanza di codifica si ha nel momento in cui si sceglie un sistema di codifica non efficiente. Per calcolare l'efficienza di un
algoritmo di compressione, dipende dal livello medio di grigio, che dipende dalla probabilità che ogni livello di grigio dell'immagine
sia presente in tale immagine e quanti bit occupa ($l(r_k)$).
\begin{center}
\begin{minipage}{0.48\textwidth} % Colonna Sinistra (48% della larghezza del testo)
    \centering
    \begin{equation*}
        p_r(r_k)=\frac{n_k}{MN}
    \end{equation*}
\end{minipage}
\hfill % Spazio elastico per separare le due colonne
\begin{minipage}{0.48\textwidth} % Colonna Destra (48% della larghezza del testo)
    \centering
    \begin{equation*}
        L_{avg}=\sum_{0}^{L-1}l(r_k)p_r(r_k)
    \end{equation*}
\end{minipage}
\end{center}
Per esempio data un'immagine $256 \times 256$ con le seguenti caratteristiche:
\begin{table}[H] % Usa [H] per renderla non flottante
    \centering
    \begin{tabular}{| C{5cm} | C{5cm} |} 
            \hline
            
            % Usiamo \thead per le intestazioni:
            \textbf{$r_k$} & \textbf{$p_k(r_k)$} \\ 
            \hline % Linea spessa sotto le intestazioni
            $r_{87}=87$ & $3/10$ \\
            \hline % Linea spessa sotto le intestazioni
            $r_{128}=128$ & $1/2$ \\
            \hline % Linea spessa sotto le intestazioni
            $r_{186}=186$ & $1/10$ \\
            \hline % Linea spessa sotto le intestazioni
            $r_{255}=255$ & $1/10$ \\
            \hline % Linea spessa sotto le intestazioni
            $r_{k} \, se \, k \neq 87, 128, 186, 255$ & ${0}$ \\
            \hline
    \end{tabular}
\end{table}
per esempio si usa per ogni livello di grigio il numero massimo rappresentabile del livello di grigio più alto ($255$), che è $8$ ($\log_{2}{(255)}$): Il
livello medio di grigio è $L_{avg_{1}}=8\, b$. A questo punto conviene assegnare il numero di bit a seconda della probabilità: maggiore è la probabilità,
meno bit deve avere quel livello di grigio, così lo spazio occupato sarebbe decisamente minore.
\begin{table}[H] % Usa [H] per renderla non flottante
    \centering
    \begin{tabular}{| C{5cm} | C{5cm} | C{5cm}|} 
            \hline
            
            % Usiamo \thead per le intestazioni:
            \textbf{$r_k$} & \textbf{$p_k(r_k)$} & \textbf{$l_k(r_k)$} \\ 
            \hline % Linea spessa sotto le intestazioni
            $r_{87}=87$ & $3/10$ & $2$\\
            \hline % Linea spessa sotto le intestazioni
            $r_{128}=128$ & $1/2$ & $1$\\
            \hline % Linea spessa sotto le intestazioni
            $r_{186}=186$ & $1/10$ & $3$\\
            \hline % Linea spessa sotto le intestazioni
            $r_{255}=255$ & $1/10$ & $3$ \\
            \hline % Linea spessa sotto le intestazioni
    \end{tabular}
\end{table}
\begin{equation*}
    L_{avg_{2}}=\frac{3}{10}\times 2 + \frac{1}{2}\times 1 + \frac{1}{10} \times 3 + \frac{1}{10} \times 3 = \frac{17}{10} \, b = 1,7\, b
\end{equation*}
Infine, si calcolano le due grandezze.
\begin{center}
\begin{minipage}{0.48\textwidth} % Colonna Sinistra (48% della larghezza del testo)
    \centering
    \begin{equation*}
        C_R = \frac{8\times10}{17}=\frac{80}{17}
    \end{equation*}
\end{minipage}
\hfill % Spazio elastico per separare le due colonne
\begin{minipage}{0.48\textwidth} % Colonna Destra (48% della larghezza del testo)
    \centering
    \begin{equation*}
        R_D = 1 - \frac{17}{80} = \frac{63}{80} = 78,75\%
    \end{equation*}
\end{minipage}
\end{center}
\subsection{Ridondanza interpixel}
La ridondanza interpixel implica che qualsiasi valore di un pixel può essere predetto dai sui vicini, grazie alla correlazione. Per ridurre ciò,
i dati dovrebbere essere mappati.
\begin{figure}[htbp]
        \centering
        \includegraphics[width=0.4\textwidth]{cap5/interpixel} 
        \caption{Esempio di ridondanza di interpixel}
        \label{fig:interpixel}
\end{figure}
\subsection{Ridondanza psicovisiva}
La ridondanza psicovisiva sta nel fatto che alcuni pixel sono talmente simili che l'occhio umano non li percepisce, dato che il sistema visivo umano
non percepisce tutta l'informazione visiva con la stessa intensità: ma cerca solo caratteristiche importanti, come angoli e texture. Il classico esempio
è un immagine a tinta unita.
\begin{figure}[htbp]
        \centering
        \includegraphics[width=0.6\textwidth]{cap5/psico} 
        \caption{Esempio di ridondanza psicovisiva}
        \label{fig:psicovisiva}
\end{figure}