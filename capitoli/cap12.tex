In questo capitolo viene spiegato il funzionamento dei multimedia.
\section{Introduzione ai multimedia}
L'audio si può dividere in naturale ed elettronici. Gli audio elettronici sono ciò che interessa nel corso.
\subsection{Audio elettronico}
L'audio elettronico comprende:
\begin{itemize}
    \item sorgenti sonore ($20\,\text{Hz}\,-\,20\,\text{kHz}$);
    \item amplificazione e condizionamento del segnale;
    \item trasduttore elettroacustico;
    \item elaborazione per l'applicazione prevista;
    \item mezzi di trasmissione;
    \item archiviazione;
    \item estrazione e misurazione delle informazioni;
    \item riproduzione.
\end{itemize}
\subsection{Aspetti integrativi del multimedia}
Il sistema del multimedia viene suddiviso in:
\begin{itemize}
    \item basi, che includono le architetture informatiche e soprattutto la compressione;
    \item sistemi, che comprende ad esempio il database ed il media server;
    \item servizi, che comprende l'elaborazione dei contenuti, sincronizzazione e comunicazione di gruppo;
    \item utilizzo, che riguarda applicazioni come l'interfaccia utente.
\end{itemize}
Per quanto riguarda i multimedia audio, ne fanno parte:
\begin{itemize}
    \item la cattura audio;
    \item la rappresentazione delle informazioni audio;
    \item la trasmissione;
    \item l'elaborazione;
    \item l'archiviazione;
    \item la riproduzione.
\end{itemize}
\begin{figure}[htbp]
        \centering
        \includegraphics[width=0.9\textwidth]{cap12/aspetti} 
        \caption{Esempio di aspetto integrativo di un multimedia audio} 
        \label{fig:prog}
\end{figure}
\section{Audio sintetico, analisi e sintesi del discorso}
Il modello generale del parlato suggerisce che i suoni del parlato, si possono analizzare determinando gli stati delle componenti del parlato 
per ogni fonema. In particolare, si fa riferimento al:
\begin{itemize}
    \item parlato vocale, se è presente aria pulsata;
    \item parlato non vocale, se è presente solamente flusso continuo oppure raffica d'aria.
\end{itemize}
\begin{figure}[H]
        \centering
        \includegraphics[width=0.7\textwidth]{cap12/rec} 
        \caption{Registrazione vocale} 
        \label{fig:messaggio_vocale}
\end{figure}
A questo si esegue l'analisi spettrale del parlato vocale, che è in grado di rilevare l'inviluppo spettrale, che ne indica che la caratteristiche
fondamentali del tratto vocale, detti formanti; ed i pitch delle corde vocali, denominati picchi armonici. Inoltre, si esegue l'analisi temporale per
rilevare l'inviluppo temporale, che indica le dinamiche del volume; e le oscillazioni, che indicano il movimento delle corde vocali. Per visualizzare
l'andamento delle frequenze nel tempo, si usa il cosiddetto spettrogramma, che mostra le variazioni di intensità di diverse componenti delle frequenze. 
\begin{figure}[H]
        \centering
        \includegraphics[width=0.7\textwidth]{cap12/spettrogramma} 
        \caption{Esempio di spettrogramma} 
        \label{fig:spettrogramma}
\end{figure}
Infine, per quanto riguarda il riconoscimento vocale, può variare in base a:
\begin{itemize}
    \item alla dipendenza o meno del parlante;
    \item riconoscimento di parlato isolato, continuo o spot;
    \item dimensione del vocabolario, complessità delle grammatica e stile del parlato;
    \item condizione di registrazione.
\end{itemize}
Le componenti del riconoscimento vocale includono:
\begin{enumerate}
    \item parlato in ingresso;
    \item trasduzione del parlato;
    \item front-end;
    \item corrispondenza locale;
    \item modello linguistico del rilevatore globale;
    \item stringa di parlato rilevata.
\end{enumerate}
\begin{figure}[htbp]
        \centering
        \includegraphics[width=0.4\textwidth]{cap12/google} 
        \caption{Icona del riconoscimento vocale Google} 
        \label{fig:google}
\end{figure}
\vfill