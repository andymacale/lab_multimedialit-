In questo capitolo, viene ripreso il concetto di stereoscopia, applicato però ai video: perciò è necessario avere delle competenze dello stereoscopia
applicate alle immagini.
\section{Indicatori monoculari di profondità dinamici}
Gli indicatori monoculari di profondità dinamici sono segnali visivi che il nostro sistema percettivo utilizza con un solo occhio per percepire la
profondità e la distanza degli oggetti; richiedendo un movimento od un cambiamento di scena.
\subsection{Parallasse di movimento}
Il parallasse di movimento afferma che oggetti che si trovano:
\begin{itemize}
    \item più in lontanza sembrano che si muovono nella stessa direzione dell'osservatore;
    \item più vicini sembrano che si muovono nella direzione opposta dell'osservatore.
\end{itemize}
Per esempio, mentre si osserva dalla finestrino di un treno un paesaggio di montagna:
\begin{itemize}
    \item se si guarda un'albero a pochi metri di distanza, sembra muoversi nella direzione opposta;
    \item se si guardano le montagne sullo sfondo, sembrano muoversi nella stessa direzione.
\end{itemize}
\begin{figure}[htbp]
        \centering
        \includegraphics[width=0.6\textwidth]{cap09/montagna} 
        \caption{Esempio pratico del funzionamento del parallasse di movimento} 
        \label{fig:montagne}
\end{figure}
\subsection{Velocità angolare relativa}
La velocità angolare relativa descrive quanto velocemente un corpo ruota, misurata da un osservatore che sta ruotando anche lui. In particolare:
\begin{itemize}
    \item l'osservatore fisso misura la rotazione totale del corpo;
    \item l'osservatore girevole vede solamente la rotazione del corpo in eccesso od in difetto rispetto alla sua rotazione.
\end{itemize}
Un classico esempio è un genitore che vede il proprio figlio girare sull'altalena girevole: il genitore, che è l'osservatore fisso, vede la velocità angolare assoluta del figlio,
mentre il figlio, che è l'osservatore girevole, vede gli oggetti immobile che girano attorno a lui, mentre la giostra immobile.
\begin{figure}[htbp]
        \centering
        \includegraphics[width=0.5\textwidth]{cap09/giostra} 
        \caption{Esempio pratico del funzionamento della velocità angolare relativa} 
        \label{fig:giostra}
\end{figure}
\subsection{Espansione radiale}
L'espansione radiale è l'effetto che si verifica quando un osservatore si muove in avanti oppure al contrario quando un oggetto si muove verso 
l'osservatore, le immagini degli oggetti sulla retina si allargano o si espandono in modo centrifugo a partire dal punto centrale, eccetto per un punto
centrale, detto centro di espansione, in cui non si muove nulla dalla retina.\\
Un esempio pratico è quando si guida in autostrada di notte: le strisce bianche sembrano nascere dal centro di espansione, mentre gli altri oggetti sembrano allontanarsi dal centro di espansione.
\begin{figure}[htbp]
        \centering
        \includegraphics[width=0.6\textwidth]{cap09/autostrada} 
        \caption{Esempio pratico del funzionamento dell'espansione radiale} 
        \label{fig:autostrada}
\end{figure}
\subsection{Movimento delle ombre}
Per quanto riguarda il movimento delle ombre, è una combinazione di semplici principi fisici e fenomeni astronomici. Un'ombra si muove perché la 
sorgente luminosa, l'oggetto che la proietta oppure la superficie che la riceve sono in movimento. Tale movimento può aiutare il cervello a 
processare la direzione del movimento dell'oggetto o della sorgente luminosa.
\section{Tecnologie per il video 3D}
Adesso si entra nel cuore della tridimensionalità, trattando il funzionamento del video 3D, esploso negli anni 2010, parlando sia del cinema 3D, ma
anche delle TV 3D.
\subsection{Dolby 3D}
Il Dolby 3D è una tecnologia del cinema 3D che funziona grazie a proiettori con un refresh di:
\begin{itemize}
    \item $96 \, \text{Hz}$, in cui garantisce un refresh di $48 \, \text{Hz}$ per occhio, comportando allo spettatore una percezione di immagine instabile;
    \item $144 \, \text{Hz}$ in cui garantisce un refresh di $72 \, \text{Hz}$ per occhio, comportando allo spettatore una visione più piacevole e meno affaticante.
\end{itemize}
Tuttavia, per funzionare correttamente si devono indossare gli occhiali Dolby 3D, in cui sono composti da due lenti passive non polarizzate e colorate
liviemente, in questo modo viene filtrata più luce e si ha un decadimento della qualità cromatica dell'immagine.
\subsection{Introduzione alla luce polarizzata e Real D}
A questo punto, viene usata la luce polarizzata, in cui due proiettori inviano sullo schermo le immagini di sinistra ed immagini di destra, in cui i 
spettatori devono essere dotati di occhiali con lenti polarizzate, in modo tale da filtrare uno dei due fasci luminosi: così, ciascun occhio vede 
solamente uno dei due segnali. Inoltre, ciascun proiettore è dotato di un filtro che polarizza la luce per far sì che i due segnali luminosi siano
polarizzati in modo ortogonale tra loro. Uno schermo che fa ciò è il cosiddetto Silver Screen, che mantiene la polarizzazione delle immagini proiettate
ed allo stesso tempo compensa la perdita di luce.\\
La prima implementazione avviene con il cinema Real D, in cui possono avvenire due tipologie di polarizzazione:
\begin{itemize}
    \item polarizzazione lineare, in cui due immagini parallele vengono proiettate e sovrapposte sullo schermo attraverso dei filtri polarizzatori
            ortogonali, con l'effetto indesiderato della visione sfocata nel caso in cui lo spettatore inclini la testa;
    \item polarizzazione circolare, in cui due immagini vengono sovrapposte sullo stesso schermo circolare, attraverso dei filtri di polarizzazione
            opposta, che risolve il problema della polarizzazione lineare.
\end{itemize}
\subsection{Sistemi LCD}
I sistemi LCD usano degli occhiali con filtri a cristalli liquidi (LCD) alimentati a batterie, che sono in grado di lavorare con il proiettore. Questi
occhiali sono attrezzati con due lenti LCD sincronizzati con un filtro infrarosso generato dal sistema di proiezione che oscura un LCD in maniera
alternata, facendo da otturatore.
\subsection{Cinema 3D più avanzati: Xpand3D ed High Frame Rate 3D}
Il cinema Xpand3D presenta l'enorme vantaggio di non richiedere un silverscreen: in questo modo si ha un'intensità luminosa omogenea in qualsiasi posizione
della sala. Grazie a questo vantaggio, l'intera sala viene costantemente illuminata da un segnale infrarossi, che permette di sincronizzare gli occhiali
Xpand con il comando toggle, in cui oscura le lenti LCD in maniera alternata.\\
Tuttavia, la tecnologia migliore è l'High Frame Rate 3D, in cui presenta un frame rate tale da non percepire sfarfallii ed artefatti a
$30 \, \text{fps}$ fino frame non distinguibili come separati dall'occhio umano dai $55 \, \text{fps}$, fino ad arrivare a $120 \, \text{fps}$.
\subsection{TV 3D}
Le TV 3D si dividono in due categorie principali.\\
La prima categoria contiene le TV con il 3D attivo, in cui lo spettatore indossa degli occhiali con otturatori attivi con lenti a cristalli liquidi, che si
oscurano in presenza di un segnale elettrico. Inviando una sequenza di tali segnali in modo sincrono con la riproduzione dei fotogrammi sullo schermo,
la frequenza deve superare la persistenza della visione sulla retina, in modo tale che ciascun occhio veda soltanto il fotogramma a lui destinato. Per
questo motivo, si rende necessario aumentare il numero di fotogrammi inviati almeno del doppio ($100-120 \, \text{Hz}$). Infine, la TV deve anche disporre
di un emittitore wireless che invii agli occhiali un segnale di sincronizzazione. Tale tecnologia tende ad avere immagine piuttosto scure\\
La seconda ed ultima categoria appartiene alle TV con il 3D passivo, che si basa su occhiali polarizzati come nel cinema. A questo punto, la TV è dotata
di un particolare filtro che polarizza ogni riga di pixel, con l'obiettivo di mostrare le righe dispari ad un occhio e le righe pari all'altro occhio:
infatti, ciascun fotogramma viene scomposto in righe verticali, dove metà sono visibili solo da un occhio e l'altra metà solamente dall'altro occhio.
Tuttavia, tale tecnologia porta ad un riduzione della risoluzione, poiché ogni occhio percepisce 1920x540 (Full HD), ma gli occhiali sono
decisamente meno fastidiosi.\\
Infine, in entrambe le tecnologie hanno un problema di non adattabilità per uso prolungato.
\section{Tecnologie di visualizzazione avanzata: l'autostereoscopia}
Le tecnologie di visualizzazione avanzata permettono di dare l'impressione della tridimensionalità, senza l'utilizzo di occhiali, implementando la
cosiddetta autostereoscopia. Questi schermi, solamente di tipo LCD, permettono il funzionamento sia in modalità 2D che in 3D.
\begin{figure}[htbp]
        \centering
        \includegraphics[width=0.9\textwidth]{cap09/3ds} 
        \caption{Esempio pratico di schermo autostereoscopico: serie Nintendo 3DS} 
        \label{fig:3ds}
\end{figure}
\subsection{Schermi a microlenti}
Gli schermi a microlenti possiedono un pannello a LCD od al plasma con una lamina trasparente composta da lenti semicilindriche o semisferiche. La
superficie del pannello coincide con il piano focale delle lenti, che operano una diffrazione della luce, inviata nella direzione voluta. In questo
modo gli occhi vedono due insieme di pixel differente per ogni microlente. Infine, per ottenere la visione tridimensionale, si associa ad ogni insieme
una coppia stereoscopica.\\
Tali schermi hanno il vantaggio di avere la possibilità di un sistema multi-view, a discapito della riduzione significativa della risoluzione, poiché
viene divisa da ciascun occhio per il numero di viste.
\begin{figure}[htbp]
        \centering
        \includegraphics[width=0.8\textwidth]{cap09/microlenti} 
        \caption{Schermi a microlenti} 
        \label{fig:microlenti}
\end{figure}
\subsection{Schermi a barriera di parallasse}
La barriera di parallasse è costituita da un pannello elettro-ottico con fessure sottili verticali equidistanti fra loro, che viene applicata sopra al
display LCD.  A questo punto, le immagini di destra e di sinistra vengono divise in strisce verticali, ognuna associata ad una colonna di pixel del
pannello LCD. Inoltre, la barriera rende visibile solo la componente stereoscopia relativa ad un singolo occhio, mascherando la componente dell'altro
occhio. Infine, presenta come vantaggi la buona qualità dell'immagine e l'assenza di sfarfallii; tuttavia ha una luminosità ridotta, il dimezzamento
della definizione orizzontale e la necessità dell'osservatore di porsi ad una distanza opportuna ed entro un certo angolo visuale.
\begin{figure}[htbp]
        \centering
        \includegraphics[width=0.8\textwidth]{cap09/barriera} 
        \caption{Schermi a barriera di parallasse} 
        \label{fig:barriera_di_parallasse}
\end{figure}
\vfill