In questo capitolo, viene ripreso il concetto di stereoscopia, applicato però ai video: perciò è necessario avere delle competenze dello stereoscopia
applicate alle immagini.
\section{Indicatori monoculari di profondità dinamici}
Gli indicatori monoculari di profondità dinamici sono segnali visivi che il nostro sistema percettivo utilizza con un solo occhio per percepire la
profondità e la distanza degli oggetti; richiedendo un movimento od un cambiamento di scena.
\subsection{Parallasse di movimento}
Il parallasse di movimento afferma che oggetti che si trovano:
\begin{itemize}
    \item più in lontanza sembrano che si muovono nella stessa direzione dell'osservatore;
    \item più vicini sembrano che si muovono nella direzione opposta dell'osservatore.
\end{itemize}
Per esempio, mentre si osserva dalla finestrino di un treno un paesaggio di montagna:
\begin{itemize}
    \item se si guarda un'albero a pochi metri di distanza, sembra muoversi nella direzione opposta;
    \item se si guardano le montagne sullo sfondo, sembrano muoversi nella stessa direzione.
\end{itemize}
\begin{figure}[htbp]
        \centering
        \includegraphics[width=0.6\textwidth]{cap09/montagna} 
        \caption{Esempio pratico del funzionamento del parallasse di movimento} 
        \label{fig:montagne}
\end{figure}
\subsection{Velocità angolare relativa}
La velocità angolare relativa descrive quanto velocemente un corpo ruota, misurata da un osservatore che sta ruotando anche lui. In particolare:
\begin{itemize}
    \item l'osservatore fisso misura la rotazione totale del corpo;
    \item l'osservatore girevole vede solamente la rotazione del corpo in eccesso od in difetto rispetto alla sua rotazione.
\end{itemize}
Un classico esempio è un genitore che vede il proprio figlio girare sull'altalena girevole: il genitore, che è l'osservatore fisso, vede la velocità angolare assoluta del figlio,
mentre il figlio, che è l'osservatore girevole, vede gli oggetti immobile che girano attorno a lui, mentre la giostra immobile.
\begin{figure}[htbp]
        \centering
        \includegraphics[width=0.5\textwidth]{cap09/giostra} 
        \caption{Esempio pratico del funzionamento della velocità angolare relativa} 
        \label{fig:giostra}
\end{figure}
\subsection{Espansione radiale}
L'espansione radiale è l'effetto che si verifica quando un osservatore si muove in avanti oppure al contrario quando un oggetto si muove verso 
l'osservatore, le immagini degli oggetti sulla retina si allargano o si espandono in modo centrifugo a partire dal punto centrale, eccetto per un punto
centrale, detto centro di espansione, in cui non si muove nulla dalla retina.\\
Un esempio pratico è quando si guida in autostrada di notte: le strisce bianche sembrano nascere dal centro di espansione, mentre gli altri oggetti sembrano allontanarsi dal centro di espansione.
\begin{figure}[htbp]
        \centering
        \includegraphics[width=0.6\textwidth]{cap09/autostrada} 
        \caption{Esempio pratico del funzionamento dell'espansione radiale} 
        \label{fig:autostrada}
\end{figure}
\subsection{Movimento delle ombre}
Per quanto riguarda il movimento delle ombre, è una combinazione di semplici principi fisici e fenomeni astronomici. Un'ombra si muove perché la 
sorgente luminosa, l'oggetto che la proietta oppure la superficie che la riceve sono in movimento. Tale movimento può aiutare il cervello a 
processare la direzione del movimento dell'oggetto o della sorgente luminosa.
\vfill