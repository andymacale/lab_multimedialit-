In questo capitolo, viene ripreso il concetto di stereoscopia, applicato però ai video: perciò è necessario avere delle competenze dello stereoscopia
applicate alle immagini.
\section{Indicatori monoculari di profondità dinamici}
Gli indicatori monoculari di profondità dinamici sono segnali visivi che il nostro sistema percettivo utilizza con un solo occhio per percepire la
profondità e la distanza degli oggetti; richiedendo un movimento od un cambiamento di scena.
\subsection{Parallasse di movimento}
Il parallasse di movimento afferma che oggetti che si trovano:
\begin{itemize}
    \item più in lontanza sembrano che si muovono nella stessa direzione dell'osservatore;
    \item più vicini sembrano che si muovono nella direzione opposta dell'osservatore.
\end{itemize}
Per esempio, mentre si osserva dalla finestrino di un treno un paesaggio di montagna:
\begin{itemize}
    \item se si guarda un'albero a pochi metri di distanza, sembra muoversi nella direzione opposta;
    \item se si guardano le montagne sullo sfondo, sembrano muoversi nella stessa direzione.
\end{itemize}
\begin{figure}[htbp]
        \centering
        \includegraphics[width=0.5\textwidth]{cap09/montagna} 
        \caption{Esempio pratico del funzionamento del parallasse di movimento} 
        \label{fig:montagne}
\end{figure}
\subsection{Velocità angolare relativa}
La velocità angolare relativa 
\vfill